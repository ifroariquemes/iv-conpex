\documentclass[article,12pt,onesidea,4paper,english,brazil]{abntex2}

\usepackage{lmodern, indentfirst, color, graphicx, microtype, lipsum}			
\usepackage[T1]{fontenc}		
\usepackage[utf8]{inputenc}		

\setlrmarginsandblock{2cm}{2cm}{*}
\setulmarginsandblock{2cm}{2cm}{*}
\checkandfixthelayout

\setlength{\parindent}{1.3cm}
\setlength{\parskip}{0.2cm}

\SingleSpacing

\begin{document}
	
	\selectlanguage{brazil}
	
	\frenchspacing 
	
	\begin{center}
		\LARGE PRÁTICA DE VEGETAIS MINIMAMENTE PROCESSADOS INTEGRADA À ATIVIDADE PEDAGÓGICA NA AGROINDÚSTRIA DO IFRO CAMPUS COLORADO DO OESTE -- RO\footnote{Trabalho realizado dentro de práticas exitosas no ensino.}
		
		\normalsize
		Donizete Alves de Lima Júnior\footnote{Discente do curso Técnico em Alimentos donny\_alves@hotmail.com, Campus Colorado do Oeste.} 
		Josiane Capellaro Varela\footnote{Discente do curso Técnico em Alimentos, josy\_capellaro@hotmail.com, Campus Colorado do Oeste.} 
		Nélio Ranieli Ferreira de Paula \footnote{Orientador e professor, nelio.ferreira@ifro.edu.br, Campus Colorado do Oeste.} 
	\end{center}
	
	\noindent Durante o ensino prático de Processamento Mínimo de Vegetais, realizado na disciplina de Tecnologia de Alimentos, Hortaliças e Bebidas, foi possível a percepção da importância deste método para o atual mercado consumidor, uma vez que esta técnica procura atender adequadamente aos requisitos contemporâneos de saúde, praticidade e segurança, encontrados nos parâmetros dos 5 S. Desta forma, o objetivo geral desta pratica foi promover a fixação de conhecimento técnico-científico e pedagógico no laboratório agroindustrial no Campus Colorado do Oeste - RO. Durante este ensino prático, foram realizados vários processos metodológicos, como: recepção de vegetais (matéria prima), seleção, higienização para a retirada de resíduos orgânicos, a sanitização com hipoclorito de sódio, descascamento, corte,enxágue, embalagem e armazenamento. Para o desenvolvimento desta metodologia foi aplicada as boas práticas de fabricação e manipulação na elaboração de vegetais minimamente processados agregando conveniência e praticidade. Da mesma forma, foi utilizado outros materiais específicos, como: álcool 70\%, facas, bacias, tábuas de corte, medidores precisos para água e solução, balança de precisão e embalagens em geral.  Os resultados obtidos foram de 100\% de aprendizagem, pois, ficou evidenciado que o controle microbiológico dos produtos minimamente processados envolve fatores tais como: qualidade da matéria prima, condições de processamento, de embalagem, de armazenamento, e manejo sanitário correto ao longo da cadeia de produção.
	
	\vspace{\onelineskip}
	
	\noindent
	\textbf{Palavras-chave}: Processamento. Alimentos. Aprendizagem.
	
\end{document}
