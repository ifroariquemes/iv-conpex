\documentclass[article,12pt,onesidea,4paper,english,brazil]{abntex2}

\usepackage{lmodern, indentfirst, color, graphicx, microtype, lipsum}			
\usepackage[T1]{fontenc}		
\usepackage[utf8]{inputenc}		

\setlrmarginsandblock{2cm}{2cm}{*}
\setulmarginsandblock{2cm}{2cm}{*}
\checkandfixthelayout

\setlength{\parindent}{1.3cm}
\setlength{\parskip}{0.2cm}

\SingleSpacing

\begin{document}
	
	\selectlanguage{brazil}
	
	\frenchspacing 
	
	\begin{center}
		\LARGE PROTÓTIPO DE UM CARRO MOVIDO\\A ENERGIA SOLAR\footnote{Área do conhecimento CNPQ/CAPES: Ciências Exatas e da Terra.}
		
		\normalsize
		Marisa Rodrigues Lima\footnote{Orientadora, marisa.rodrigues@ifro.edu.br, Campus Vilhena.} 
		Marlene Urup Tossue\footnote{Colaboradora,marleneurup@hotmail.com, Campus Vilhena.} 
		Thais Camila\footnote{Colaboradora, thaiis3@hotmail.com, Campus Vilhena.} 
		Eldina Lopes dos Santos\footnote{Colaboradora, d.nal@hotmail.com, CampusVilhena} 
	\end{center}
	
	\noindent Este trabalho versa sobre o estudo e a confecção de um protótipo de carro elétrico movido à energia solar. Nosso objetivo é demostrar de forma prática o funcionamento dos painéis solares para produção de energia elétrica e sua aplicação na forma de energia mecânica. Esse carro elétrico transforma a energia proveniente do sol por meio dos painéis em energia elétrica que por sua vez coloca o motor em movimento. Objetivou-se também mostrar à comunidade escolar a possibilidade do uso de um automóvel que não libera poluição na atmosfera, uma solução para a minimização de poluentes que prejudica não só a saúde do ser humano como também a preservação do ambiente. Para a confecção do carro foram utilizados materiais simples e reaproveitados como: papelão cola quente, palito de madeira (churrasco), motor elétrico de trava de porta de carro, placa voltaica, tinta, canudo e engrenagem tirado de toca fita. O trabalho foi desenvolvido em grupo de trabalho para cumprir créditos da disciplina de física III, como prática de laboratório. A metodologia utilizada deu-se em três etapas: na primeira etapa foi feito uma pesquisa de modelos de carros elétricos; na segunda etapa, já com o modelo selecionado e com os materiais em mãos foi construído o protótipo como está descrito a seguir: com os palitos foi feito os eixos das rodas que foram colocados dentro de um pedaço de canudinho de refrigerante para reduzir o atrito; em seguida foi colocado o motor elétrico no assoalho do carrinho e encaixado a engrenagem de modo que ocorresse a tração na roda traseira; foi instalada a placa voltaica no teto do carro e em seguida foi conectado o motor elétrico, assim quando colocado o mesmo no sol a energia solar é captada pela placa e faz com que acione o motor elétrico, fazendo girar a engrenagem e o carrinho se movimenta. Com o desenvolvimento desse trabalho facilitou a compreensão sobre energia solar, placas fotovoltaicas, funcionamento de motores elétricos e ainda sensibilizou os alunos com relação à busca de novas alternativas de energia que não agrida o ambiente.
	
	\vspace{\onelineskip}
	
	\noindent
	\textbf{Palavras-chave}: Física. Eletricidade. Aprendizagem.
	
\end{document}
