\documentclass[article,12pt,onesidea,4paper,english,brazil]{abntex2}

\usepackage{lmodern, indentfirst, color, graphicx, microtype, lipsum}			
\usepackage[T1]{fontenc}		
\usepackage[utf8]{inputenc}		

\setlrmarginsandblock{2cm}{2cm}{*}
\setulmarginsandblock{2cm}{2cm}{*}
\checkandfixthelayout

\setlength{\parindent}{1.3cm}
\setlength{\parskip}{0.2cm}

\SingleSpacing

\begin{document}
	
	\selectlanguage{brazil}
	
	\frenchspacing 
	
	\begin{center}
		\LARGE RAIZ FORTE: UM DRONE DE BAIXO CUSTO\footnote{Trabalho realizado para o Projeto de Ensino “Batalha dos Drones” da disciplina de Redes de Computadores I, 2º ano B do Curso Técnico em Manutenção e Suporte em Informática.}
		
		\normalsize
		Juliane Martinez Galiano\footnote{Orientadora, juliane.martinez@ifro.edu.br, CampusAriquemes.} 
		Rafaella David de Souza\footnote{Colaboradora,rafaellas5639@gmail.com, Campus Ariquemes.} 
		Vitoria Santos Ferreira\footnote{Colaboradora, vick.vs101@gmail.com, Campus Ariquemes.} 
	\end{center}
	
	\noindent Aeronaves de pequeno porte não tripuladas, controladas remotamente, pequena e adequada para voo de observação, por exemplo, em galpões e laboratório. Esse quadrimotor tem propulsão gerada através de quatro motores com dimensões e potência iguais. Dois desses motores e pares de hélices geram propulsão sentido horário, os outros dois pares geram propulsão sentido anti-horário. Essa propulsão de giro nos motores e hélices permite que ele possa se mover para qualquer direção. Os quadrimotores são estáveis e apresentam os mesmos benefícios de um helicóptero também não tripulado, porém tem uma capacidade de carga de voo um pouco menor. A modelagem dessas aeronaves é abordada de duas formas, a primeira com relação nas equações físicas do sistema e a segunda relacionada na identificação de sistema. A utilização de material reciclável para montagem de um drone seria essencial, pois utiliza uma quantidade menor de material como plástico, fibra de carbono e metal ou alumínio, que de alguma forma contaminam e degradam o meio ambiente, levando um tempo muito grande para se decompor e se juntar ao solo. A escolha dos materiais utilizados foi pensando nisso. Então, reduzimos e utilizamos o mínimo possível dos materiais citado, optando por criar uma estrutura toda em papel, reutilizando caixas de chiclete, pois esse material é reciclável e de baixo custo. O resultado final foi um drone que visa à parte ecológica, mas não esquecendo a parte funcional. A robótica usada no drone é apenas a alteração da velocidade de rotação de determinados motores, o que é gerenciado pela placa de controle do Skytech M62R, que apesar de pequena, possui sensores que ajudam a determinar em que posição o quadricóptero se encontra, auxiliando no voo. Determinadores motores aumentam ou diminuem de velocidade para movimentos específicos como ir para direita, esquerda, girar no sentido horário e sentido anti-horário.
	
	\vspace{\onelineskip}
	
	\noindent
	\textbf{Palavras-chave}: Drone. Reciclagem. Baixo custo.
	
\end{document}
