\documentclass[article,12pt,onesidea,4paper,english,brazil]{abntex2}

\usepackage{lmodern, indentfirst, color, graphicx, microtype, lipsum}			
\usepackage[T1]{fontenc}		
\usepackage[utf8]{inputenc}		

\setlrmarginsandblock{2cm}{2cm}{*}
\setulmarginsandblock{2cm}{2cm}{*}
\checkandfixthelayout

\setlength{\parindent}{1.3cm}
\setlength{\parskip}{0.2cm}

\SingleSpacing

\begin{document}
	
	\selectlanguage{brazil}
	
	\frenchspacing 
	
	\begin{center}
		\LARGE PVC TRIGONOMÉTRICO
		
		\normalsize
		Matheus Mendes dos Santos Lipinski\& Gabriel Ernanes Alves Feitosa \footnote{Lipinski M.M.S., matheuslipinski.mm@gmail.com, Campus IFRO Porto Velho Calama\&
			Feitosa G.E.A., gabrielernanealves0@gmail.com, Campus IFRO Porto Velho Calama.} 
		Maria Gabriela Baim Gonzaga, GessicaStefany Alves Setúbal \&Eliciana da Conceição Silva dos Santos\footnote{Gonzaga M.G.B., gabriela.baim@hotmail.com, Campus IFRO Porto Velho Calama.
			Setúbal G.S.A., stefanysetubal97@gmail.com, Campus IFRO Porto Velho Calama\&
			Santos E.C.S, elicianasantos@hotmail.com ,Campus IFRO Porto Velho Calama.} 
		Adel Rayol de Oliveira Silva\footnote{Silva A.R.O. , adel.rayol@gmail.com, Campus IFRO Porto Velho Calama.} 
	\end{center}
	
	\noindent O desenvolvimento de uma ferramenta para auxílio de estudos trigonométricos para estudantes do ensino médio deu início a este projeto. Na notável dificuldade dos estudantes na área de matemática viu-se que os estudos em trigonometria demanda maior atenção para a efetiva compreensão dos estudantes, visto isto principalmente para alunos do curso técnico de edificações do Instituto Federal de Rondônia, pois estes possuem este tópico de estudo como essencial para desenvolvimento de cálculos referentes às estabilidades das construções, em plano inclinado. Teve como intuito a utilização de materiais de fácil acesso para que, futuramente, possa esta ferramenta ser levada e confeccionada em outras instituições de ensino com baixa demanda monetária.  O método para compreensão dos estudantes nesta vertente disciplinar trouxe como meio de compreensão, através de experiências empíricas, que o contato físico com tópicos como: equivalência de valores de ângulos quanto às funções de seno e coseno na área de trigonometria. Isto possibilitará melhor entendimento destas relações e de como os ângulos influenciam nos momentos de força, conhecimentos necessários para estudos estruturais das edificações. Até o momento da escrita deste resumo os únicos resultados obtidos foram: a compreensão de noções trigonométricas e conhecimento de como estas noções são capazes de influenciar sobre os resultados matemáticos, por parte dos desenvolvedores deste projeto, mas como a ferramenta ainda não foi levada para dentro de sala de aula ou apresentada a alunos do curso técnico em edificações, fora deste projeto, ainda não foi possível presenciar a influencia que estas ferramentas irão influenciar no conhecimento e compreensão dos demais estudantes.
	
	\vspace{\onelineskip}
	
	\noindent
	\textbf{Palavras-chave}: Trigonometria. Compreensão dos conceitos. Noções trigonométricas.
	
\end{document}
