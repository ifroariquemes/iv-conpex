\documentclass[article,12pt,onesidea,4paper,english,brazil]{abntex2}

\usepackage{lmodern, indentfirst, color, graphicx, microtype, lipsum}			
\usepackage[T1]{fontenc}		
\usepackage[utf8]{inputenc}		

\setlrmarginsandblock{2cm}{2cm}{*}
\setulmarginsandblock{2cm}{2cm}{*}
\checkandfixthelayout

\setlength{\parindent}{1.3cm}
\setlength{\parskip}{0.2cm}

\SingleSpacing

\begin{document}
	
	\selectlanguage{brazil}
	
	\frenchspacing 
	
	\begin{center}
		\LARGE DESENVOLVIMENTO DE PROJETO SOB A METODOLOGIA P.B.L’S - GRUPO ABAQUAR - QUELÔNIOS DO MÉDIO MADEIRA\footnote{Trabalho realizado dentro da área de Conhecimento CNPq/CAPES: Métodos e Técnicas de Ensino}
		
		\normalsize
		Ludmylla Sanchez\footnote{2Ludmylla Sanchez, sanchez.lsc@hotmail.com, Campus Porto Velho –Calama.} 
		Bruno Barros\footnote{3Bruno Barros, brunim\_s@hotmail.com, Campus Porto Velho –Calama.} 
		Diones dos Santos\footnote{4Diones dos Santos, dyonesgs@gmail.com, Campus Porto Velho – Calama.} 
		Daiana Ayala\footnote{Daiana Ayala, dayana\_ayalla@hotmail.com. Campus Porto Velho – Calama.} 
		Dayana Gonçalves\footnote{Dayana Gonçalves, dayana.angelica@hotmail.com, Campus Porto Velho – Calama.}
		Eliezer Silva Sergio Junior\footnote{Eliezer Silva, casa44pvh@gmail.com, Campus Porto Velho –Calama.}
		Sergio Junior\footnote{Sergio Junior, sc.santosjr@gmail.com, Campus Porto Velho – Calama.}
		Antônio dos Santos Junior\footnote{Orientador: Antônio dos Santos Junior, antonio.junior@ifro.edu.br , Campus Porto Velho – Calama.}
	\end{center}
	
	\noindent Diferente das abordagens tradicionais baseadas em conteúdos e disciplinas puramente expositivas, o método de ensino que o Instituto Federal de Educação, Ciência e Tecnologia Rondônia - IFRO desenvolve junto à turma de Pós-Graduação Latu Senso em Gestão Ambiental (2016) modelos de ensino estruturados na interdisciplinaridade, na contextualização do conteúdo e na autonomia dos alunos. As atividades desenvolvidas no programa foram inspiradas nas experiências de ensino e aprendizagem dos Projetos e Problemas (do inglês P.B.L.’s), desenvolvidas em  universidades do Canadá e Finlândia, que são referências para diversas instituições de ensino no mundo. Assim, o presente trabalho tem como objetivo relatar a aplicabilidade desta nova proposta metodológica que visa à formação de um grupo empresarial, o estabelecimento de vínculo e interação entre os alunos, provocando estes dentro da temática ambiental que cerne o curso, transcendam o espaço acadêmico e alcancem a sociedade, trazendo soluções efetivas para situações problemas locais.Nessa perspectiva de aprendizado a sala de aula foi equipada com mobília especialmente projetada para o trabalho em equipe e as aulas foram ministradas concomitantemente por professores diversas áreas, levando os alunos a debates e síntese de projetos.Nesse contexto, a equipe Abaquar foi guiada na disciplina Normas e Técnicas para Elaboração de Trabalho Científico na elaboração do Projeto Quelônios do Médio Madeira, o qual abrange uma problemática ambiental local que repercute em vários segmentos socioeconômicos regionais. Em declínio, as populações das espécies mais comuns de quelônios (Podocnemisexpansa e P. unifilis) sofrem não apenas pela construção dos complexos hidroelétricos na região que modificaram significativamente os ambientes aquáticos e terrestres, acarretando no confinamento desses animais e no desaparecimento de ambientes importantes para forrageio e desova. Mas, também devido ao consumo predatório e saqueio dos ovos por parte da população ribeirinha que tem como habito cultural a sua inserção na alimentação. Diante desse panorama faz-se necessário propostas e ações de conservação dessas espécies, que ainda serão executadas pela equipe, e que referem-se: a pesquisa, identificação, desenvolvimento de conhecimento técnico, formulação de estratégias conservacionistas e ações de extensão (palestras, capacitação e atuação dos agentes comunitários) no médio Madeira.
	
	\vspace{\onelineskip}
	
	\noindent
	\textbf{Palavras-chave}: Metodologia. Ensino. Quelônios.
	
\end{document}
