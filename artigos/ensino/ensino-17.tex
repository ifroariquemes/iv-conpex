\documentclass[article,12pt,onesidea,4paper,english,brazil]{abntex2}

\usepackage{lmodern, indentfirst, color, graphicx, microtype, lipsum}			
\usepackage[T1]{fontenc}		
\usepackage[utf8]{inputenc}		

\setlrmarginsandblock{2cm}{2cm}{*}
\setulmarginsandblock{2cm}{2cm}{*}
\checkandfixthelayout

\setlength{\parindent}{1.3cm}
\setlength{\parskip}{0.2cm}

\SingleSpacing

\begin{document}
	
	\selectlanguage{brazil}
	
	\frenchspacing 
	
	\begin{center}
		\LARGE REAPROVEITAMENTO DE SACOLAS PLÁSTICAS NA CONFECÇÃO DE OBJETOS DECORATIVOS/UTILITÁRIOS E APLICAÇÃO DE CÁLCULO DE ÁREA\footnote{Área de Conhecimento CNPq/CAPES, Ciências Exatas, com financiamento da (CAPES).}
		
		\normalsize
		Claudinei de Oliveira Pinho\footnote{Orientador, claudinei.pinho@ifro.edu.br, Campus Vilhena.} 
		Marlene UrupTossue\footnote{Bolsista (PIBID), marleneurup@hotmail.com, Campus Vilhena.} 
		Valdir José Alves\footnote{Colaborador, valdir\_icaro@hotmail.com, Campus Vilhena.} 
		Gleiciane Ferreira de Souza\footnote{Colaboradora, gleiciane.ninha@hotmail.com, Campus Vilhena.} 
	\end{center}
	
	\noindent O uso e descarte desenfreado das sacolas plásticas, e o déficit de aprendizagem na disciplina de matemática têm sido duas questões muito discutidas entre alunos do curso de Licenciatura Matemática do Instituto Federal de Rondônia durante a realização do estágio. Sendo assim, a presente pesquisa foi desenvolvida pelos alunos do curso de Licenciatura em Matemática do Instituto Federal de Rondônia aplicado a turma do 8º ano E do ensino fundamental na escola Estadual Álvares de Azevedo, localizada em Vilhena Rondônia. O objetivo desse estudo foi conscientizar e sensibilizar esses alunos acerca da preservação do meio ambiente através do reaproveitamento das sacolas plásticas descartadas no dia a dia e ainda contribuir no processo de ensino e aprendizagem de Matemática. Essa pesquisa teve como base metodológica o método qualitativo descritivo. Para analisar os dados foram realizadaspesquisas bibliográficas, palestras, oficinas, exposições, avaliação diagnóstica e registros fotográficos. Os objetos criados nas oficinas foram utilizados como instrumentos didáticos pelos professores de matemática no cálculo de área, na qual sua utilização foi imprescindível devido à constatação, através da avaliação diagnóstica, de que a maior parte dos alunos confundia o cálculo de área com o cálculo do perímetro. Através dos dados obtidos no projeto, foi possível verificar que os alunos são capazes de distinguir e reconhecer as figuras planas em qualquer ambiente. Constatou-se por meio desta metodologia que os alunos passaram a compreender os conceitos matemáticos antes não visualizados.
	
	\vspace{\onelineskip}
	
	\noindent
	\textbf{Palavras-chave}: Reaproveitamento. Aprendizagem. Matemática.
	
	\noindent
	\textbf{Agência financiadora}: CAPES
	
\end{document}
