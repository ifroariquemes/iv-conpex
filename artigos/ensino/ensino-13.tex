\documentclass[article,12pt,onesidea,4paper,english,brazil]{abntex2}

\usepackage{lmodern, indentfirst, color, graphicx, microtype, lipsum}			
\usepackage[T1]{fontenc}		
\usepackage[utf8]{inputenc}		

\setlrmarginsandblock{2cm}{2cm}{*}
\setulmarginsandblock{2cm}{2cm}{*}
\checkandfixthelayout

\setlength{\parindent}{1.3cm}
\setlength{\parskip}{0.2cm}

\SingleSpacing

\begin{document}
	
	\selectlanguage{brazil}
	
	\frenchspacing 
	
	\begin{center}
		\LARGE ENSINO DE FÍSICA INTEGRADO À ROBÓTICA EDUCACIONAL: CORRIDA DE ROBÔS\footnote{Área de Conhecimento CNPq/CAPES: 1.05.00.00-6 Física.}
		
		\normalsize
		Mirian Rodrigues Pedrosa\footnote{Bolsista/graduação, mirian5304@hotmail.com Campus Porto Velho Calama.} 
		Izabela Vieira Lima de Oliveira\footnote{Bolsista/graduação, vloliveiraizabela@gmail.com Campus Porto Velho Calama.} 
		Kelverton Willes\footnote{Colaborador/graduação, kelvertonwillis@gmail.com Campus Porto Velho Calama.}\\ 
		Mauro Guilherme Ferreira Bezerra\footnote{Orientador, mauro.guilherme@ifro.edu.br, Campus Porto Velho Calama} 
		Willians de Paula Pereira\footnote{Co-orientador, willians.pereira@ifro.edu.br, Campus Porto Velho Calama.}
	\end{center}
	
	\noindent O presente resumo tem como objetivo relatar o projeto que está sendo desenvolvido com alunos das escolas de Escolas Públicas do Ensino Médio na cidade de Porto Velho – RO.  As atividades estão sendo desenvolvidas através dos acadêmicos/bolsistas do PIBID (Programa Institucional de Bolsa de Iniciação à Docência) da Graduação em Licenciatura em Física do Instituto Federal de Rondônia – Campus Porto VelhoCalama. O trabalho desenvolvido objetiva demonstrar aos alunos a aplicação do ensino de física com robótica, utilizando o auxílio de mídias. A Físicaé um processo de descoberta do mundo natural e de suas propriedades. Talvez a parte mais difícil no ensino da física seja a tradução do fenômeno observado em símbolo. Uma coisa é ver um objeto se movendo em velocidade constante, outra é escrever uma equação que represente tal variação. Mas é justamente aqui que o desafio pode ser transformado em bônus. O ensino da física deve se conectar a visualização do fenômeno e sua expressão matemática. Como alternativas ao método convencional de ensino-aprendizagem, propomos aos alunos a construção de sua aprendizagem usando da Robótica Educacional, que por sua vez, se apresenta como uma forma inovadora, dinâmica e efetiva para o aprendizado de conceitos interdisciplinares como matemática, física, mecânica, pensamento lógico entre outros. Trabalhamos com o conteúdo da cinemática, que é parte da mecânica, sendo o primeiro contato dos alunos do primeiro ano. Foram selecionadas mídias (vídeos do Youtube) para auxiliar e orientar no processo de execução das atividades. Os alunos foram levados a discutir ideias, montagens, opiniões e conclusões, aprendendo a importante habilidade de trabalhar em equipe. O conteúdo foi o de M.R.U.na experiência em questão “Corrida de Robôs”, que consiste na uniformidade de espaços em intervalos de tempo iguais, o que implica uma velocidade constante (sem aceleração). Ao longo de uma linha reta, como robô que trafegando por uma pista retilínea. Os alunos construíram seus robôs para que pudessem executar a corrida em linha reta e deveriam calcular o desempenho dos mesmos. Desta forma podemos constatar que a física encontrou na robótica educacional uma ferramenta para facilitar o aprendizado de seus conceitos.
	
	\vspace{\onelineskip}
	
	\noindent
	\textbf{Palavras-chave}: Ensino de Física. Robótica Educacional. Mídias.
	
	\noindent
    \textbf{Fonte de financiamento}: IFRO – Porto Velho/Calama, CNPq bolsa de pesquisa PIBID.
	
\end{document}
