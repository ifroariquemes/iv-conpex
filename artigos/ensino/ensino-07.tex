\documentclass[article,12pt,onesidea,4paper,english,brazil]{abntex2}

\usepackage{lmodern, indentfirst, color, graphicx, microtype, lipsum}			
\usepackage[T1]{fontenc}		
\usepackage[utf8]{inputenc}		

\setlrmarginsandblock{2cm}{2cm}{*}
\setulmarginsandblock{2cm}{2cm}{*}
\checkandfixthelayout

\setlength{\parindent}{1.3cm}
\setlength{\parskip}{0.2cm}

\SingleSpacing

\begin{document}
	
	\selectlanguage{brazil}
	
	\frenchspacing 
	
	\begin{center}
		\LARGE QUALIDADE DE VIDA NO AMBIENTE DE TRABALHO NO IFRO\footnote{Trabalho realizado dentro da área de Conhecimento de Ensino CNPq/CAPES: Ciências Sociais Aplicadas com apoio do IFRO – Campus Porto Velho Zona Norte.}
		
		\normalsize
		Geiza Mendonça\footnote{Graduanda do curso de Gestão Pública, geizabotelhomendonca@hotmail.com, Campus Porto Velho Zona Norte.} 
		Ângela Dalmolini\footnote{Ângela Dalmolini, angeladalmolininunes@gmail.com, Campus Porto Velho Zona Norte.}
		Milene Coelho\footnote{Milene Coelho, milenecoelho@outlook.com,Campus Porto Velho Zona Norte.}
		Lady Day Pereira de Souza\footnote{Milene Coelho, milenecoelho@outlook.com,Campus Porto Velho Zona Norte.} 
	\end{center}
	
	\noindent  AAgendaAmbiental na Administração Pública – A3P éumaferramentainovadoraqueestimulaosgestorespúblicosaincorporaremprincípiosecritériosdegestão socioambientalemsuasatividadescotidianas.
	Estaorientaçãoenfatizaseiseixostemáticos,dentreeles,a“Qualidadedevidanoambientedotrabalho”,
	enfoquenestetrabalho.IniciadonadisciplinadeGestãoAmbientaleResponsabilidadeSocialdo5ºperíodo
	\\docursodeGestãoPública, o referido projetodeensinotevecomoobjetivoavaliaraqualidadedevidanoambiente do trabalhodosservidoresdo IFRO CampusPortoVelhoZona Norte. Ametodologiadeaplicação\\baseou-seemuma entrevista com o coordenador do setor de serviços gerais que nos acompanhou durante essa atividade dando todo suporte apoio durante o trabalho dentro do Campus e aplicação de questionário aos servidores através da ferramenta googledrive. O projeto partiu doprincípioqueoservidornodesempenhodesuasfunçõeslaboraiséumserholísticoequesuaatividadeprofis-sional,realizaçãopesso- aleprodutividadenotrabalhoestãoatreladas e são aspectosdebem-estarsocial,fí-sico,intelectual,psicológicos sustentáveis. Como resultado, identificamos alguns fatores de qualidade e elaboramos sugestões para a melhoria da qualidade de vida no ambiente de trabalho do servidor do IFRO – Campus Porto Velho Zona Norte, como: aplicação regular de questionário de avaliação aos servidores; melhoria da sala de convivência; criação de Comissão Interna de Prevenção de Acidentes – CIPA; previsão de Semana Interna de Prevenção de Acidente de Trabalho no calendário, e; implantação de símbolos internacionais de acessibilidade das edificações, mobiliários, espaços e equipamentos utilizáveis por pessoas deficientes. Comofuturosgestorespúblicostemos a consciênciadequeas ações que produzem qualidadedevidanotrabalho são ferramentas degestãoque visam proporcionar satisfaçãonotrabalho,alémdemelhorarascondiçõesda vida aoreduzirosníveisdeestresses dos servidores nas suas funções.
	
	\vspace{\onelineskip}
	
	\noindent
	\textbf{Palavras-chave}: Qualidade de vida no Trabalho. Agenda Ambiental na Administração Pública.IFRO.
	
\end{document}
