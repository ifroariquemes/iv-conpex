\documentclass[article,12pt,onesidea,4paper,english,brazil]{abntex2}

\usepackage{lmodern, indentfirst, color, graphicx, microtype, lipsum}			
\usepackage[T1]{fontenc}		
\usepackage[utf8]{inputenc}		

\setlrmarginsandblock{2cm}{2cm}{*}
\setulmarginsandblock{2cm}{2cm}{*}
\checkandfixthelayout

\setlength{\parindent}{1.3cm}
\setlength{\parskip}{0.2cm}

\SingleSpacing

\begin{document}
	
	\selectlanguage{brazil}
	
	\frenchspacing 
	
	\begin{center}
		\LARGE PRÁTICAS EXITOSAS NO ENSINO DE CIÊNCIAS DA NATUREZA E TERRA E SUAS TECNOLOGIAS NO CAMPUS VILHENA\footnote{Trabalho realizado dentro da (área de Conhecimento CNPq/CAPES: Ciências Exatas e da Terra com financiamento do IFRO.}
		
		\normalsize
		Marisa Rodrigues de Lima\footnote{Orientadora, marisa.rodrigues@ifro.edu.br, CampusVilhena.} 
		Leonardo Pereira da Silva\footnote{Colaborador, leonardo.silva@ifro.edu.br, CampusVilhena.} \\
		Tatiana Abreu Curado Rezende\footnote{Colaboradora, tatiana.rezende@ifro.edu.br, Campus Vilhena.} 
		Melquisedeque da Conceição Lima\footnote{Colaborador, melquisedeque.lima@ifro.edu.br, Campus Vilhena.} 
	\end{center}
	
	\noindent Este trabalho objetivou melhorar o processo de ensino e aprendizagem das disciplinas de ciências e suas tecnologias a fim de elevar o índice de permanência dos alunos. Aproximou nossos alunos da realidade de aplicação dos conteúdos aos quais eles têm acesso em suas aulas teóricas. O público alvo deste projeto são os alunos do IFRO Campus Vilhena (do 2º ao 4º anos), que atuaram como monitores, bolsistas, colaboradores e participantes realizando experimentos e ainda pretende atingir de maneira indireta os alunos das redes pública e particular de Vilhena que visitarem a mostra de modelos e experimentos contendo 29 trabalhos envolvendo os principais temas da física e biologia que foi organizada para apresentação no Congresso de Pesquisa, Ensino e Extensão do IFRO - CONPEX. Esta proposta justifica-se pela necessidade de responder a perguntas realizadas pelos alunos do Ensino Médio na área das ciências da natureza e da Terra. Eles questionam o porquê devem estudar determinados conteúdos e quais seriam suas aplicações na sua formação e vida prática.  Sendo assim, este projeto visou trabalhar a melhoria do processo de ensino e aprendizagem por meio da resolução de problemas, estimulando a curiosidade dos educandos e dando significado a sua aprendizagem, contribuindo assim, para a diminuição do índice de permanência. Nosso objetivo geral foi alcançado, pois melhorou significativamente o processo de ensino e aprendizagem da disciplina de física verificada por meio dos depoimentos dados pelos alunos quanto a um maior interesse na disciplina após a confecção dos trabalhos. É por esse viés, de ensino e aprendizagem significativa, que pensamos em trabalhar os conteúdos do ensino de ciências da natureza e da Terra, numa abordagem inovadora. Levando nossos alunos a descobertas e reflexões sobre os novos conhecimentos.
	
	\vspace{\onelineskip}
	
	\noindent
	\textbf{Palavras-chave}: Aprendizagem. Experimentos. Ensino Médio.
	
\end{document}
