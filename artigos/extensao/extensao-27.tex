\documentclass[article,12pt,onesidea,4paper,english,brazil]{abntex2}

\usepackage{lmodern, indentfirst, color, graphicx, microtype, lipsum}			
\usepackage[T1]{fontenc}		
\usepackage[utf8]{inputenc}		

\setlrmarginsandblock{2cm}{2cm}{*}
\setulmarginsandblock{2cm}{2cm}{*}
\checkandfixthelayout

\setlength{\parindent}{1.3cm}
\setlength{\parskip}{0.2cm}

\SingleSpacing

\begin{document}
	
	\selectlanguage{brazil}
	
	\frenchspacing 
	
	\begin{center}
		\LARGE A PRÁTICA DO BADMINTON PARA ESCOLARES\footnote{Trabalho realizado dentro da (área de Conhecimento CNPq/CAPES: Educação) com financiamento do DEPEX Campus Vilhena.}
		
		\normalsize
		Mario Mecenas Pagani\footnote{Professor EBTT – Educação Física - IFRO – mario.pagani@ifro.edu.br – Campus Vilhena.} 
	\end{center}
	
	\noindent O desporto é tratado como fenômeno sociocultural, e deve ser considerado como um bem da humanidade. O Badminton é um esporte que utiliza raquetes leves e é disputado com uma peteca (volante) entre duas ou quatro pessoas nas modalidades simples (masculino e feminino) duplas (masculino e feminino) e duplas mistas. Geralmente é disputado em campo coberto sendo dividido por uma rede com altura de 1,55m. A prática do Badminton se faz golpeando uma peteca (elemento móvel) com uma raquete. Os objetivos do estudo foram: situar a modalidade de Badminton no universo das atividades corporais e da Educação Física; vivenciar a modalidade de Badminton para maior compreensão e contribuição como cultura corporal; conhecer e aprender a jogar o Badminton, levando em conta a estrutura necessária, as regras, os equipamentos e as habilidades necessárias - mais precisamente o rebater. Fizeram parte do projeto, 62 alunos, sendo 39 do IFRO e 21 das escolas municipais e estaduais do município de Vilhena. As aulas foram expositivas e dialogadas, a parte prática dos fundamentos do Badminton foram realizados em uma quadra poliesportiva coberta, utilizando para esta atividades os seguintes materiais: raquetes, petecas, redes oficiais de Badminton e hastes metálicas para sustentação das redes, cordas e elásticos. Os dados foram coletados através da observação do autor (professor). Os resultados obtidos foram:  participação maciça dos alunos, onde os mesmos apresentaram um grande interesse em conhecer e praticar a modalidade; a totalidade dos participantes não conheciam a modalidade, onde muitos nunca ouviram falar; a evolução dos alunos foi significativa, onde a maioria dos alunos, 95\% conseguiu a rebater a peteca na primeira aula, uma característica deste esporte, onde o aluno em menos de 30 minutos já está rebatendo a peteca com um colega. Conclui-se que: todos os participantes conheceram o Badminton satisfatoriamente, e conseguiram realizar os movimentos básicos do esporte, bem como, jogá-lo de uma maneira adequada, utilizando os conceitos básicos do esporte e respeitando as regras do mesmo.
	
	\vspace{\onelineskip}
	
	\noindent
	\textbf{Palavras-chave}: Badminton. Iniciação. Escolares.
	
\end{document}
