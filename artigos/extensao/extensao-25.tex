\documentclass[article,12pt,onesidea,4paper,english,brazil]{abntex2}

\usepackage{lmodern, indentfirst, color, graphicx, microtype, lipsum}			
\usepackage[T1]{fontenc}		
\usepackage[utf8]{inputenc}		

\setlrmarginsandblock{2cm}{2cm}{*}
\setulmarginsandblock{2cm}{2cm}{*}
\checkandfixthelayout

\setlength{\parindent}{1.3cm}
\setlength{\parskip}{0.2cm}

\SingleSpacing

\begin{document}
	
	\selectlanguage{brazil}
	
	\frenchspacing 
	
	\begin{center}
		\LARGE FOTOGRAFIA E CEGUEIRA: \\UM EXPERIMENTO SOBRE AS FRONTEIRAS DA COMUNICAÇÃO VISUAL\footnote{Trabalho realizado dentro da (área de Conhecimento CNPq/CAPES: Linguística, Letras e Artes) com financiamento do Instituto Federal de Rondônia, Campus Porto Velho Zona Norte.}
		
		\normalsize
		Carina Michele Oliveira dos Santos Gomes\footnote{Bolsista: Carina Michele Oliveira dos Santos Gomes, carina-michele95@outlook.com, Campus Porto Velho Zona Norte.} 
		Luiza Nascimento Campos\footnote{Bolsista: Luiza Nascimento Campos, luizacampos233@gmail.com, Campus Porto Velho Zona Norte.} 
		Caroline Alves Dias\footnote{Bolsista: Caroline Alves Dias, dias.caroline@hotmail.com, Campus Porto Velho Zona Norte.} \\
		Aloir Pedruzzi Junior\footnote{Colaborador(a): Aloir Pedruzzi Junior, aloir.pedruzzi@ifro.edu.br, Campus Porto Velho Zona Norte.}
		Walteir Costa\footnote{Colaborador(a): Walteir Costa, walteir2008@gmail.com, SEDUC }
		Ana Cláudia Dias Ribeiro\footnote{Orientador(a): Ana Cláudia Dias Ribeiro, ana.ribeiro@ifro.edu.br, Campus Porto Velho Zona Norte.}
		Emi Silva de Oliveira\footnote{Co-orientador(a): Emi Silva de Oliveira, emi.oliveira@ifro.edu.br, Campus Porto Velho Zona Norte.} 
	\end{center}
	
	\noindent O presente projeto reflete sobre a conceituação da arte da fotografia para além das fronteiras visuais, tendo como público alvo a comunidade interna e externa, identificando jovens e adultos com graus variados de deficiência visual juntamente com os membros visuais desta comunidade. Assim, aprenderão as técnicas de fotografia, interagindo de forma humanista, valorizarão a comunicação nas relações humanas e entenderão melhor como é o universo dos deficientes visuais, percebendo que não há limites para eles, pois o sentido da visão não está na capacidade física de enxergar, mas sim no sentido amplo de ver. Com isso, o objetivo deste é fortalecer o processo de inclusão social, compartilhando experiências sensoriais e promovendo um estreitamento nas relações entre pessoas visuais e com deficiência visual, contribuindo para a melhoria da autoestima das pessoas com essa deficiência. Por meio do recurso cognitivo e constitutivo da fotografia como linguagem visual e processo de comunicação, incrementando a assimilação dos processos visuais como elementos interpretativos na construção de uma comunicação criativa e expressiva. Assim, através dos recursos da linguagem fotográfica com foco na deficiência visual, realizaremos a pesquisa teórica e prática acerca das possibilidades da atividade fotográfica na produção do conhecimento e fomentando a inclusão social através das possibilidades interpretativas da deficiência visual.  A metodologia aplicada será incluir pessoas visuais que ajudarão como intérpretes de imagens junto aos deficientes, auxiliando-os na descrição do ambiente ao qual irão fotografar e com as técnicas de fotografia, participando das dinâmicas realizadas durante as oficinas, interagindo com os deficientes, com a finalidade de se colocar no lugar do outro, no caso da pessoa com deficiência visual.  Até o momento, pudemos observar que não há fronteiras para quem possui deficiência visual, há um universo amplo a ser explorado, o que falta muitas vezes é a oportunidade.
	
	\vspace{\onelineskip}
	
	\noindent
	\textbf{Palavras-chave}: Fotografia. Deficiência visual. Comunicação visual.
	
\end{document}
