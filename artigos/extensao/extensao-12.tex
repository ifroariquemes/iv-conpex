\documentclass[article,12pt,onesidea,4paper,english,brazil]{abntex2}

\usepackage{lmodern, indentfirst, color, graphicx, microtype, lipsum}			
\usepackage[T1]{fontenc}		
\usepackage[utf8]{inputenc}		

\setlrmarginsandblock{2cm}{2cm}{*}
\setulmarginsandblock{2cm}{2cm}{*}
\checkandfixthelayout

\setlength{\parindent}{1.3cm}
\setlength{\parskip}{0.2cm}

\SingleSpacing

\begin{document}
	
	\selectlanguage{brazil}
	
	\frenchspacing 
	
	\begin{center}
		\LARGE RELATO DE EXPERIÊNCIA:\\ESTÁGIO NO INSTITUTO LABORATORIAL CRIMINIAL DA POLITEC DE RONDÔNIA \footnote{Trabalho realizado no âmbito de uma parceria entre Instituto Federal de Rondônia e Policia Técnico-Científica.}
		
		\normalsize
		Géssica Leal de Oliveira\footnote{Aluna Estagiária, glprpo@gmail.com, Campus Porto Velho Calama.} 
		Andressa Teixeira de Souza Guedes \,\,
		Jamile Macedo Mariano 
		
	\end{center}
	
	\noindent O presente relato descreve as experiências durante o Estágio Curricular Obrigatório do Curso Técnico em Química, através do Instituto Laboratorial Criminal (ILC) da Polícia Técnico-Científica do Estado de Rondônia (POLITEC-RO), realizada no período de abril a agosto de 2016. Os objetivos deste trabalho consistem em descrever os procedimentos executados no laboratório e elencar as contribuições na formação acadêmica da aluna/estagiária. O ILC realiza análises de drogas de abuso e objetos utilizados para preparo de drogas que são apreendidos na cidade de Porto Velho e nos demais municípios do estado de Rondônia. Os principais exames realizados são exames preliminares de constatação e exames definitivos. O exame de constatação consiste na triagem dos materiais apreendidos com suspeita de serem drogas, como uma garantia aos direitos da pessoa sendo válido até que se possa proceder ao exame definitivo. As drogas apreendidas com mais freqüência são cocaína e maconha. No exame preliminar de cocaína utilizamos o método Scott Modificado. Para o exame preliminar de maconha utilizamos o método\textbf{ Fast Blue B}, no qual, após a realização das etapas, a formação de um anel avermelhado indica positividade (presença de THC). No exame definitivo utilizamos um método chamado de Cromatografia em Camada Delgada (CCD). É retirada uma alíquota da amostra e colocada no \textbf{Ependorf®} com metanol para extração, a solução extraída é puncionada com tubos capilares em uma placa de sílica junto a um padrão do tipo de droga que está sendo analisada. Essa placa de sílica é colocada em cubas de vidro com reagentes que são absorvidos pelas placas, as placas são retiradas e após secarem são reveladas com uma substância que aponta os níveis de padrão. A parceria IFRO-POLITEC permitiu, além do aprimoramento individual dos alunos participantes do Estágio Obrigatório, a formação de força de trabalho que futuramente poderá atuar nos mais diversos seguimentos da Toxicologia forense no estado de Rondônia, no intuito de combater a comercialização de drogas ilícitas na região.
	
	\vspace{\onelineskip}
	
	\noindent
	\textbf{Palavras-chave}: Toxicologia forense. Análises de Drogas de Abuso. Laboratório de química.
	
\end{document}
