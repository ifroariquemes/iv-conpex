\documentclass[article,12pt,onesidea,4paper,english,brazil]{abntex2}

\usepackage{lmodern, indentfirst, color, graphicx, microtype, lipsum, textcomp}			
\usepackage[T1]{fontenc}		
\usepackage[utf8]{inputenc}		

\setlrmarginsandblock{2cm}{2cm}{*}
\setulmarginsandblock{2cm}{2cm}{*}
\checkandfixthelayout

\setlength{\parindent}{1.3cm}
\setlength{\parskip}{0.2cm}

\SingleSpacing

\begin{document}
	
	\selectlanguage{brazil}
	
	\frenchspacing 
	
	\begin{center}
		\LARGE RELATO DE EXPERIÊNCIA:\\ESTÁGIO NO LABORATÓRIO DE BIOGEOQUÍMICA WOLFGANG C. PFEIFFER DA UNIVERSIDADE FEDERAL DE RONDÔNIA \footnote{Trabalho realizado dentro da (área de Conhecimento CNPq/CAPES: Ciências Exatas e da Terra) executado no âmbito de uma parceria entre Instituto Federal de Rondônia e Universidade Federal de Rondônia.}
		
		\normalsize
		Vagner Nunes dos Santos\footnote{Aluno Estagiário, vagner.nunes14@gmail.com, Campus Porto Velho Calama.} 
		Josenaldo Santos Porto \footnote{Orientador de Estágio, josenaldo.porto@ifro.edu.br, Campus Porto Velho Calama.} 
		 
	\end{center}
	
	\noindent 
	Este relato descreve as experiências vividas pelo aluno Vagner Nunes dos Santos no Laboratório de Biogeoquímica Wolfgang C. Pfeiffer (LBWCP) da Universidade Federal de Rondônia. O estudo ora apresentado foi estruturado com base no método descritivo-observacional, realizado no período de abril a junho de 2016, no âmbito de uma parceria entre Instituto Federal de Rondônia (IFRO) e Universidade Federal de Rondônia (Unir). Os objetivos deste estudo consistem em descrever os procedimentos executados no laboratório e elencar a contribuições na formação acadêmica do aluno/estagiário. O LBWCP analisa amostras de peixes contaminados por Mercúrio total (Hg) e Metil-Mercúrio [CH\textsubscript{3}Hg]. Os principais equipamentos utilizados nos procedimentos experimentais são: purificador de água UV, e espectrofotômetro de absorção atômica. O processo de análise de Hg e CH\textsubscript{3}Hg em peixes são executados de acordo com a classificação dos peixes (herbívoros/carnívoros). As amostras de peixes carnívoros eram pesadas em quantidades menores (0,2g), devido ao fato desses animais se alimentarem de peixes contaminados e o Mercúrio ser um elemento acumulativo, já as amostras de peixes herbívoros exigiam uma quantidade maior de material-amostra a ser analisada (0,4g).  As etapas seguintes consistiam na adição de Ácido Sulfonitrico de proporção 1:1 na solução-amostra e deposição no bloco digestor a 70°C por 30 minutos. Ao atingir a temperatura ambiente a amostra recebe a adição de permanganato e volta ao bloco digestor por 20 minutos, posteriormente a amostra é deixada em repouso por 12 horas, a Hidroxilamina [NH\textsubscript{2}OH] é adicionada para promover a oxidação do permanganato, em seguida a amostra é posta em tubo falcon e analisada no espectrofotômetro de absorção atômica, onde a quantidade de Mercúrio é determinada. O trabalho desenvolvido no laboratório alerta a população e o poder público sobre a contaminação de peixes, plantas e águas por Mercúrio e Metil-Mercúrio, nesse sentido o estágio no LBWCP contribui de forma significativa no aprendizado do aluno, articulando os saber teóricos e práticos, configurando importante experiência na formação profissional do estudante do curso técnico.
	
	\vspace{\onelineskip}
	
	\noindent
	\textbf{Palavras-chave}: Relato de experiência. Análise de Mercúrio. Laboratório de biogeoquímica.
	
\end{document}
