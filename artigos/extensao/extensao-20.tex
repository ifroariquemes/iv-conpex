\documentclass[article,12pt,onesidea,4paper,english,brazil]{abntex2}

\usepackage{lmodern, indentfirst, color, graphicx, microtype, lipsum}			
\usepackage[T1]{fontenc}		
\usepackage[utf8]{inputenc}		

\setlrmarginsandblock{2cm}{2cm}{*}
\setulmarginsandblock{2cm}{2cm}{*}
\checkandfixthelayout

\setlength{\parindent}{1.3cm}
\setlength{\parskip}{0.2cm}

\SingleSpacing

\begin{document}
	
	\selectlanguage{brazil}
	
	\frenchspacing 
	
	\begin{center}
		\LARGE PROJETO DE XADREZ ESCOLAR “IFRO SOBRE O TABULEIRO\footnote{Trabalho realizado dentro da (área de Conhecimento CNPq/CAPES: Ciências Humanas) com financiamento do Edital 40/2015/DEPEX/Cacoal.}
		
		\normalsize
		José Lucas Bezerra de Oliveira\footnote{Bolsista (Extensão), joselucas\_estudos@outlook.com, IFRO/Campus Cacoal} 
		Wildeilson Alexandre Carneiro\footnote{Bolsista (Extensão), wildeilsoncarneirooliveira@gmail.com, IFRO/Campus Cacoal} 
		Juliano Alves de Deus\footnote{Orientador, juliano.alves@ifro.edu.br, IFRO/Campus Cacoal} 
	 
	\end{center}
	
	\noindent A prática de xadrez é histórica e culturalmente associada ao desenvolvimento da inteligência e da memória, ao manifesto do intelecto, ao respeito ao oponente e ao exercício da paciência e da concentração. O projeto “IFRO sobre o Tabuleiro” é uma atividade de difusão da prática do xadrez escolar, e tem por objetivos: incentivar a prática do xadrez, contribuindo para a formação cultural e intelectual da sociedade acadêmica; promover a boa vivência entre os alunos do Campus Cacoal e outros Campi da rede IFRO; promover o desenvolvimento do xadrez como modalidade esportiva e permitir que a comunidade do entorno escolar tenha contato com a prática do xadrez. O projeto disponibiliza relógios, jogos e tabuleiros de xadrez que são distribuídos nas mesas do Campus a fim de permitir a participação de estudantes, professores e servidores. É realizado, geralmente, no pátio da instituição, e praticado nos horários de almoço e horários destinados a projetos, permitindo assim a participação de públicos variados e contribuindo para a interação entre os monitores e os demais participantes interessados, tais como alunos, servidores e convidados; possibilita assim a existência de um ambiente de descontração, entretenimento e integração escolar. Além disso, o projeto promove a realização de torneios e minicursos no Campus, que contam com a participação de escolas parceiras, promovendo o aprendizado coletivo e a prática para competições. Desde o início da realização do projeto nota-se que o número de praticantes no âmbito escolar vem aumentando de maneira progressiva, favorecendo desta forma o desenvolvimento de um ambiente de crescimento e estímulo intelectual. Também é mantido pelos monitores do projeto um blog, onde são postados desafios, resultados de torneios e curiosidades enxadrísticas, fazendo com que os amantes do xadrez tenham contato regular com o esporte. Assim, este projeto tem alcançado com êxito a sua proposta de difusão do xadrez no meio escolar.
	
	\vspace{\onelineskip}
	
	\noindent
	\textbf{Palavras-chave}: Xadrez. Entretenimento. Cultura.
	
\end{document}
