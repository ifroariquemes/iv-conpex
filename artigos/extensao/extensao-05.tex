\documentclass[article,12pt,onesidea,4paper,english,brazil]{abntex2}

\usepackage{lmodern, indentfirst, color, graphicx, microtype, lipsum}			
\usepackage[T1]{fontenc}		
\usepackage[utf8]{inputenc}		

\setlrmarginsandblock{2cm}{2cm}{*}
\setulmarginsandblock{2cm}{2cm}{*}
\checkandfixthelayout

\setlength{\parindent}{1.3cm}
\setlength{\parskip}{0.2cm}

\SingleSpacing

\begin{document}
	
	\selectlanguage{brazil}
	
	\frenchspacing 
	
	\begin{center}
		\LARGE CONTADORES DE HISTÓRIAS
		
		\normalsize
		Karina Almeida André\footnote{Bolsista (IFRO), Karina-sama@live.com, Campus Ariquemes} 
		Andreia dos Santos Oliveira\footnote{Colaborador(a), andreia.oliveira@ifro.edu.br, Campus Porto Velho Calama} 
		
	\end{center}
	
	\noindent O projeto, em desenvolvimento desde março de 2016, pelo Instituto Federal de Rondônia-Campus Ariquemes, tem como objetivo proporcionar o contato de crianças com a Literatura Infantil por meio não só da contação de histórias mas também da leitura oral e, estimular o interesse dos pequenos leitores por esse gênero textual. Para que o objetivo seja alcançado, desenvolvemos o projeto Contadores de Histórias que tem contribuído com a formação de leitores no município. A atividade consiste em semanalmente, durante uma hora e trinta minutos proporcionar atividades de leitura e contação de histórias às crianças que frequentam uma praça pública do município localizado no interior do estado. O trabalho é desenvolvido por uma coordenadora (Professora de Língua Portuguesa do IFRO) e uma aluna (Curso integrado em Manutenção e Suporte em Informática) que semanalmente escolhem histórias infantis e desenvolvem técnicas para contá-las. Percebe-se que tanto as crianças quantos os pais têm acolhido a proposta, participado assiduamente da atividade, o que nos faz concluir que estamos alcançando os objetivos propostos de disseminar a Literatura Infantil. Os resultados obtidos por meio de diálogos com os pais demonstram que a comunidade recebeu a referida ação de forma positiva. As crianças, nosso público alvo, demonstram a aceitação do projeto de várias maneiras: participando assiduamente da atividade, semanalmente solicitando aos pais que as levem à praça onde o projeto está em execução e, muitas vezes, deixando de lado outras atividades que podem ser realizadas no local e se dedicando apenas à leitura e a contação de histórias.
	
	\vspace{\onelineskip}
	
	\noindent
	\textbf{Palavras-chave}: Contadores de Histórias. Praça Municipal. Formação de Leitores.
	
\end{document}
