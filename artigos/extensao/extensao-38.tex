\documentclass[article,12pt,onesidea,4paper,english,brazil]{abntex2}

\usepackage{lmodern, indentfirst, color, graphicx, microtype, lipsum,textcomp}			
\usepackage[T1]{fontenc}		
\usepackage[utf8]{inputenc}		

\setlrmarginsandblock{2cm}{2cm}{*}
\setulmarginsandblock{2cm}{2cm}{*}
\checkandfixthelayout

\setlength{\parindent}{1.3cm}
\setlength{\parskip}{0.2cm}

\SingleSpacing

\begin{document}
	
	\selectlanguage{brazil}
	
	\frenchspacing 
	
	\begin{center}
		\LARGE ÉTICA E CIDADANIA: A IMPORTÂNCIA DO DEBATE DESSES CONTEXTOS NO AMBIENTE ESCOLAR\footnote{Trabalho realizado dentro da área de Conhecimento CNPq/CAPES: Ciências Humanas com financiamento do DEPEX/IFRO/Campus Cacoal.}
		
		\normalsize
		Karine da Silva Verbeno\footnote{(Ensino Médio), karineverbeno98@gmail.com, IFRO/Campus Cacoal.} 
		Vitor Fernandes\footnote{Bolsista (Ensino Médio), vitorfernandes001@gmail.com, IFRO/Campus Cacoal.} 
		Sirley Leite Freitas\footnote{Orientadora, sirley.freitas@ifro.edu.br, IFRO/Campus Cacoal.} 
	\end{center}
	
	\noindent As relações sociais, para que ocorram de forma harmoniosa, necessitam da imposição de regras comportamentais. Essas, normalmente, são estabelecidas por costumes, valores morais e condutas éticas normalizadas, em certos casos, em leis. No fim do século XX e início do século XXI houve diversas transformações nas áreas socioeconômicas, políticas e culturais devido aos espantosos avanços da ciência, da tecnologia e da polêmica globalização. Nessa perspectiva, as instituições educacionais devem ensinar a crianças e adolescentes, o que é, e como ser um cidadão. Desse modo, os conhecimentos relacionados à ética, cidadania, direitos e deveres são essenciais para a formação e consolidação de uma sociedade democrática de direito. Visando difundir a cultura da ética, da cidadania e conscientização sobre direitos fundamentais básicos no IFRO/Campus Cacoal e nas EEEFM Bernardo Guimarães e EEEFM Aurélio Buarque de Holanda Ferreira, instituições parceiras, o projeto possibilitou a efetivação de debates, criação de grupos de reflexão, capacitação de alunos e realização de palestras sobre ética, cidadania e direitos fundamentais básicos, o que contribuiu com a formação ética, política e cidadã dos adolescentes beneficiados pelo projeto. O público alvo constituiu-se dos alunos do curso técnico em Agroecologia e em Agropecuária integrado ao Ensino Médio do IFRO/Campus Cacoal, alunos do 9° ano do Ensino Fundamental e do projeto Salto das EEEFM Bernardo Guimarães e EEEFM Aurélio Buarque de Holanda Ferreira. Com desenvolvimento do projeto percebeu-se que a discussão de temáticas voltadas para ética, cidadania e direitos fundamentais é essencial para a formação dos jovens, para que esses, de alguma forma, possam aplicar no cotidiano o que lhes foi transmitido. Cabe salientar que todo estudo resulta em um aprendizado e uma experiência e o desenvolvimento desse projeto proporcionou aos alunos direta e indiretamente envolvidos um conhecimento mais profundo no que diz respeito a realidade do cenário da cultura da ética, da cidadania e dos direitos fundamentais básicos no espaço escolar e na realidade além dos muros da escola.
	
	\vspace{\onelineskip}
	
	\noindent
	\textbf{Palavras-chave}: Ética. Cidadania. Espaço escolar.
	
\end{document}
