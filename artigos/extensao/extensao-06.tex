\documentclass[article,12pt,onesidea,4paper,english,brazil]{abntex2}

\usepackage{lmodern, indentfirst, color, graphicx, microtype, lipsum}			
\usepackage[T1]{fontenc}		
\usepackage[utf8]{inputenc}		

\setlrmarginsandblock{2cm}{2cm}{*}
\setulmarginsandblock{2cm}{2cm}{*}
\checkandfixthelayout

\setlength{\parindent}{1.3cm}
\setlength{\parskip}{0.2cm}

\SingleSpacing

\begin{document}
	
	\selectlanguage{brazil}
	
	\frenchspacing 
	
	\begin{center}
		\LARGE CULTURA, MEMÓRIA E TRADIÇÃO ESTÉTICA: A CONSTRUÇÃO DA CIDADE CENOGRÁFICA DO ARRAIAL DO IFRO CAMPUS JI-PARANÁ \footnote{Trabalho realizado dentro da área “Ciências Humanas” com financiamento da Pró-Reitoria de Extensão (PROEX) do Instituto Federal de Rondônia.}
		
		\normalsize
		Bruna S. Mandu\footnote{Bolsista do projeto extensão “CULTURA, MEMÓRIA E TRADIÇÃO ESTÉTICA: A CONSTRUÇÃO DA CIDADE CENOGRÁFICA DO ARRAIAL DO IFRO CAMPUS JI-PARANÁ”, bruna.manduu@gmail.com, IFRO – Campus Ji-Paraná,} 
		Waldelaine R. Hoffmann\footnote{Colaboradora, waldelaine-hoffmann@hotmail.com, IFRO – Campus Ji-Paraná} 
		Jucielly E. de Almeida\footnote{Colaboradora, juciellyalmeida72@gmail.com, IFRO – Campus Ji-Paraná} 
		Lourival Inácio Filho\footnote{Orientador, lourival.filho@ifro.edu.br, IFRO – Campus Ji-Paraná.} 
	\end{center}
	
	\noindent A Festa Junina proporciona excelente oportunidade de desenvolvimento de diversas atividades interdisciplinares e possibilita ampliar os recursos didáticos para formação dos alunos, pois constitui-se em uma temática rica, onde podem ser explorados diversos tipos de linguagens estéticas e culturais que perpassam o resgate de brincadeiras, culinária típica e, principalmente memória de nossas tradições. Há alguns anos, como em outros Institutos Federais, o IFRO Campus Ji-Paraná vem realizando anualmente um dia dedicado às festividades juninas, onde ocorre a realização das famosas festas brasileiras dedicadas a Santo Antônio, São João e São Pedro e que atraem a todos que valorizam este tipo de cultura, da comunidade interna do Instituto à sociedade local. Partindo dessa ideia, o projeto buscou valorizar tal cultura por meio do resgate histórico-cultural desta tradição, despertando o interesse em conhecer as histórias das tradicionais festas juninas bem como suas origens por meio da idealização e execução da cenografia destinada ao arraial inerente a festa junina do IFRO Campus Ji-Paraná. O processo de execução dividiu-se em duas etapas: a formação de grupo de estudos para o levantamento de aspectos culturais e históricos da festa junina e também a execução; sendo idealizado e desenhado um croqui da cidade cenográfica seguido da construção da cidade cenográfica de acordo com os estudos realizados e o croqui desenhado. A construção da cidade cenográfica compreendeu-se em um período de aproximadamente quatro meses, tempo este que proporcionou aos alunos envolvidos a pesquisa e o conhecimento aprofundado acerca de assuntos antes estudados em sala de aula, tais como a história e formação da cultura e população brasileira.
	
	\vspace{\onelineskip}
	
	\noindent
	\textbf{Palavras-chave}: Festa Junina. Resgate Histórico-Cultural. Cidade Cenográfica.
	
\end{document}
