\documentclass[article,12pt,onesidea,4paper,english,brazil]{abntex2}

\usepackage{lmodern, indentfirst, color, graphicx, microtype, lipsum}			
\usepackage[T1]{fontenc}		
\usepackage[utf8]{inputenc}		

\setlrmarginsandblock{2cm}{2cm}{*}
\setulmarginsandblock{2cm}{2cm}{*}
\checkandfixthelayout

\setlength{\parindent}{1.3cm}
\setlength{\parskip}{0.2cm}

\SingleSpacing

\begin{document}
	
	\selectlanguage{brazil}
	
	\frenchspacing 
	
	\begin{center}
		\LARGE EXPERIÊNCIA DE ENSINO E APRENDIZAGEM – PIPEEX 2015\footnote{Trabalho realizado dentro da área de Finanças/Administração com financiamento do (a) Instituto Federal de Rondônia.}
		
		\normalsize
		Riane Souza Cabral\footnote{Técnica em Finanças pelo Instituto Federal de Rondônia Campus Porto Velho Zona Norte - Polo de Ouro Preto do Oeste, participante do PIPEEX na Universidade Nacional de Colômbia – UNAL/Sede Palmira e-mail: rianesouzacabral@gmail.com.} 
		Samuel dos Santos Junio\footnote{Orientador - Professor EBTT do Instituto Federal de Rondônia Campus Porto Velho Zona Norte área Administração. E-mail: samuel.santos@ifro.edu.br} 
	\end{center}
	
	\noindent O Programa de Internacionalização da Pesquisa e Extensão – PIPEEX, do IFRO no ano de 2015, com intuito de fazer a inclusão na modalidade estudantil como forma de melhoramento de seus alunos, tanto no âmbito da ciência quanto da tecnologia, com participação de estágio e pesquisa em instituição internacional, enviou uma aluna do curso de Técnico em Finanças da modalidade EAD, para a realização do estágio em Finanças na Universidade Nacional de Colômbia – UNAL/Sede Palmira, na Colômbia. Ao início do período do estágio, sobre o incentivo do professor e orientador, foi indicado a ela cursar a matéria de Matemática Financeira, pois a contribuição dessa matéria para o projeto, que estava sendo desenvolvido, seria de grande contribuição para chegar ao resultado final do trabalho de estágio. Para a coleta e análise dos dados foi a partir de leituras de temas sobre finanças, encontrados em teses, trabalhos de conclusão e artigos, e posteriormente para publicações de como é feita uma plantação de amoras, pois o trabalho envolvia a produção de amoras da região Valle. Com busca em saber o quanto custava os materiais usados, aprender o momento em que o produtor decide ter a sua própria plantação até o momento de vender e quanto o investimento custaria para cada produtor. Tempo esse passado em sala, com o professor responsável, em torno das leituras de livros e artigos também encontrados na internet. Como atividade desenvolvida fora da universidade, foi feita uma visita a alguns moricultores de Costa Rica – Valle, onde coletamos informações sobre quantidades de insumos, de quantas plantas cada um tinha em sua propriedade, do que era necessário para cada um manter a sua plantação, entre outras coisas. É foi obtido um aprendizado em analisar e avaliar melhor um projeto, de como colocar em pauta toda a sua estrutura e recursos que necessita, tendo em vista todo mercado empresarial exigente que temos nos dias atuais. Concluindo então, que necessitamos buscar mais, aprender mais, conviver e conhecer novas formas de ensino e que uma cultura diferente da nossa mostra que para mudar devemos apenas tentar e querer para que haja mudanças.
	
	\vspace{\onelineskip}
	
	\noindent
	\textbf{Palavras-chave}: IFRO. PIPEEX - 2015. Extensão.
	
\end{document}
