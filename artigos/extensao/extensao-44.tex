\documentclass[article,12pt,onesidea,4paper,english,brazil]{abntex2}

\usepackage{lmodern, indentfirst, color, graphicx, microtype, lipsum}			
\usepackage[T1]{fontenc}		
\usepackage[utf8]{inputenc}		

\setlrmarginsandblock{2cm}{2cm}{*}
\setulmarginsandblock{2cm}{2cm}{*}
\checkandfixthelayout

\setlength{\parindent}{1.3cm}
\setlength{\parskip}{0.2cm}

\SingleSpacing

\begin{document}
	
	\selectlanguage{brazil}
	
	\frenchspacing 
	
	\begin{center}
		\LARGE O ESPORTE EDUCACIONAL NO IFRO CAMPUS ZONA NORTE\footnote{Trabalho realizado dentro da área de Conhecimento CNPq/CAPES: Educação, com financiamento do Departamento de Extensão do Campus Porto Velho Zona Norte}
		
		\normalsize
		Matheus Mozart S. N. Borges\footnote{Bolsista de extensão, email mmsnborges@gmail.com, Campus Porto Velho Zona Norte} 
		Lidiane Cristina J. G. Jardim\footnote{Colaborador(a), email lidiane.jardim@ifro.edu.br, Campus Porto Velho Zona Norte} 
		Thiago P. Lima\footnote{Orientador(a), email thiago.lima@ifro.edu.br, Campus Porto Velho Zona Norte} 
		Ilma Paula C. Silva\footnote{Co-orientador(a), email ilma.silva@ifro.edu.br, Campus Porto Velho Zona Norte} 
	\end{center}
	
	\noindent Os Institutos Federais de Educação foram criados com a finalidade de consolidar a estratégica modalidade de ensino profissional e tecnológico como instrumento substantivo na construção e resgate da cidadania e transformação social. Este texto discute como a construção de espaços para a manifestação do esporte educacional contribui para a formação cidadã e para a descaracterização do processo linear de formação, agregando reflexões e valores que contribuirão para o êxito das diretrizes e práticas pedagógicas do IFRO Campus Zona Norte. As ações propostas convergem para os objetivos institucionais estabelecidos no Plano de Desenvolvimento Institucional, bem como ao que propõe a Política de Assistência Estudantil, ampliando o universo sociocultural, artístico e esportivo da comunidade acadêmica. A proposta dos jogos de integração teve como objetivo a integração de práticas educativas e de formação, ao proporcionar momentos de agregação de sujeitos, reflexão de valores e atitudes, confraternização e mediação de conflitos, os quais são inerentes nas relações sociais. Para concretização dessas ações buscou-se construir espaços de manifestações do esporte educacional no sentido da valorização dos elementos de formação da cidadania existentes na prática esportiva, bem como de confraternização, respeito e cooperação entre os educandos e servidores, considerando os sujeitos que constituem a comunidade institucional. Os resultados alcançados foram: a aquisição de equipamentos para criação do espaço de convivência incentivando o convívio em grupo, o trabalho em equipe, respeito às diferenças e a participação de alunos, servidores e colaboradores. Os jogos de integração têm propiciado possibilidades de melhorias nos processos cognitivos, atenção, concentração, liderança e planejamento, além de tornar a Instituição mais atrativa, transformando-a em um local onde o educando sinta-se integrado. O ambiente escolar, por fazer parte da vida do aluno, deve também propiciar "momentos de diversão, de competição e de cooperação". Nessa perspectiva as atividades esportivas vêm propiciando à comunidade acadêmica bons momentos de convivência e de aprendizagem, fortalecendo e criando novas relações, primando e estimulando o respeito entre os indivíduos dentro e fora da instituição.
	
	\vspace{\onelineskip}
	
	\noindent
	\textbf{Palavras-chave}: Espaços esportivos. Integração. Cidadania.
	
\end{document}
