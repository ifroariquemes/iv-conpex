\documentclass[article,12pt,onesidea,4paper,english,brazil]{abntex2}

\usepackage{lmodern, indentfirst, color, graphicx, microtype, lipsum}			
\usepackage[T1]{fontenc}		
\usepackage[utf8]{inputenc}		

\setlrmarginsandblock{2cm}{2cm}{*}
\setulmarginsandblock{2cm}{2cm}{*}
\checkandfixthelayout

\setlength{\parindent}{1.3cm}
\setlength{\parskip}{0.2cm}

\SingleSpacing

\begin{document}
	
	\selectlanguage{brazil}
	
	\frenchspacing 
	
	\begin{center}
		\LARGE SENSIBILIZAÇÃO SÓCIOAMBIENTAL DO ANTIGO PONTO DE DISPOSIÇÃO DE LIXO DO MUNICÍPIO DE JI-PARANÁ/RONDÔNIA\footnote{Trabalho realizado dentro da (área de Conhecimento CNPQ/CAPES: Educação Ambiental) com financiamento do Instituto Federal de Rondônia.}
		
		\normalsize
		Isabella Navarro da Silva\footnote{Colaboradora, isabellanavarro7k@gmail.com Campus Ji-Paraná} 
		Erica Patricia Navarro\footnote{Orientadora , erica.navarro@ifro.edu.br, Campus Ji-Paraná} 
		Andreza Pereira Mendonça\footnote{Co-orientadora, andreza.mendonca@ifro.ediu.br, Campus, Campus Ji-Paraná} 
		Alice Sperandio Porto\footnote{Co-orientadora, alice.porto@ifro.edu.br Campus, Campus Ji-Paraná} 
	\end{center}
	
	\noindent No Brasil, estima-se que mais de 90\% do lixo é jogado a céu aberto, gerando uma ameaça constante de doenças e contaminação ambiental. O município de Ji-Paraná, Rondônia insere-se nessa realidade, pois até os dias atuais não possui aterro sanitário. Além disso, as áreas dos lixões desativados não são devidamente controladas e/ou recuperadas, a exemplo do antigo ponto de disposição do lixo ocupado de maneira irregular pela população local. Sabe-se que nessas áreas há riscos a saúde da população e ao meio ambiente. Diante deste cenário, o projeto teve como objetivo sensibilizar a população local por meio de 500 alunos das escolas públicas: Celso Rocco e Silvio Micheluzzi, localizadas nas imediações do antigo lixão da T-28, no segundo distrito de Ji-Paraná, Rondônia.  Os trabalhos de sensibilização ambiental junto a população e aos alunos das escolas municipais e estaduais foram realizados por meio de oficinas de jogos didáticos ambientais e distribuição de cartilhas relacionadas a temática: Resíduos Sólidos. A distribuição dos materiais confeccionados deu-se após uma apresentação dialogada com os alunos do ensino fundamental sobre a importância da preservação ambiental, ressaltando os principais impactos ambientais negativos que ocorrem no município, com o descarte incorreto dos resíduos e orientação para o descarte consciente de resíduos sólidos domésticos. Foram validados 30 jogos sobre a temática ambiental trabalhada no projeto e atendidos cerca de 500 alunos nas oficinas de jogos e conscientização ambiental nas escolas. Os jogos trataram dos mais diversos tipos de resíduos e poluição. De forma lúdica buscou-se levar os alunos a uma reflexão sobre as ações do cotidiano e seus impactos no meio ambiente.
	
	\vspace{\onelineskip}
	
	\noindent
	\textbf{Palavras-chave}: Educação Ambiental. Jogos didáticos. Ensino.
	
\end{document}
