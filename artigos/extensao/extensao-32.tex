\documentclass[article,12pt,onesidea,4paper,english,brazil]{abntex2}

\usepackage{lmodern, indentfirst, color, graphicx, microtype, lipsum}			
\usepackage[T1]{fontenc}		
\usepackage[utf8]{inputenc}		

\setlrmarginsandblock{2cm}{2cm}{*}
\setulmarginsandblock{2cm}{2cm}{*}
\checkandfixthelayout

\setlength{\parindent}{1.3cm}
\setlength{\parskip}{0.2cm}

\SingleSpacing

\begin{document}
	
	\selectlanguage{brazil}
	
	\frenchspacing 
	
	\begin{center}
		\LARGE PRODUTOS DE PESCADO DESENVOLVIDO POR MULHERES PESCADORAS E AQUICULTORAS DE ARIQUEMES E PORTO VELHO­RO\footnote{Trabalho realizado dentro da (área de Conhecimento CNPq/CAPES: Ciências de alimentos) com financiamento do (a) (PROEX).}
		
		\normalsize
		Luziana Teixeira Cruz Galvão\footnote{(modalidade), email,Campus Ariquemes} 
		Carlos Henrique dos Santos\footnote{Bolsista (modalidade), email,Campus Ariquemes} 
		Raica Esteves Xavier Meante\footnote{Orientador(a), raica.xavier@ifro.edu.br,Campus Ariquemes}  
	\end{center}
	
	\noindent O projeto tem como linhas de atuação a inclusão social, geração de renda e oportunidades de trabalho com o objetivo de desenvolver e promover produtos a base de pescado a fim de incentivar práticas empreendedoras de modo a incrementar a renda familiar de mulheres pescadoras e aquicultoras dos municípios de Porto Velho e Ariquemes. Observou­se que as mulheres vêm ocupando o lugar de chefe­provedora das famílias, consequentemente, muitas vivem hoje, da atividade de pesca artesanal em todo o Brasil. Trabalhando, principalmente, na captura de mariscos, no beneficiamento de produtos e na confecção e reparo de apetrechos de pesca, as mulheres, vem aos poucos, se impondo num setor que guarda uma cultura de preconceitos em relação a elas. A elaboração de produtos de pescado aproveitando espécies de baixo valor comercial de água doce se apresenta como uma proposta inovadora, pois não existem produtos processados elaborados com espécies nativas no mercado regional. A utilização de matéria prima de baixo valor significa maiores ganhos para o setor e aproveitamento racional dos recursos pesqueiros. Há uma crescente demanda por produtos de pescado devido ao conhecimento de suas qualidades nutricionais, elevada digestibilidade, melhor balanço de aminoácidos essenciais, presença de vitaminas, minerais e ômega­3. A tendência de consumo atualmente é por produtos alimentícios prontos, semiprontos, de conveniência, principalmente de alto grau de aceitabilidade nutricional, sanitário, sensorial e aqueles que apresentem a menor dificuldade de práticas culinárias domésticas. Neste projeto estão sendo elaborados produtos como a linguiça de peixe, kibe de peixe, empanados, hambúrguer de peixe e o presunto de peixe, todos elaborados com espécies de baixo valor comercial ou com resíduos da filetagem de espécies nobres como o tambaqui \textbf{(Colossoma macropomum)}. Os produtos elaborados terão seu valor nutricional identificado e embalagem própria para comercialização.
	
	\vspace{\onelineskip}
	
	\noindent
	\textbf{Palavras-chave}:Processamento. Subprodutos. Pescado.
	
\end{document}
