\documentclass[article,12pt,onesidea,4paper,english,brazil]{abntex2}

\usepackage{lmodern, indentfirst, color, graphicx, microtype, lipsum}			
\usepackage[T1]{fontenc}		
\usepackage[utf8]{inputenc}		

\setlrmarginsandblock{2cm}{2cm}{*}
\setulmarginsandblock{2cm}{2cm}{*}
\checkandfixthelayout

\setlength{\parindent}{1.3cm}
\setlength{\parskip}{0.2cm}

\SingleSpacing

\begin{document}
	
	\selectlanguage{brazil}
	
	\frenchspacing 
	
	\begin{center}
		\LARGE DESENVOLVIMENTO DE SOFTWARE PARA PESQUISAS CIENTÍFICAS UTILIZANDO O FRAMEWORK METEOR\footnote{Trabalho realizado dentro da (área de Conhecimento CNPq/CAPES: Ciência da Computação) com financiamento do IFRO.}
		
		\normalsize
		Jéssica Giori Silva\footnote{Colaborador(a), giorijessica@gmail.com, Campus Ji-Paraná} 
		Leonardo Henrique de Braz\footnote{Bolsista (PIBIC), lhleonardo05@gmail.com, Campus Ji-Paraná} 
		Jackson Henrique da Silva Bezerra\footnote{Orientador(a), jackson.henrique@ifro.edu.br, Campus Ji-Paraná} 
		Adriana Aparecida Rigolon Guimarães\footnote{Orientador(a), jackson.henrique@ifro.edu.br, Campus Ji-Paraná } 
	\end{center}
	
	\noindent O presente trabalho tem por objetivo apresentar os resultados obtidos através do projeto de extensão intitulado “DESENVOLVIMENTO DE SOFTWARE PARA PESQUISAS CIENTÍFICAS UTILIZANDO O FRAMEWORK METEOR”. Através do projeto em questão foi construído um software para web com o objetivo de auxiliar pesquisadores da Universidade Federal de Rondônia que utilizam o método de pesquisa Delphi. Este é um método de prospecção que busca encontrar análises sobre o futuro de um determinado objeto de pesquisa. Para a construção do software os desenvolvedores utilizaram o framework Meteor, visto que este oferece facilidade para a sua utilização gerando um bom resultado para os usuários, já que o Meteor auxilia na produção de uma aplicação leve. O projeto de extensão foi desenvolvido no Instituto Federal de Rondônia, Campus Ji-Paraná e aplicado na UNIR na forma de transferência de tecnologia e minicursos para os alunos do curso de Estatística. Além do framework Meteor foram utilizados o Sistema de Gerenciamento de Banco de Dados (SGDB) MongoDB, sendo diferente dos bancos de dados relacionais, pois são orientados a documentos, além da ferramenta de gerenciamento de projeto Redmine, o repositório de código GitHub, utilizando a ferramenta de controle de versão Git e ferramenta de gerenciamento de atividades Trello. Os projetos de extensão são ferramentas importantes para a boa formação de bons técnicos e a oportunidade de ter acesso a este programa ainda no Ensino Médio é de uma grandeza inigualável. Com este projeto os estudantes puderam colocar em prática conhecimentos obtidos ao longo do curso, fazendo com que suas formações fossem complementadas. Com essa experiência os alunos puderam conhecer como seria a área de desenvolvedor de software na prática e enfrentar todas as dificuldades encontradas no processo de desenvolvimento de software. Somado a isso há ainda a contribuição para a área da pesquisa com a oferta de um software para acelerar pesquisa e auxiliar pesquisadores que utilizam o método Delphi. Este projeto de extensão foi realizado em conjunto com um projeto de pesquisa com objetivo de pesquisar linguagens de programação aplicada na construção de um software de pesquisa estatística com o método Delphi.
	
	\vspace{\onelineskip}
	
	\noindent
	\textbf{Palavras-chave}: Framework Meteor. Método Delphi. Software Web.
	
\end{document}
