\documentclass[article,12pt,onesidea,4paper,english,brazil]{abntex2}

\usepackage{lmodern, indentfirst, color, graphicx, microtype, lipsum}			
\usepackage[T1]{fontenc}		
\usepackage[utf8]{inputenc}		

\setlrmarginsandblock{2cm}{2cm}{*}
\setulmarginsandblock{2cm}{2cm}{*}
\checkandfixthelayout

\setlength{\parindent}{1.3cm}
\setlength{\parskip}{0.2cm}

\SingleSpacing

\begin{document}
	
	\selectlanguage{brazil}
	
	\frenchspacing 
	
	\begin{center}
		\LARGE ANÁLISE E TECNOLOGIA DE SEMENTES E PRODUÇÃO DE MUDAS\footnote{Área da Capes: Engenharia florestal. Fonte de Financiamento:IFRO}
		
		\normalsize
		Polyana Barros Nascimento\footnote{Bolsista (PIBIC - EM), polyanabarrosnc@gmail.com, Campus Ji-Paraná} 
		Sendy Mayra de Souza\footnote{Bolsista (PIBIC - EM), sendy.mayra@gmail.com, Campus Ji-Paraná} \\
		Maria Elessandra Rodrigues Araujo\footnote{Orientadora, maria.elessandra@ifro.edu.br, Campus Ji-Paraná} 
		Andreza Pereira Mendonça\footnote{Co-orientadora, andreza.mendonca@ifro.edu.br, Campus Ji-Paraná} 
	\end{center}
	
	\noindent Em Rondônia, a ocupação desordenada aliada a práticas inadequadas de manejo dos solos e associada a uma política equivocada de utilização da terra estimulou o desmatamento. Essa situação gerou no estado uma posição de destaque no cenário nacional com alto índice de desmatamento entre os estados da região Amazônica. É eminente a necessidade de mudanças qualitativas, baseadas em formas mais adequadas de uso e manejo dos recursos naturais. Desse modo, a formação de mão de obra qualificada em tecnologia de sementes florestais é o primeiro passo para garantir uso de propágulos com alto padrão genético na formação de mudas vegetais.  Este trabalho teve por objetivo colaborar na formação técnica de profissionais que atuarão na área de tecnologia e processamento de sementes florestais e ainda na formação de mudas florestais. Inicialmente foram elaborados material didático sobre análise de sementes e produção de mudas e utilizados como material de apoio para os cursos ministrados. Os cursos foram realizados no Laboratório de Sementes e Produtos Não Madeireiros do Instituto Federal de Rondônia, Campus Ji-Paraná para 50 participantes com aulas teóricas e práticas sobre análise e tecnologia de sementes e produção de mudas de essências florestais, tendo como público alvo discentes do curso Técnico Em Florestas, Ciências Biológicas, Agronomia e Engenharia Florestal. Durante os cursos foram utilizadas sementes com potencial econômico, tecnológico e nutricional, servindo como alternativa viável para superar as dificuldades socioeconômicas da população local por meio da diversificação da produção e abundância de espécies florestais na região amazônica. Os discentes foram capacitados para atuarem na área de tecnologia e processamento dessas sementes em especial na aplicação da metodologia do teste de tetrazólio para determinação de viabilidade, bem como em práticas de transplante, semeadura direta, formação de substratos, biometria e sanidade de mudas de espécies Amazônicas em viveiros.
	
	\vspace{\onelineskip}
	
	\noindent
	\textbf{Palavras-chave}: Essências florestais. Capacitação. Amazônia
	
\end{document}
