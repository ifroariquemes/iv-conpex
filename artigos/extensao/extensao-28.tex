\documentclass[article,12pt,onesidea,4paper,english,brazil]{abntex2}

\usepackage{lmodern, indentfirst, color, graphicx, microtype, lipsum}			
\usepackage[T1]{fontenc}		
\usepackage[utf8]{inputenc}		

\setlrmarginsandblock{2cm}{2cm}{*}
\setulmarginsandblock{2cm}{2cm}{*}
\checkandfixthelayout

\setlength{\parindent}{1.3cm}
\setlength{\parskip}{0.2cm}

\SingleSpacing

\begin{document}
	
	\selectlanguage{brazil}
	
	\frenchspacing 
	
	\begin{center}
		\LARGE INICIAÇÃO A NATAÇÃO\footnote{Trabalho realizado dentro da área de Conhecimento CNPq/CAPES: Educação com financiamento do DEPEX Campus Vilhena.}
		
		\normalsize
		Mario Mecenas Pagani\footnote{Professor EBTT – Educação Física - IFRO – mario.pagani@ifro.edu.br – Campus Vilhena.} 
	\end{center}
	
	\noindent Elemento cultural da humanidade, a natação foi sistematizada em quatro tipos de nados (crawl, peito, costa e borboleta) que são utilizados essencialmente para deslocamento na água, se transformando em estilos com técnicas e regras próprias. Ensinados, basicamente em piscina, esses estilos da natação, ainda não vieram a ser oportunizados a uma grande parcela da sociedade e respeitando os diferentes momentos históricos, a natação é uma atividade elitista. A modalidade de natação (nado crawl) foi ofertada para toda a comunidade de Vilhena, sendo realizada na piscina do IFRO Campus Vilhena. O objetivo do projeto foi oportunizar aos interessados uma vivência e aprendizado da natação. As aulas tiveram a duração de 90 minutos, ocorreram nas quartas feiras, no período matutino e vespertino. Foram ofertadas 25 vagas em cada período. Participaram do projeto 43 alunos, assim distribuídos: 4 professores, 2 técnicos administrativos do IFRO, 29 alunos internos e 8 da comunidade externa. As aulas foram desenvolvidas da seguinte forma: adaptação ao meio líquido, respiração, braçadas e pernadas. Em alguns momentos, as aulas acontecerem de forma recreativa, objetivando descontrair os alunos, principalmente os que não sabiam nadar. Nove alunos não apresentaram qualquer domínio corporal sobre a água, sendo realizado um trabalho diferenciado dos demais. Utilizou-se como materiais de apoio: pranchas, polibóia (flutuadores para as pernas e quadril) e tubos de flutuação (macarrão). Aos que dominavam a técnica do nado, foram desenvolvidas atividades de aperfeiçoamento. A análise dos dados foi observacional. As atividades foram muito satisfatórias para todos os participantes, a maioria melhorou a sua condição inicial do nado, os que não sabiam nadar, conseguiram se locomover na água, sendo este um dos primeiros passos para o aprendizado da natação. O programa continua em atividade. A natação é considerada um dos esportes mais saudáveis, pois se trabalha com diversos grupos musculares e articulações do corpo em um ambiente prazeroso e diferente do que vivemos.
	
	\vspace{\onelineskip}
	
	\noindent
	\textbf{Palavras-chave}: Natação. Iniciação. Extensão.
	
\end{document}
