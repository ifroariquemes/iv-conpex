\documentclass[article,12pt,onesidea,4paper,english,brazil]{abntex2}

\usepackage{lmodern, indentfirst, color, graphicx, microtype, lipsum, textcomp}			
\usepackage[T1]{fontenc}		
\usepackage[utf8]{inputenc}		

\setlrmarginsandblock{2cm}{2cm}{*}
\setulmarginsandblock{2cm}{2cm}{*}
\checkandfixthelayout

\setlength{\parindent}{1.3cm}
\setlength{\parskip}{0.2cm}

\SingleSpacing

\begin{document}
	
	\selectlanguage{brazil}
	
	\frenchspacing 
	
	\begin{center}
		\LARGE PROJETO DE INTERVENÇÃO \\CURRICULAR DO IFRO - PICIFRO\footnote{Trabalho realizado dentro da área de Conhecimento CNPq/CAPES: Ciências Humanas, com financiamento do IFRO/Campus Cacoal. }
		
		\normalsize
		Eduarda Fonteles\footnote{Bolsista de Projeto de Extensão: Eduarda Fonteles,  eduarda.fonteles.3.facebook@gmail.com, Campus Cacoal.} 
		Vera Lucia Lopes Silveira\footnote{Orientadora: Vera Lucia Lopes Silveira, vera.lucia@ifro.edu.br, Campus Cacoal.} 
	\end{center}
	
	\noindent Uma das políticas do IFRO é promover o desenvolvimento das comunidades em seu entorno. A partir desta visão, o projeto de extensão PICIFRO prioriza o atendimento aos alunos da escola estadual de ensino fundamental e médio Celso Ferreira da Cunha, situada no bairro Riozinho do município de Cacoal, situada a 2 km do IFRO, Campus Cacoal. O objetivo é desenvolver um projeto de intervenção na referida escola por intermédio de oficinas, visando preparar os alunos do 9º ano do ensino fundamental e 3º ano do ensino médio, para ingresso no Instituto Federal de Educação, Ciência e Tecnologia – IFRO.  A metodologia utilizada foram reuniões com equipe pedagógica da escola e professora parceira, visando definir ações do projeto e oficinas quinzenais para alunos do 9º ano do ensino fundamental e alunos do 3º ano do ensino médio. Para a primeira turma, no intuito de auxiliá-los, reforçando o aprendizado em sua proposta curricular na disciplina de língua portuguesa e para a segunda turma, visando prepará-los para a redação do Exame Nacional do Ensino Médio-ENEM. 
	Os resultados obtidos ao decorrer da realização do projeto estão diretamente relacionados ao aprendizado da língua materna, tanto no âmbito da linguagem, quanto da argumentação escrita. Houve também, um maior interesse em dar continuidade nas suas respectivas carreiras acadêmicas e profissionais; visto que os alunos se apresentavam desmotivados, diante de poucas perspectivas para o ingresso no nível superior. O que ressalta a importância da realização do projeto em uma comunidade carente como a do Riozinho, é exatamente a ideia de fomentar ações que oportunizem o vislumbre de um melhor futuro melhor, por intermédio da educação. 
	Vale destacar que o PICIFRO é resultado de um projeto de pesquisa do Mestrado Profissional em Educação – MEPE, e que esta experiência foi relatada e discutida em um livro, que será lançado no próximo mês pela editora Appris em todo o país.
	
	
	\vspace{\onelineskip}
	
	\noindent
	\textbf{Palavras-chave}: Currículo. Língua Portuguesa. Multiculturalismo.
	
\end{document}
