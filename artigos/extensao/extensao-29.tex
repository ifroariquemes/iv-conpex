\documentclass[article,12pt,onesidea,4paper,english,brazil]{abntex2}

\usepackage{lmodern, indentfirst, color, graphicx, microtype, lipsum}			
\usepackage[T1]{fontenc}		
\usepackage[utf8]{inputenc}		

\setlrmarginsandblock{2cm}{2cm}{*}
\setulmarginsandblock{2cm}{2cm}{*}
\checkandfixthelayout

\setlength{\parindent}{1.3cm}
\setlength{\parskip}{0.2cm}

\SingleSpacing

\begin{document}
	
	\selectlanguage{brazil}
	
	\frenchspacing 
	
	\begin{center}
		\LARGE PROJETO DE EXTENSÃO: CONTANDO E ENCANTANDO\footnote{Trabalho realizado dentro da área de Conhecimento CNPq/CAPES: 8.00.00.00-2 Linguística, Letras e Artes com financiamento do IFRO.}
		
		\normalsize
		Patrícia Camilo Benicio\footnote{Bolsista Ensino Médio Presencial, patriciacb2003@hotmail.com, Campus Porto Velho Zona Norte.} 
		Dinalva Barbosa da Silva Fernandes\footnote{Coordenadora, dinalva.fernandes@ifro.edu.br, Campus Porto Velho Zona Norte} 
		Ícaro Alexsander Costa\footnote{Colaborador,  icaro.costa@ifro.edu.br, Campus Porto Velho Zona Norte}
	\end{center}
	
	\noindent A leitura desempenha um importante papel na vida do ser humano. Ela é capaz de desenvolver uma concepção de mundo a partir das narrativas que são lidas. Com a habilidade da leitura é possível criar e recriar histórias para serem contadas e encantadas. Foi por pensar assim que desenvolvemos o projeto de extensão “Contando e Encantando” na Escola Municipal de Ensino Fundamental Senador Darcy Ribeiro, localizada no município de Porto Velho. O objetivo era identificar os problemas de leitura no ensino em sala de aula para, com base nos problemas levantados, propor atividades teóricas e práticas para melhorar os níveis de leitura para uma educação melhor. Os procedimentos metodológicos utilizados consistiram em fazer um levantamento de dados através de questionários aplicados para os alunos e profissionais da escola, após a catalogação de dados, houve um estudo para escolher a melhor forma de propor oficinas para os mediadores de leituras nas escolas, com isso, foi oferecido como formação complementar oficinas de fantoches e máscaras, envolvendo atividades práticas com os participantes, que promoverão a contação de histórias como incentivo ao hábito da leitura. Mesmo com as dificuldades que envolvem disponibilidade de tempo, assiduidade dos participantes, consideramos como satisfatório os resultados, pois aconteceu a oficina de fantoches, apesar de não ter acontecido à oficina de máscaras. Os participantes elogiaram o desempenho da instituição em oferecer a oficina e os materiais gratuitamente, o que os motivou a, mesmo com dificuldades, continuar frequentando as atividades oferecidas. O espaço disponibilizado pela escola estava em ótimas condições. O que mais nos deixou satisfeitas por estar ali contribuindo foi ver o desempenho de professores e alunos motivados por um amor incondicional ao confeccionar os seus fantoches e saber que professores utilizarão o seu tempo na sala de aula para confeccionar com alunos os fantoches e treinar com eles a contação de histórias.
	
	\vspace{\onelineskip}
	
	\noindent
	\textbf{Palavras-chave}: Extensão. IFRO. Leitura.
	
\end{document}
