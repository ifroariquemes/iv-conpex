\documentclass[article,12pt,onesidea,4paper,english,brazil]{abntex2}

\usepackage{lmodern, indentfirst, color, graphicx, microtype, lipsum}			
\usepackage[T1]{fontenc}		
\usepackage[utf8]{inputenc}		

\setlrmarginsandblock{2cm}{2cm}{*}
\setulmarginsandblock{2cm}{2cm}{*}
\checkandfixthelayout

\setlength{\parindent}{1.3cm}
\setlength{\parskip}{0.2cm}

\SingleSpacing

\begin{document}
	
	\selectlanguage{brazil}
	
	\frenchspacing 
	
	\begin{center}
		\LARGE A DANÇA E O TEATRO COMO FORMA DE INTERAÇÃO E APRENDIZAGEM NO CONTEXTO ESCOLAR\footnote{Trabalho realizado dentro da área de Conhecimento CNPq/CAPES: Linguística, Letras e Artes com financiamento do Instituto Federal de Educação, Ciência e Tecnologia/ PROEX.}
		
		\normalsize
		Eliane Ricarte Rodrigues\footnote{Orientadora: eliane.rodrigues@ifro.edu.br /Campus Cacoal} 
		Vera Lucia Lopes Silveira \footnote{Co-orientadora: vera.lucia@ifro.edu.br /Campus Cacoal} 
		 
	\end{center}
	
	\noindent O presente projeto tem como foco proporcionar conhecimentos acerca de duas modalidades artísticas distintas, sendo: a Dança e o Teatro. Tem como objetivo promover a integração entre ensino e extensão durante a execução do projeto, realizar apresentações artísticas nas modalidades: Dança e Teatro, com entre outros que favoreça o conhecimento por parte dos envolvidos.Segundo Strazzacappa, 2001, p. 21) “A introdução de atividades corporais artísticas na escola, ou seja, a realização de trabalhos de dança-educativa ou dança-expressiva, como são comumente chamadas [...] tem mudado significativamente as atitudes de crianças e professores na escola”.
	A dança será desenvolvida a partir de ritmos brasileiros de forma teórica e prática, será organizado por etapas e por números de eventos ocorridos nos campi IFRO no período de Maio a Novembro de 2016.  A segunda modalidade artística será organizada pela criação de um grupo de Teatro, cujo enfoque de suas peças será sempre o humor. Esta atividade será coordenada pela colaboradora do projeto. Será levado em consideração o interesse do aluno e principalmente a afinidade dele com cada modalidade e temática a ser desenvolvida. O envolvimento dos discentes será indispensável, já que o projeto visa à interação e o conhecimento dos mesmos. O projeto visa realizar apresentações no âmbito do Campus e em escolas participantes, onde os alunos do IFRO poderão compartilhar conhecimento com a comunidade do entorno à instituição. A proposta beneficiará nas duas modalidades, aproximadamente 60 alunos do Campus, envolvidos ativamente no projeto, cerca de 150 alunos da escola Celso Ferreira da Cunha e aproximadamente 400 pessoas da comunidade envolvida nos eventos do Campus.
	
	\vspace{\onelineskip}
	
	\noindent
	\textbf{Palavras-chave}: Dança. Teatro. Aprendizagem.
	
\end{document}
