\documentclass[article,12pt,onesidea,4paper,english,brazil]{abntex2}

\usepackage{lmodern, indentfirst, color, graphicx, microtype, lipsum}			
\usepackage[T1]{fontenc}		
\usepackage[utf8]{inputenc}		

\setlrmarginsandblock{2cm}{2cm}{*}
\setulmarginsandblock{2cm}{2cm}{*}
\checkandfixthelayout

\setlength{\parindent}{1.3cm}
\setlength{\parskip}{0.2cm}

\SingleSpacing

\begin{document}
	
	\selectlanguage{brazil}
	
	\frenchspacing 
	
	\begin{center}
		\LARGE RESUMO: CLUBE DA LEITURA\footnote{Trabalho realizado dentro da Área de Conhecimento CNPq/CAPES: Linguística, Letras e Artes com financiamento do Instituto Federal de Educação, Ciência e Tecnologia de Rondônia – DEPEX.}
		
		\normalsize
		Zenilda Moreira do Carmo\footnote{Bolsista estudante: Zenilda Moreira do Carmo, zenilda\_carmo@hotmail.com, Campus Porto Velho Zona Norte. } 
		Edilene Barbosa de Almeida\footnote{Bolsista estudante: Edilene Barbosa de Almeida, edilenefalca@hotmail.com, Campus Porto Velho Zona Norte.} 
		Vanessa Sá dos Santos\footnote{Bolsista estudante: Vanessa Sá dos Santos, vanessa-ssantos@outlook.com, Campus Porto Velho Zona Norte.} 
		Marcilene Assunção\footnote{(a): Marcilene Assunção, marcilenea2010@hotmail.com, UNIR.} 
		Ana Cláudia Dias Ribeiro\footnote{Orientador(a), Ana Cláudia Dias Ribeiro, ana.ribeiro@ifro.edu.br, Campus Porto Velho Zona Norte. }
		Taianni Rocha de Santana Fernandes\footnote{Co-orientador(a): Taianni Rocha de Santana Fernandes, taianni.fernandes@ifro.edu.br, Campus Porto Velho Zona Norte.}
	\end{center}
	
	\noindent O Projeto Clube da Leitura realizou-se nos meses de setembro a dezembro de 2015, na escola Santa Marcelina - Marcello Cândia/BR, Km 17, zona rural de Candeias do Jamary, com cinco turmas de 1ª série do ensino fundamental, sendo três turmas no período matutino e outras duas no vespertino. A escola é administrada pelas irmãs Marcelinas com recursos oriundos dos governos estadual e municipal. E recebe crianças e jovens de diferentes estruturas familiares, oriundas de diversos setores econômicos, culturais e religiosos. O projeto “Clube de Leitura” teve como objetivo geral fomentar a leitura, investindo na formação de uma nova geração de leitores. Formamos um grupo com três alunas bolsistas e uma voluntária, todas do IFRO, duas professoras de Língua Portuguesa do IFRO e uma professora voluntária, além da parceria com as professoras titulares das turmas trabalhadas. Dispõe de um ambiente agradável para leitura. O método utilizado foi indutivo, pois se valiam de referências de livros, dinâmicas entre os alunos, material com fichas recortadas com letras formando nomes dos personagens da história, computador, Lápis de cor, tinta para tecido, tecido, Data show, filme e cartolinas. Através de um questionário de avaliação respondido pelas professoras das turmas, mencionaram que o Projeto foi muito positivo, ainda perceberam um avanço significativo no processo de leitura/escrita dos alunos e que o projeto ajudou muito nesse processo de alfabetização e letramento. O resultado alcançado foi que todos os alunos avançaram no domínio do sistema alfabético-ortográfico de ensino que leva a uma maior autonomia na relação com o mundo letrado em que vivemos. Promovemos uma atividade que favoreceu concretização dessa aquisição dentro de um contexto significativo de escrita e leitura permitindo aos educandos se inserirem melhor na nossa sociedade permeada pela cultura escrita.
	
	\vspace{\onelineskip}
	
	\noindent
	\textbf{Palavras-chave}: Leitura. Escrita. Literatura Infantil.
	
\end{document}
