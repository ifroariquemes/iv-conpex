\documentclass[article,12pt,onesidea,4paper,english,brazil]{abntex2}

\usepackage{lmodern, indentfirst, color, graphicx, microtype, lipsum}			
\usepackage[T1]{fontenc}		
\usepackage[utf8]{inputenc}		

\setlrmarginsandblock{2cm}{2cm}{*}
\setulmarginsandblock{2cm}{2cm}{*}
\checkandfixthelayout

\setlength{\parindent}{1.3cm}
\setlength{\parskip}{0.2cm}

\SingleSpacing

\begin{document}
	
	\selectlanguage{brazil}
	
	\frenchspacing 
	
	\begin{center}
		\LARGE CIÊNCIA BÁSICA NO ENFRENTAMENTO A DEPENDÊNCIA QUÍMICA PARA ASSOCIAÇÃO RESGATE DE VIDAS E SEU ENTORNO SOCIAL\footnote{1	Trabalho realizado dentro da área Ciências Humanas com financiamento do Instituto Federal de Educação, Ciência e Tecnologia de Rondônia.}
		
		\normalsize
		Wêmilly Cristina Reis Teixeira\footnote{Bolsista (extensão), wemillycristinar@gmail.com, Campus Ji-Paraná} 
		Weliton do Nascimento Alexandre\footnote{Colaboradores, weliton.nascimento98@gmail.com,  Campus Ji-Paraná} \\
		Fernanda Rodrigues de Siqueira\footnote{Colaboradores, fernanda.siqueira@ifro.edu.br, Campus Ji-Paraná} 
		Alexandre Machado da Silva\footnote{Colaboradores, alexandre.silva@ifro.edu.br,  Campus Ji-Paraná}
		Ana Maria de Paula Cordeiro\footnote{Colaborador, anaspcordeiro@hotmail.com, Campus Ji-Paraná}
		Alice Cristina Souza Lacerda Melo Souza\footnote{Orientadora, alice.cristina@ifro.edu.br, Campus Ji-Paraná} 
	\end{center}
	
	\noindent O projeto Ciência Básica no enfrentamento à Dependência Química para Associação Resgate de Vidas e seu entorno social destacou-se pelo seu caráter preventivo e por desenvolver um trabalho entre dependentes Químicos. Tendo por objetivo ofertar palestras educativas e oficinas com divulgação da ciência básica visando o combate a dependência química tanto aos internos como a comunidade, oportunizando a inclusão social por meio da geração de renda e melhoria da qualidade de vida, destacando o papel social da escola e atendendo as diretrizes curriculares que exigem o que os temas transversais ética, pluralidade cultural, saúde e trabalho sejam abordados. Aos internos da associação foram ministradas palestras referentes à higiene, motivação, meio ambiente e cursos produção de caixas de presente e sabão caseiro, este também foi ofertado aos adultos da comunidade externa. Para as crianças e adolescentes foram realizadas oficinas pedagógicas com temas voltados às drogas e ao tabaco e promovida uma visita ao Instituto Federal de Educação, Ciência e Tecnologia de Rondônia. Os objetivos do projeto foram alcançados sendo destacados a atenção dispensada aos internos que puderam usufruir de palestras educacionais e cursos. Essas atividades contribuíram significativamente para a proposta de melhoria de saúde e da autoestima tão comprometida num grupo de alta vulnerabilidade. Os cursos de caixas de presente e sabão contemplaram uma média de 20 pessoas e foram um estímulo à criatividade, à capacidade de produção e sobretudo à valorização do ser humano. Quanto às atividades destinadas às crianças a mesma teve um grande impacto social, os integrantes do projeto tiveram contato com experiências educacionais bem fundamentadas: palestras, oficinas, construção de material pedagógico bem como tiveram a oportunidade de novas experiências. Na visita realizada ao Instituto o conhecimento sobre química através dos experimentos realizados foi priorizado. Quanto à distribuição do kit educativo a ação foi positiva. Destaca-se ainda a elaboração de um almanaque produzido pelos participantes no decorrer das oficinas pedagógicas que abordaram os temas tabaco e drogas.
	
	\vspace{\onelineskip}
	
	\noindent
	\textbf{Palavras-chave}: Oficinas. Palestras. Dependentes químicos.
	
\end{document}
