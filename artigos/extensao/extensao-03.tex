\documentclass[article,12pt,onesidea,4paper,english,brazil]{abntex2}

\usepackage{lmodern, indentfirst, color, graphicx, microtype, lipsum}			
\usepackage[T1]{fontenc}		
\usepackage[utf8]{inputenc}		

\setlrmarginsandblock{2cm}{2cm}{*}
\setulmarginsandblock{2cm}{2cm}{*}
\checkandfixthelayout

\setlength{\parindent}{1.3cm}
\setlength{\parskip}{0.2cm}

\SingleSpacing

\begin{document}
	
	\selectlanguage{brazil}
	
	\frenchspacing 
	
	\begin{center}
		\LARGE ROBÓTICA PARA TODOS\footnote{Trabalho realizado dentro da área Ciências Humanas com financiamento do Instituto Federal de Educação, Ciência e Tecnologia de Rondônia.}
		
		\normalsize
		Álvaro Victor de Oliveira Aguiar\footnote{Álvaro Victor de Oliveira Aguiar (Bolsista), alvarovctoliveira@gmail.com , Campus Guajará-Mirim} 
		Bryan da Silva Moreira\footnote{Bryan da Silva Moreira (Colaborador), bryanmoreira24@gmail.com , Campus Guajará-Mirim} 
		Davi Pereira Rossell\footnote{Davi Pereira Rossell (Colaborador), davipereirarossell11@gmail.com , Campus Guajará-Mirim} 
		Angelo Maggioni e Silva\footnote{Angelo Maggioni e Silva (Coordenador), angelo.silva@ifro.edu.br , Campus Guajará-Mirim.}
		Rogério Lopes Vieira Cesar\footnote{Lopes Vieira Cesar, rogerio.cesar@ifro.edu.br, Campus Guajará-Mirim.} 
	\end{center}
	
	\noindent O projeto de extensão “Robótica para Todos” teve como objetivo, aproximar uma seleção de alunos das comunidades interna e externa do IFRO Campus Guajará-Mirim ao domínio robótico, capacitando-os para desenvolver processos básicos com a iniciação na plataforma Arduíno. Sendo que através de oficinas bem estruturadas eles poderiam estar inclusos na área de robótica, também os instruindo a aperfeiçoar-se para desenvolver processos mais complexos posteriormente. O projeto de extensão tem o seu desenvolvimento no município de Guajará-Mirim, mais precisamente no Instituto Federal de Educação, Ciência e Tecnologia de Rondônia (IFRO), no Campus da mesma cidade. Visando explorar soluções práticas do domínio robótico relacionando a teoria vista em sala de aula com artefatos autônomos utilizando o Arduíno, com uma proposta de desenvolvimento de aulas práticas de robótica com auxilio dessas placas, no qual os membros da
	
	comunidade regional,   preferencialmente	jovens,   teriam a   oportunidade   de
	conhecer	e desenvolver	atividades	previamente	idealizadas,	assim
	
	compartilhando as experiências dos ministrantes com os inscritos. O trabalho foi estruturado em três etapas, sendo a primeira a elaboração do material, desmembrando o grupo em duas duplas que planejaram como se procederia às suas aulas, a primeira preferiu aplicações de atividades com leds, sendo que a segunda optou por trabalhar com servos motores, posteriormente a seleção dos participantes das aulas e por último a culminância do projeto, aplicando todo o material desenvolvido. Lecionaram-se aulas nas quais sucedeu a apresentação da teoria, posteriormente o desmembramento da classe para grupos menores em que cada indagação fosse esclarecida, a fim de tornar a temática robótica um assunto nítido e fácil de ser assimila do por todos. O ensino proporcionado através das aulas lecionadas obterá sucesso, pois a estruturação de todo o projeto oportunizou isso, desde a escolha da placa Arduíno para se trabalhar até a abordagem do tema de forma simples. Tento que a robótica é compreendida como uma ferramenta inclusiva oportuniza aos que necessitam de alguma necessidade específica um meio de exercitar sua criatividade com projetos práticos. A iniciação desses participantes na robótica foi importante para a criação de uma massa de inventores na região que em algum momento irão compartilhar o conhecimento com outros.
	
	
	\vspace{\onelineskip}
	
	\noindent
	\textbf{Palavras-chave}: Oficina. Arduíno. Inclusão..
	
\end{document}
