\documentclass[article,12pt,onesidea,4paper,english,brazil]{abntex2}

\usepackage{lmodern, indentfirst, color, graphicx, microtype, lipsum}			
\usepackage[T1]{fontenc}		
\usepackage[utf8]{inputenc}		

\setlrmarginsandblock{2cm}{2cm}{*}
\setulmarginsandblock{2cm}{2cm}{*}
\checkandfixthelayout

\setlength{\parindent}{1.3cm}
\setlength{\parskip}{0.2cm}

\SingleSpacing

\begin{document}
	
	\selectlanguage{brazil}
	
	\frenchspacing 
	
	\begin{center}
		\LARGE DESENVOLVIMENTO DE UMA EMPRESA CENTRADA EM PERSIANAS AUTOMATIZADAS\footnote{Trabalho realizado dentro da área de Conhecimento CNPq/CAPES: Administração de Empresas e Engenharia Elétrica, com financiamento próprio.}
		
		\normalsize
		Débora de Azevedo Rodrigues\footnote{Bolsista PIBIC, debora9rodrigues@gmail.com , Campus Porto Velho Calama} 
		Carlos Vitor Gomes Faria\footnote{Voluntário , carlosvvitor@outlook.com, Campus Porto Velho Calama} 
		Isa Thalya Mendes Terço\footnote{Voluntária, isathalyamt@gmail.com, Campus Porto Velho Calama} \\
		Itacatiane Brilhante da Silva\footnote{Voluntária, ita.brilhante@hotmail.com, Campus Porto Velho Calama}
		Jaine do Nascimento Xavier\footnote{Bolsista PIBIC, jainexvr@hotmail.com, Campus Porto Velho Calama}\\
		Luiz Fernando Albuquerque Barbosa\footnote{Bolsista PIBIC, fernando.albuquerque2310@outlook.com, Campus Porto Velho Calama}
		Lígia Silvéria Vieira da Silva\footnote{Orientadora, ligia.silva@ifro.edu.br, Campus Porto Velho Calama}
	\end{center}
	
	\noindent Automação residencial é assunto bem atual nos dias de hoje, o que se faz necessário a busca por conhecimentos nessa área, seja na área científica como empresarial. Desta forma, o presente trabalho apresenta o desenvolvimento de uma empresa no ramo de automação residencial e comercial, de forma que o foco é a instalação de persianas inteligentes, o que possibilita ao cliente uma economia de energia além do conforto no ambiente. Para que a empresa fosse então idealizada, o grupo buscou por materiais bibliográficos de cunho empresarial para que o plano de negócios fosse montado, de tal modo a levar em considerações o atual mercado da cidade, potencialidade da empresa, aspectos positivos e possíveis pontos de dificuldades. Sendo assim, o nome fictício escolhido foi “SWS - persianas inteligentes”. Assim, a SWS visa instalar sistemas com capacidade de permitir o movimento de suas persianas através de sensores, de modo a otimizar a utilização da própria luz solar para iluminar o ambiente, diminuindo o consumo de energia elétrica das lâmpadas convencionais, além de controlar a temperatura do condicionador de ar, aumentando o conforto às pessoas presentes no recinto. De modo a representar o potencial da empresa, a equipe fez uso de recursos tecnológicos da área técnica e desenvolveu um protótipo de uma persiana instalada em uma maquete, representando uma janela, com os devidos equipamentos que reproduzissem um ambiente real, como é o caso do cooler ao invés da central de ar. O controle da persiana é feito através de motores, que podem ser comandados de forma automática, através de um arduino, ou de forma manual por meio de um botão. Para a percepção da intensidade de luz solar, são utilizados resistores dependentes de luz, em que seus sinais são enviados ao sistema embarcado presente no arduino. Com o projeto do protótipo pronto foi possível perceber a aplicabilidade desse sistema de persianas automáticas, além conseguir mostrar a eficiência desse tipo de instalação. Então, a abertura de uma empresa no ramo de automação residencial aplicado a persianas inteligentes possui grande potencial no mercado, tendo em vista a otimização e conforto no ambiente.
	
	\vspace{\onelineskip}
	
	\noindent
	\textbf{Palavras-chave}: Empresa. Persiana automatizada. Automação residencial.
	
\end{document}
