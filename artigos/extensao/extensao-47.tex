\documentclass[article,12pt,onesidea,4paper,english,brazil]{abntex2}

\usepackage{lmodern, indentfirst, color, graphicx, microtype, lipsum}			
\usepackage[T1]{fontenc}		
\usepackage[utf8]{inputenc}		

\setlrmarginsandblock{2cm}{2cm}{*}
\setulmarginsandblock{2cm}{2cm}{*}
\checkandfixthelayout

\setlength{\parindent}{1.3cm}
\setlength{\parskip}{0.2cm}

\SingleSpacing

\begin{document}
	
	\selectlanguage{brazil}
	
	\frenchspacing 
	
	\begin{center}
		\LARGE PROJETO CANTA JUVENTUDE\footnote{Trabalho realizado dentro da área de Ciências Humanas com financiamento do IFRO.}
		
		\normalsize
		Fernanda Kellen dos Santos Almeida\footnote{Bolsista de Extensão, fernandakellendsa@gmail.com , Campus Porto Velho Calama.} 
		Luiz Henrique Parloti da Silva\footnote{Bolsista de Extensão, luizparloti@hotmail.com , Campus Porto Velho Calama.} \\
		Madson Silva de Souza Junior\footnote{Colaborador, madsonsleeve@gmail.com , Campus Porto Velho Calama.} 
		Xênia de Castro Barbosa\footnote{Orientadora, xenia.castro@ifro.edu.br , Campus Porto Velho Calama.} \\
		Alexandre Santos de Oliveira\footnote{Co-orientador, alexandre.oliveira@ifro.edu.br , Campus Porto Velho Calama.}
		Tiago Lins de Lima\footnote{ Colaborador do projeto, tiago.lins@ifro.edu.br, Reitoria IFRO.}
	\end{center}
	
	\noindent Canta Juventude é um projeto de extensão desenvolvido no Campus Porto Velho Calama, com recurso da PROEX e do DEPEX do referido Campus. Tem como objetivo a realização de um CD de músicas expressivas da cultura rondoniense, em especial daquela presente no Território Rural de Identidade Rio Machado. Músicas que serão cantadas pelo Coral do IFRO e seus convidados (jovens rurais e urbanos e indígenas). Estima-se, assim, promover o conhecimento sobre a cultura local e sua musicalidade, aprofundando relações interétnicas pautadas no respeito e na colaboração. O Método que dá suporte ao estudo é o materialismo histórico-dialético, que tem por princípio o estudo das bases materiais que dão suporte à vida social, evidenciando suas contradições, dinâmicas e desafios. Para apreender as dinâmicas culturais apresentadas pelos indígenas servimo-nos de procedimentos da pesquisa etnográfica, a saber, a imersão em campo, a observação participante e o registro fotográfico de rituais e objetos da cultura material. A oitiva e o ensaio das músicas também é atividade cotidiana. Diante da complexidade dos processos históricos vivenciados por aquela comunidade e da riqueza daquela cultura, consideramos necessário aprofundar o trabalho de campo e os registro etnográficos para então chegarmos ao objetivo específico de identificar, aprender e reproduzir fonograficamente, ao menos uma música expressiva da identidade daquela etnia. Até o momento foi possível ampliar os conhecimentos acerca da diversidade étnica e cultural da sociedade rondoniense e aprofundar o estudo teórico e prático sobre o canto. A problemática em tela trás como desafio pessoal e institucional o maior envolvimento com as comunidades tradicionais e o apoio para registro e divulgação de sua cultura nos termos definidos pelas próprias comunidades.
	
	\vspace{\onelineskip}
	
	\noindent
	\textbf{Palavras-chave}: Relações Interétnicas. Comunidades Tradicionais. Música.
	
\end{document}
