\documentclass[article,12pt,onesidea,4paper,english,brazil]{abntex2}

\usepackage{lmodern, indentfirst, color, graphicx, microtype, lipsum}			
\usepackage[T1]{fontenc}		
\usepackage[utf8]{inputenc}		

\setlrmarginsandblock{2cm}{2cm}{*}
\setulmarginsandblock{2cm}{2cm}{*}
\checkandfixthelayout

\setlength{\parindent}{1.3cm}
\setlength{\parskip}{0.2cm}

\SingleSpacing

\begin{document}
	
	\selectlanguage{brazil}
	
	\frenchspacing 
	
	\begin{center}
		\LARGE ESPORTE NA ESCOLA: DIVERSIDADE NA CULTURA ESPORTIVA DA ESCOLA\footnote{Trabalho realizado dentro da (área de Conhecimento CNPq/CAPES: Educação) com financiamento do Instituto Federal de Rondônia/PROEX}
		
		\normalsize
		Bernardo Nomerg Ferreira\footnote{Colaborador, bernardo.nomerg@ifro.edu.br  Campus Cacoal} 
		Flávia Heloisa da Silva\footnote{Colaboradora, flavia.silva@ifro.edu.br  Campus Cacoal} 
		Adriano Robson Nogueira de Lucena\footnote{Orientador, adriano.lucena@ifro.edu.br  Campus Cacoal} 
		Juliano Alves de Deus\footnote{Co-orientador, juliano.alves@ifro.edu.br  Campus Cacoal} 
	\end{center}
	
	\noindent O presente trabalho objetivou atender ao anseio da comunidade escolar do Instituto Federal de Rondônia/IFRO Campus Cacoal, dessa que vê no esporte uma possibilidade de exercício social da cidadania, com vistas a aquisição de hábitos positivos para a prática esportiva, seja formal ou informal, além de identificar-se podendo adotá-la para além da idade escolar, pois o esporte como aspecto gerador de intenções e motivações corporais torna-se um veículo de grande alcance social, forjando um perfil de cidadão ativo em suas práticas de cultura corporal. Para a formação das equipes, foi utilizado o critério técnico na modalidade escolhida pelo participante, sendo selecionados aqueles com capacidades técnicas adequadas a cada modalidade, sendo também adotado o critério de interesse em participar da modalidade, dando assim, a chance do interessado mostrar sua aptidão física e técnica para as modalidades escolhidas, além de democratizar a participação da comunidade escolar no esporte da escola, obtendo a participação da comunidade escolar do ensino médio em torno de 250 participantes, sendo 106 escolhidos para representarem as equipes nas modalidades esportivas. Os treinamentos foram realizados no Campus Cacoal do IFRO. Foram obtidos resultados relevantes no aumento da participação dos alunos nas modalidades esportivas nos Jogos Escolares de Rondônia/JOER, bem como o interesse na participação de uma maior diversidade cultural esportiva no âmbito do IFRO. Conforme afirma (BRASIL, 1999), o esporte atua na sociedade como aspecto gerador de intenções e motivações corporais de grande alcance social. É válido considerar sua importância social que carrega em sua essência, o benefício para a formação cidadã com posturas ética e moral sociointeracionista (DARIDO, 1999). Além do aspecto educacional, Tubino (2001) afirma que “O esporte é para todos, tem significado social meio a socialização, desenvolve a consciência comunitária, sendo atividade de prazer e equilíbrio social, e um direito de todos, como meio de democratização”.
	
	\vspace{\onelineskip}
	
	\noindent
	\textbf{Palavras-chave}: Esporte. Escola. Educação.
	
\end{document}
