\documentclass[article,12pt,onesidea,4paper,english,brazil]{abntex2}

\usepackage{lmodern, indentfirst, color, graphicx, microtype, lipsum}			
\usepackage[T1]{fontenc}		
\usepackage[utf8]{inputenc}		

\setlrmarginsandblock{2cm}{2cm}{*}
\setulmarginsandblock{2cm}{2cm}{*}
\checkandfixthelayout

\setlength{\parindent}{1.3cm}
\setlength{\parskip}{0.2cm}

\SingleSpacing

\begin{document}
	
	\selectlanguage{brazil}
	
	\frenchspacing 
	
	\begin{center}
		\LARGE EMPREENDEDORISMO DIGITAL EM GUAJARÁ-MIRIM: BUSCA, PUBLICIDADE E PROPAGANDA WEB\footnote{Trabalho realizado dentro da área de Conhecimento CNPq/CAPES: Ciências Sociais Aplicadas, com financiamento do IFRO.}
		
		\normalsize
		Tiago Ramos\footnote{Bolsista extensionista, tiagoramosnm@gmail.com, Campus Guajará-Mirim.} 
		Nahele Melgar\footnote{Colaborador(a), nahele.ribeiro@gmail.com, Campus Guajará-Mirim.} 
		Jhordano Malacarne\footnote{Orientador(a), jhordano@ifro.edu.br, Campus Guajará-Mirim.} 
		Saiane Barros\footnote{Co-orientador(a), saiane.souza@ifro.edu.br, Campus Cacoal.} 
	\end{center}
	
	\noindent Guajará-Mirim é o segundo município a ser criado no Estado de Rondônia e faz fronteira com a Bolívia. Possui três principais segmentos que fortalecem a presença de movimentação de pessoas sendo eles: área de livre comércio; turismo, em especial para compras no comércio boliviano; e, órgão públicos. O público diversificado fomenta a diversificação dos serviços prestados e, consequentemente a necessidade de comunicação com foco no marketing, como instrumento para oportunizar os negócios locais. Com a internet como instrumento de uso cotidiano pelas pessoas, a mesma se torna uma grande aliada para a publicidade dos negócios, demonstrando assim a importância de que as empresas devem estar presentes na internet por meio sites e perfis em redes sociais. Dessa forma, por meio de uma pesquisa na região de Guajará-Mirim, identificou-se diversas empresas que não encontram-se na internet o que indicou a necessidade da criação um site que concentrem as informações das empresas para que as mesmas pudessem ser encontradas de maneira online. Para isso, realizou-se parceria junto à Associação Comercial, Industrial e Serviços de Guajará-Mirim e apresentou-se o projeto, além de serem feitas visitas presenciais nas empresas para registrar fotos, localização geográfica, realização de orientação dos serviços disponibilizados bem como assinatura de contratos. Os trabalhos desenvolvidos no período de 12 de maio a 12 de agosto de 2016 contaram com 2 alunos extensionistas e outros 3 colaboradores em que, foi realizada a criação de um site disponibilizado em guajaralocais.com; contas nas redes sociais Facebook, Instagram e Twitter. Com os contratos assinados, os dados das empresas foram publicados em páginas personalizadas  que contém além de fotos, formas de contatos e localização geográfica. Sendo assim, por se tratar de instrumento online de publicidade e propaganda, utilizou-se exclusivamente dados quantitativos para mensuração do alcance a estas empresas em que, obteve 331 curtidas na fanpage no Facebook que possibilitou o alcance de 857 pessoas distribuídas no Brasil (majoritáriamente no estado de Rondônia) e na Bolívia.
	
	\vspace{\onelineskip}
	
	\noindent
	\textbf{Palavras-chave}: Guajará-Mirim. Negócios. Internet.
	
\end{document}
