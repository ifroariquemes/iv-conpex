\documentclass[article,12pt,onesidea,4paper,english,brazil]{abntex2}

\usepackage{lmodern, indentfirst, color, graphicx, microtype, lipsum}			
\usepackage[T1]{fontenc}		
\usepackage[utf8]{inputenc}		

\setlrmarginsandblock{2cm}{2cm}{*}
\setulmarginsandblock{2cm}{2cm}{*}
\checkandfixthelayout

\setlength{\parindent}{1.3cm}
\setlength{\parskip}{0.2cm}

\SingleSpacing

\begin{document}
	
	\selectlanguage{brazil}
	
	\frenchspacing 
	
	\begin{center}
		\LARGE MULHERES DO BAIRRO IGARAPÉ: ARTESANATO COMO FERRAMENTA DE INCLUSÃO PRODUTIVA E RESGATE DA AUTO ESTIMA\footnote{Trabalho realizado dentro da (área de Conhecimento CNPq/CAPES: Interdisciplinar) com financiamento do IFRO/PROEX.}
		
		\normalsize
		Danielly Farias da Silva \footnote{Bolsista, danyellyfa@hotmail.com Campus Porto Velho Calama.} 
		Érica Cristina Dos Santos Bueno\footnote{Bolsista, cerica302@gmail.com  Campus Porto Velho Calama.} 
	    Josélia Fontenele Batista\footnote{Orientador(a), joselia.fontenele@ifro.edu.br, Campus Porto Velho Calama.} 
	  
	\end{center}
	
	\noindent A inclusão produtiva constitui-se num grande desafio em tempos de crise no Brasil em função da diminuição constante dos postos de trabalho formais.  Dessa forma, o trabalho informal pode ser uma alternativa aos grupos que se encontram excluídos do mercado de trabalho formal, por fatores diversos como a baixa escolaridade ou a impossibilidade de trabalhar em função da inexistência de redes de apoio para o cuidado dos filhos, dos baixos salários e da ausência de vagas em creches para toda a população carente, entre outros. A mulher é, muitas vezes, penalizada em termos de inclusão produtiva em razão dos múltiplos papéis que dela são requeridos os quais, muitas vezes, ela não consegue atender, gerando conflitos sociais e pessoais. A mulher fica marginalizada do acesso ao saber, da produção e não se emancipa daqueles que muitas vezes a oprimem e a exploram das muitas formas conforme indicado no trabalho Mulheres no Mundo do Trabalho – Tendências 2016 da Organização Internacional do Trabalho -OIT. Este projeto não só buscou a inclusão produtiva, mas também fomentar o empoderamento feminino a partir do resgate da sua autoestima e dos aspectos motivacionais da autoria de sua própria história podem promover. As mulheres alvo para este projeto foram aquelas que estivessem desempregadas e com filhos, situação esta que constitui-se em vulnerabilidade à exposição a risco de envolverem-se em atividades ilícitas, sujeitas a manutenção de vínculos com violência, exposição a atividades laborais sub-humanas, entre outras. Metodologia: Foram realizados, no mínimo 4 cursos simultâneo com a participação de 5 a 8 pessoas (Confecção de Sandálias com miçangas, Pano de Prato Costurado, Tiaras com laços e fitas, e Design de sobrancelhas) e uma oficina de unificação de todas as participantes com cerca de 22 pessoas de empreendedorismo feminino e marketing para auxiliar na comercialização dos produtos.  Resultados: Foram confeccionados produtos, utilizando os insumos e a doação de kits de confecção (tesouras, réguas, agulhas, linhas, além dos produtos produzidos) para garantir a emancipação produtiva uma vez que, sem a detenção dos meios de produção (Karl Marx) a capacitação na técnica por si só não é suficiente para assegurar a emancipação produtiva.
	
	\vspace{\onelineskip}
	
	\noindent
	\textbf{Palavras-chave}: Inclusão Produtiva. Mulheres. Empreendedorismo.
	
\end{document}
