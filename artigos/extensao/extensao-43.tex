\documentclass[article,12pt,onesidea,4paper,english,brazil]{abntex2}

\usepackage{lmodern, indentfirst, color, graphicx, microtype, lipsum}			
\usepackage[T1]{fontenc}		
\usepackage[utf8]{inputenc}		

\setlrmarginsandblock{2cm}{2cm}{*}
\setulmarginsandblock{2cm}{2cm}{*}
\checkandfixthelayout

\setlength{\parindent}{1.3cm}
\setlength{\parskip}{0.2cm}

\SingleSpacing

\begin{document}
	
	\selectlanguage{brazil}
	
	\frenchspacing 
	
	\begin{center}
		\LARGE JOGOS INTER-CLASSE DO IFRO CAMPUS ZONA NORTE\footnote{Trabalho realizado dentro da área de Conhecimento CNPq/CAPES:Educação , com financiamento do Departamento de Extensão do Campus Porto Velho Zona Norte}
		
		\normalsize
		Joabe Vieira Costa\footnote{Bolsista de extensão, email joabepvh4@gmail.com, Campus Porto Velho Zona Norte} 
		Lidiane Cristina J. G. Jardim\footnote{Colaborador(a), email lidiane.jardim@ifro.edu.br, Campus Porto Velho Zona Norte} 
		Thiago P. Lima\footnote{Orientador(a), email thiago.lima@ifro.edu.br, Campus Porto Velho Zona Calama} 
		Ilma Paula C. Silva\footnote{Co-orientador(a), email ilma.silva@ifro.edu.br, Campus Porto Velho Zona Norte} 
	\end{center}
	
	\noindent Este texto apresenta uma reflexão sobre as contribuições e possibilidades da prática esportiva junto aos alunos dos cursos técnicos concomitantes do IFRO Campus Porto Velho Zona Norte. Ao proporcionar situações desafiantes e envolventes, a prática esportiva estimula os alunos para a aprendizagem, facilitando e enriquecendo a atividade docente. Por ser um Campus que ofertava apenas cursos subsequentes e de graduação, a prática esportiva entre os alunos nunca foi muito presente, por essa razão, os técnicos administrativos em educação do Campus decidiram organizar pela primeira vez, jogos inter-classe entre os estudantes matriculados nos cursos concomitantes. O projeto foi aprovado em edital da Reitoria de incentivo às práticas esportivas e previa além da inclusão do esporte no dia a dia dos estudantes, o acompanhamento do rendimento escolar, do rendimento físico e a participação em ações cidadãs. Os alunos participantes puderam participar dos Jogos Internos do IFRO - JIFRO, onde conquistaram a medalha de bronze no futsal feminino. Dados preliminares do acompanhamento dos estudantes que participaram do projeto revelam que a prática esportiva contribuiu para a integração entre os estudantes, os mesmos se sentiram motivados em representar seu Campus nos Jogos, e muitos já ficaram pensando nos jogos de 2017, mostrando que a prática esportiva pode sim motivar o estudante a continuar na instituição. Paralelamente o acompanhamento do rendimento escolar tem se mostrado satisfatório, o que tem sido considerado positivo, uma vez que se mostra possível agregar esporte com bons rendimentos acadêmicos, e quando é somado a isso as ações cidadãs, grande contribuição é feita para a formação integral dos alunos.
	
	\vspace{\onelineskip}
	
	\noindent
	\textbf{Palavras-chave}: Esporte. Técnico concomitante. JIFRO.
	
\end{document}
