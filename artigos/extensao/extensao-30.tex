\documentclass[article,12pt,onesidea,4paper,english,brazil]{abntex2}

\usepackage{lmodern, indentfirst, color, graphicx, microtype, lipsum}			
\usepackage[T1]{fontenc}		
\usepackage[utf8]{inputenc}		

\setlrmarginsandblock{2cm}{2cm}{*}
\setulmarginsandblock{2cm}{2cm}{*}
\checkandfixthelayout

\setlength{\parindent}{1.3cm}
\setlength{\parskip}{0.2cm}

\SingleSpacing

\begin{document}
	
	\selectlanguage{brazil}
	
	\frenchspacing 
	
	\begin{center}
		\LARGE CURSO: NOÇÕES DE TORNO MECÂNICO\footnote{Trabalho realizado dentro da área de Ensino com financiamento do IFRO - Campus Vilhena (2016).}
		
		\normalsize
	Paulo César Macedo \footnote{Coordenador, Paulo.macedo@ifro.edu.br, IFRO – Campus Vilhena.} 
	Vítor Gustavo Rocha \footnote{Vítor Gustavo Rocha, vitorrcha-2000@hotmail.com, IFRO – Campus Vilhena.} 
	\end{center}
	
	\noindent O projeto Curso de noções de torno mecânico é uma subárea do conhecimento que propôs o ensino de operações básicas com tornos, em seu aspecto global, usando conceitos mecânicos e físicos para a aprendizagem. Foi realizado pelo aluno bolsista que estava concluindo o curso técnico em eletromecânica, Vítor Gustavo Rocha e pelo coordenador do projeto que é técnico em eletromecânica e responsável pelo laboratório de usinagem do Campus, Paulo Cesar Macedo. A oferta de cursos de curta duração é de suma importância para a comunidade, pois tem como foco o desenvolvimento prático, e o curso técnico em Eletromecânica do Instituto Federal de Rondônia (IFRO) tem tido dificuldade para oferecer essas aulas práticas. O curso também proporcionou maior contato com as empresas da área de usinagem, uma vez que, estas, poderão apresentar situações problemas no que se refere a usinar. Para os participantes do curso, espera-se melhorar o conhecimento visando à inserção dos alunos no mercado de trabalho voltado para a área. Propomos a oferta deste curso aos alunos do IFRO - Campus Vilhena e foi realizado no laboratório de usinagem do Campus, teve como participantes os alunos: Vítor Rocha, Felippe Antonio, Matheus Vieira, Wesley Pablo, Gabriel Felipe, Gabriel Felberg, Renan Sousa, Lucas Eduardo, Pedro Henrique, Leonardo Colombo, Éctor Miguel, Gabriel Marques e utilizamos no decorrer do projeto o torno mecânico, ferramentas de usinagem do torno, produtos para limpeza e manutenção básica, também utilizamos a maquina de solda, pois é um conhecimento importante pra quem vai seguir esta área profissional. O curso de torno mecânico permitiu amplo conhecimento na operação da máquina, desde apresentações de ferramentas e componentes do torno, cuidados de segurança e manutenção básica (como limpeza, regulagem de alguns componentes fundamentais no torno e fazer a troca do óleo refrigerante) até conhecimentos de torneamento externo e interno, rosca, faceamento, sangrias, recartilho, corte com bedame, metrologia (paquímetro), uso de lixadeira, rebaixo externo e processo de soldagem com eletrodo revestido. O curso proporcionou para a instituição uma melhor qualificação dos alunos concluintes do curso técnico em eletromecânica e auxiliou na manutenção do patrimônio escolar. 
	
	\vspace{\onelineskip}
	
	\noindent
	\textbf{Palavras-chave}: Torno Mecânico e Usinagem.
	
\end{document}
