\documentclass[article,12pt,onesidea,4paper,english,brazil]{abntex2}

\usepackage{lmodern, indentfirst, color, graphicx, microtype, lipsum}			
\usepackage[T1]{fontenc}		
\usepackage[utf8]{inputenc}		

\setlrmarginsandblock{2cm}{2cm}{*}
\setulmarginsandblock{2cm}{2cm}{*}
\checkandfixthelayout

\setlength{\parindent}{1.3cm}
\setlength{\parskip}{0.2cm}

\SingleSpacing

\begin{document}
	
	\selectlanguage{brazil}
	
	\frenchspacing 
	
	\begin{center}
		\LARGE HISTÓRIA, CINEMA \& TECNOLOGIA\footnote{Trabalho realizado dentro da (área de Conhecimento CNPq/CAPES: Ciências Humanas) com financiamento do DEPEX/IFRO.}
		
		\normalsize
	Alessandra Francisca de Souza\footnote{Bolsista (Extensão), alessadra\_isatkm@hotmail.com, IFRO – Campus Vilhena.} 
	Márcio Marinho Martins\footnote{Orientador, marcio.martins@ifro.edu.br, IFRO – Campus Vilhena.}  
	\end{center}
	
	\noindent O projeto visou aproximar o Instituto Federal de Educação de Rondônia – Campus de Vilhena da população local, por intermédio da exibição de películas (filmes) seguida de palestras e discussões das temáticas apresentadas. A produção cinematográfica é uma das fontes alternativas para compreendermos a realidade histórica e social. A utilização do cinema como elemento de apoio no processo ensino e aprendizagem possibilita fazer uso de uma linguagem dinâmica e atual para a difusão do saber, enquanto um poderoso recurso pedagógico, visto que o audiovisual exerce uma função informativa alternativa, que exemplifica conceitos abstratos, amplia concepções e pontos de vistas e estimula a reflexão sobre fatos e acontecimentos. Do sentido comercial ao interesse da pesquisa histórica, o cinema passou concomitantemente ao centro das discussões entre historiadores e educadores ao campo do ensino, visto como um instrumento de possibilidades didáticas variadas. Marc Ferro (1992) defendeu que o cinema é um testemunho singular do seu tempo e, segundo Vovelle (1987), a iconografia (imagens e cinema), se tornou fonte privilegiada de conhecimento a partir do século XX. Segundo Nóvoa (1995), é preciso valorizar o cinema como um documento válido para a discussão historiográfica. Com base nesse referencial procuramos compreender o papel central da imagem e do cinema na sociedade contemporânea e de como inseri-lo enquanto fonte historiográfica. As sessões de cinema foram espaço de discussão de temáticas relacionadas à história e realidade social, perfazendo um total de 10 sessões e destinou-se a uma média de público de 80 estudantes por sessão, oriundos de escolas da rede pública estadual. Antes de cada sessão foi feita uma apresentação sobre o IFRO, os cursos existentes e as formas de ingresso. Após a exibição do filme seguia-se uma palestra seguida de debate. O projeto também previu a aquisição de um acervo de filmes a ser disponibilizado na biblioteca do Campus. Como resultado do projeto, conseguimos envolver cerca de 800 estudantes da rede pública estadual. Também foi possível observar a presença de populares, professores da rede pública e também de estudantes do IFRO que de forma espontânea e seguindo os interesses por cada temática participaram das sessões.
	
	\vspace{\onelineskip}
	
	\noindent
	\textbf{Palavras-chave}: Historiografia. Cinema. Conhecimento
	
\end{document}
