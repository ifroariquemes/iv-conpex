\documentclass[article,12pt,onesidea,4paper,english,brazil]{abntex2}

\usepackage{lmodern, indentfirst, color, graphicx, microtype, lipsum}			
\usepackage[T1]{fontenc}		
\usepackage[utf8]{inputenc}		

\setlrmarginsandblock{2cm}{2cm}{*}
\setulmarginsandblock{2cm}{2cm}{*}
\checkandfixthelayout

\setlength{\parindent}{1.3cm}
\setlength{\parskip}{0.2cm}

\SingleSpacing

\begin{document}
	
	\selectlanguage{brazil}
	
	\frenchspacing 
	
	\begin{center}
		\LARGE INCLUSÃO ESCOLAR: “CONVIVENDO COM A DIVERSIDADE”\footnote{Trabalho realizado dentro da Área Educação, com financiamento do Instituto Federal de Educação Ciência e Tecnologia de Rondônia - IFRO}
		
		\normalsize
		Weliton do Nascimento Alexandre\footnote{Bolsista Ensino Médio, email: weliton.nascimento98@gmail.com, Campus Ji-Paraná} 
		Nayara Gomes de Oliveira \footnote{Bolsista Ensino Médio, email: nayhgomes@gmail.com, Campus Ji-Paraná } 
		Sônia Carla Gravena Cândido da Silva\footnote{Orientador(a), email, sonia.carla@ifro.edu.br Campus Ji-Paraná}
	\end{center}
	
	\noindent 
	A inclusão de pessoas com deficiências e transtornos globais do desenvolvimento é uma realidade em todas as escolas do país.  Implantada por meios legais como a Política Nacional de Educação Especial na Perspectiva da Inclusão de 2008 que destaca ainda como um movimento mundial pela inclusão, sendo uma ação política, cultural, social e pedagógica, desencadeada em defesa do direito de todos os alunos de estarem juntos, aprendendo e participando, sem nenhum tipo de discriminação. O Instituto Federal de Educação Ciência e Tecnologia de Rondônia, por meio do Núcleo de Atendimento a Pessoa com Necessidades Especiais tem como objetivo principal criar na instituição a cultura da "educação para a convivência", a aceitação da diversidade, a eliminação de barreiras arquitetônicas, educacionais e atitudinais, incluindo socialmente a todos através da educação. Desta forma a educação para diversidade envolve questões que vão além do espaço intraescolar, como o presente projeto que contou com a parceria com o Centro do Autismo como forma de fortalecer as ações desenvolvidas pelo NAPNE Campus Ji-Paraná, trazendo para a escola todas as questões macrossociais que envolvem a inclusão educacional. O objetivo do projeto proposto foi oferecer oportunidades de inclusão social e educacional aos alunos atendidos no Centro de Atendimento Educacional Especializado para Autismo de Ji-Paraná, por meio de oficinas de preparação de material pedagógico e aulas de recreação, sensibilizando docentes, técnicos e alunos do IFRO a convivência com a diversidade. As ações desenvolvidas incluíram palestras com professores, alunos e comunidade externa sobre importantes temas de diversidade, inclusão e educação especial. Oficinas de construção de material pedagógico para utilização no atendimento educacional especializado do Centro de Autismo de Ji-Paraná. As oficinas foram realizadas com os alunos do curso de Licenciatura em Química, e os materiais produzidos foram doados ao centro. A entrega dos materiais foi realizada no centro com a presença dos alunos colaboradores do projeto e participantes do projeto, onde foi possível observar o importante e valoroso trabalho desenvolvido pelo Centro de Autismo, estabelecendo uma parceria com os profissionais especializados em educação especial e inclusão, fortalecendo assim as ações do NAPNE do Campus.
	
	
	\vspace{\onelineskip}
	
	\noindent
	\textbf{Palavras-chave}: Educação. Inclusão. Diversidade
	
\end{document}
