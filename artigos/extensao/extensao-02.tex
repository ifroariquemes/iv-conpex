\documentclass[article,12pt,onesidea,4paper,english,brazil]{abntex2}

\usepackage{lmodern, indentfirst, color, graphicx, microtype, lipsum}			
\usepackage[T1]{fontenc}		
\usepackage[utf8]{inputenc}		

\setlrmarginsandblock{2cm}{2cm}{*}
\setulmarginsandblock{2cm}{2cm}{*}
\checkandfixthelayout

\setlength{\parindent}{1.3cm}
\setlength{\parskip}{0.2cm}

\SingleSpacing

\begin{document}
	
	\selectlanguage{brazil}
	
	\frenchspacing 
	
	\begin{center}
		\LARGE ESCRITÓRIO MODELO DO INSTITUTO FEDERAL\\DE RONDÔNIA -- EMIFRO\footnote{Trabalho realizado dentro da área Ciências Humanas com financiamento do Instituto Federal de Educação, Ciência e Tecnologia de Rondônia.}
		
		\normalsize
		Luiza Souza\footnote{Colaborador(a), email: luiza.ariel.m.s@gmail.com, Campus Porto Velho Calama} 
		Suely Schneider\footnote{(a), email: suelymariaks@gmail.com, Campus Porto Velho Calama} 
		Allan Augusto\footnote{Orientador(a), email: allan.augusto@ifro.edu.br, Campus Porto Velho Calama} 
		Antônio Junior\footnote{Co-orientador(a), email: antonio.junior@ifro.edu.br, Campus Porto Velho Calama} 
	\end{center}
	
	\noindent O Escritório Modelo para Assistência Técnica e Desenvolvimento Tecnológico Sustentável ao Instituto Federal de Rondônia – EMIFRO visa, sobretudo, o desenvolvimento estrutural institucional e contribuir para a integração didática entre alunos e professores envolvidos. O EMIFRO ficou encarregado de realizar projetos para o IFRO readequando suas instalações físicas conforme suas pesquisas aplicadas. Com o propósito de se desenvolver como uma espécie de espaço “empresarial” onde os alunos podem vivenciar o dia a dia de um escritório de engenharia e arquitetura sendo uma experiência de ensino e aprendizagem proporcionando a aproximação dos discentes ao ambiente trabalho. Com o principal objetivo de promover a criação, desenvolvimento e manutenção das instalações do edifício escolar do IFRO Porto Velho Calama, alia-se os conhecimentos adquiridos na instituição através da inovação de sistemas que possam melhorar continuamente seus processos que envolvam a operação predial como sistemas voltados a acessibilidade e sustentabilidade ambiental, focando a economia de recursos energéticos e a integração de pessoas com deficiência ou necessidade especial. Dentre as atividades já realizadas pelo EMIFRO incluem-se a readequação dos espaços físicos e setores da instituição e o planejamento da quadra poliesportiva que não existe ainda no Campus, está em andamento também o concurso interno de decoração e paisagismo que possibilita a integração de todos os alunos do curso Técnico em Edificações com as atividades do Escritório Modelo, na elaboração de propostas de intervenção no ambiente escolar.
	
	\vspace{\onelineskip}
	
	\noindent
	\textbf{Palavras-chave}: Escritório Modelo. Projetos. Acessibilidade.
	
\end{document}
