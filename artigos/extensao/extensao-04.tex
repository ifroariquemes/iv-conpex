\documentclass[article,12pt,onesidea,4paper,english,brazil]{abntex2}

\usepackage{lmodern, indentfirst, color, graphicx, microtype, lipsum}			
\usepackage[T1]{fontenc}		
\usepackage[utf8]{inputenc}		

\setlrmarginsandblock{2cm}{2cm}{*}
\setulmarginsandblock{2cm}{2cm}{*}
\checkandfixthelayout

\setlength{\parindent}{1.3cm}
\setlength{\parskip}{0.2cm}

\SingleSpacing

\begin{document}
	
	\selectlanguage{brazil}
	
	\frenchspacing 
	
	\begin{center}
		\LARGE TREINAMENTO ESPORTIVO NO CAMPUS ARIQUEMES\footnote{Trabalho realizado dentro da área de conhecimento CNPq/CAPES: Ciências da saúde. Projeto de extensão do Edital de Esporte PROEX 40/2016.}
		
		\normalsize
		Amisley G. Araujo\footnote{Coordenador do projeto, amisley.araujo@ifro.edu.br, Campus Ariquemes} 
		Juliano V. Cenci\footnote{Equipe executora (professor), juliano.cenci@ifro.edu.br, Campus Ariquemes} 
		 
	\end{center}
	
	\noindent O projeto Treinamento Esportivo Campus Ariquemes tem por objetivo promover a prática esportiva no âmbito escolar, sustentada nos princípios básicos de totalidade, co-educação, cooperação, participação, emancipação e autonomia, proporcionando ao aluno um desenvolvimento físico, mental e social harmonioso através das atividades físicas propostas, contribuindo para o sua formação humana. A prática esportiva como instrumento educacional, além de contribuir para melhores ganhos de saúde, possibilita também aos seus praticantes uma maior sociabilidade, através do desenvolvimento de ferramentas sociais essenciais para a formação integral do aluno. Mais do que formar atletas, a prática esportiva bem orientada, firmada em princípios éticos, científicos e educacionais, contribui para a formação de cidadãos, promovendo valores inerentes ao ser humano, tais como, respeito, honestidade, cooperação, humildade, lealdade e justiça. A realização dos treinamentos está ocorrendo nas modalidades de futsal, voleibol, basquetebol e handebol, com duração de duas horas cada treino, duas vezes na semana, prevendo a realização de dois amistosos para cada modalidade, com participação de aproximadamente oitenta alunos. Com isso, estamos promovendo a prática recreativa, porém de forma sistematizada com o desenvolvimento do bem estar físico, mental e cultural do aluno, levando-o também a refletir sua prática, reconhecendo seus limites, de seus companheiros e oponentes, com ênfase principalmente no respeito às regras e o jogo Limpo. Além do exposto, o treinamento esportivo tem contribuído para a preparação dos alunos interessados em participar de eventos esportivos, oportunizando a estes a possibilidade de vivenciar a cultura de outros lugares da região e do país e o intercâmbio social através das diferentes competições escolares realizadas no decorrer do ano de 2016, dentre elas, os Jogos dos Institutos Federais de Rondônia (JIFRO). Ao final pretende-se o fortalecimento da prática esportiva no Campus Ariquemes, bem como no município, através da realização de Torneios do IFRO, nas modalidades de futsal, handebol, basquetebol e voleibol.
	
	\vspace{\onelineskip}
	
	\noindent
	\textbf{Palavras-chave}: Treinamento esportivo. Formação humana. Preparação esportiva.
	
\end{document}
