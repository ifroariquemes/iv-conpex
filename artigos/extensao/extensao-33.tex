\documentclass[article,12pt,onesidea,4paper,english,brazil]{abntex2}

\usepackage{lmodern, indentfirst, color, graphicx, microtype, lipsum}			
\usepackage[T1]{fontenc}		
\usepackage[utf8]{inputenc}		

\setlrmarginsandblock{2cm}{2cm}{*}
\setulmarginsandblock{2cm}{2cm}{*}
\checkandfixthelayout

\setlength{\parindent}{1.3cm}
\setlength{\parskip}{0.2cm}

\SingleSpacing

\begin{document}
	
	\selectlanguage{brazil}
	
	\frenchspacing 
	
	\begin{center}
		\LARGE VIVER SUSTENTÁVEL\footnote{Trabalho realizado dentro da (área de Conhecimento CNPq/CAPES: Engenharia sanitária) com financiamento de recursos próprios do Instituto Federal de Rondônia.}
		
		\normalsize
		Lorhena Vilela da Luz\footnote{Bolsista (modalidade), rebeca523ro@gmail.com, Campus Cacoal} 
		Rebeca Pereira de Figueiredo Santiago\footnote{(a),gizele.viana@ifro.edu.br, Campus Cacoal} 
		Sabrina Vales Vieira da Silva\footnote{Orientador(a), lucy.santos@ifro.edu.br, Campus Cacoal} 
		Tatiele Flores Santiago\footnote{Co-orientador(a), fernanda.cavalcante@ifro.edu.br, Campus Cacoal.}
		Talyson Rodrigo Felipe de Limas 
	\end{center}
	
	\noindent O presente trabalho de projeto de extensão tem como objetivo reaproveitar os resíduos gerados pelo IFRO Campus Cacoal em parceria com a cooperativa local, foi realizada um levantamento bibliográfico das cooperativas de catadores e sua importância no processo de reaproveitamento dos resíduos. A integração da cooperativa ao Campus Cacoal é indispensável, pois contribui para o aumento na renda familiar dos cooperativados e dá o destino correto aos resíduos.   Alunos e servidores realizarão a separação diária dos materiais facilitando o trabalho da cooperativa de modo, a colaborar na sustentabilidade no Campus. Mediante pesquisas realizadas, nota-se que a profissão de catadores não é devidamente reconhecida quanto ao papel social e ambiental que exercem. Geralmente o que ocorre e que a maioria deles perâmbula em média 30 quilômetros por dia, debaixo de chuva e sol, puxando até 400 quilos de materiais, que muitas vezes, só são encontrados dentro de sacos de lixo. A coleta seletiva e a separação correta dos resíduos proporcionam ganhos não só aos catadores mais também a sociedade que é beneficiada, pois com o reuso desses materiais há uma economia considerável de recursos empregados com a finalidade de fabricar novos produtos, além disso, diminuem-se os danos ambientais. A reciclagem está se tornando indispensável nos dias atuais, pois transforma aquilo que iria ou já se encontra no lixo em novos produtos, diminuindo os resíduos que seriam lançados nos lixões, que acabam atraindo animais que vão à busca de restos de alimentos ali depositados. Partindo dos contextos acima citados, nota-se que o consumo está cada vez mais acessível, por esse motivo se faz necessário repensar sobre a correta destinação dos resíduos, fazendo a coleta seletiva, de modo que facilite a coleta realizada pelas cooperativas.
	
	\vspace{\onelineskip}
	
	\noindent
	\textbf{Palavras-chave}: Sustentabilidade. Reciclagem. Cooperativas.
	
\end{document}
