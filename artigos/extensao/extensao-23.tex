\documentclass[article,12pt,onesidea,4paper,english,brazil]{abntex2}

\usepackage{lmodern, indentfirst, color, graphicx, microtype, lipsum}			
\usepackage[T1]{fontenc}		
\usepackage[utf8]{inputenc}		

\setlrmarginsandblock{2cm}{2cm}{*}
\setulmarginsandblock{2cm}{2cm}{*}
\checkandfixthelayout

\setlength{\parindent}{1.3cm}
\setlength{\parskip}{0.2cm}

\SingleSpacing

\begin{document}
	
	\selectlanguage{brazil}
	
	\frenchspacing 
	
	\begin{center}
		\LARGE EXPOSIÇÃO ITINERANTE: “O PAPEL DA IMPRENSA NA FORMAÇÃO DE RONDÔNIA: DO ELDORADO A COLONIZAÇÃO NA SELVA (1981-1985)\footnote{Trabalho realizado dentro da (área de Conhecimento CNPq/CAPES: Ciências Humanas – História)}
		
		\normalsize
		Jucielly Espíndola de Almeida\footnote{Bolsista (modalidade), juciellyalmeida72@gmail.com, Ji-Paraná} 
		Waldelaine Rodrigues Hoffmann\footnote{(a), waldelaine-hoffmann@hotmail.com, Ji-Paraná} 
		Lourival Inácio Filho\footnote{Orientador(a), lourival.filho@ifro.edu.br,  Ji-Paraná} 
		Juliano Fischer Naves\footnote{Co-orientador(a), juliano.naves@ifro.edu.br, Ji-Paraná} 
	\end{center}
	
	\noindent A exposição itinerante “O Papel da Imprensa na Formação de Rondônia: do Eldorado a colonização na selva (1981-1985)” foi um trabalho histórico-cultural vinculado ao Núcleo Informatizado de memória e Pesquisa do IFRO (NIMPI). Trabalhou de forma articulada com os três eixos em que se alicerçam a prática docente dos Institutos Federais que são Ensino, Pesquisa e Extensão. Fruto de um trabalho contínuo arquivístico de levantamento de fontes documentais jornalísticas e criação de banco de dados sobre o desenvolvimento histórico de Rondônia, oficialmente, o projeto ocorreu do primeiro semestre de 2015 ao primeiro semestre de 2016. Utilizou-se os conceitos de hegemonia na busca de entendimento das representações da notícia frente à criação do estado de Rondônia que, de forma ambígua e arbitrária, era inserida em uma espécie de jogo de consensos e disputas hegemônicas. O recorte temporal se caracteriza como um período de múltiplos tempos e a reconstrução de espaços físicos e imaginários sobre Rondônia, provocados por um intenso processo migratório sobre a Terra da Providência e o Novo Eldorado. De abril a junho de 2015 foram desenvolvidas as pesquisas, levantamento de fontes, organização dos Layouts dos painéis temáticos, serviços de marcenaria e arte gráfica, confecção de folders e cartazes, levantamento e agendamentos junto aos locais que receberam a exposição. A Exposição esteve nas seguintes cidades: Ouro Preto do Oeste (24 a 26/06/2015); Ji-Paraná (05 a 07/07/2015); Porto Velho (04 a 29/03/2016); Ariquemes (25/04/2016 a 20/05/2016). Atingiu um público de milhares de pessoas, bem como obteve ampla divulgação pelas mídias impressas, digitais e televisivas. Contribuiu de forma consistente enquanto subsídio para uma reflexão histórica e educacional voltada para a formação de Rondônia.
	
	\vspace{\onelineskip}
	
	\noindent
	\textbf{Palavras-chave}: Exposição. Imprensa. Historia de Rondônia.
	
\end{document}
