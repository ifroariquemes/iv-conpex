\documentclass[article,12pt,onesidea,4paper,english,brazil]{abntex2}

\usepackage{lmodern, indentfirst, color, graphicx, microtype, lipsum}			
\usepackage[T1]{fontenc}		
\usepackage[utf8]{inputenc}		

\setlrmarginsandblock{2cm}{2cm}{*}
\setulmarginsandblock{2cm}{2cm}{*}
\checkandfixthelayout

\setlength{\parindent}{1.3cm}
\setlength{\parskip}{0.2cm}

\SingleSpacing

\begin{document}
	
	\selectlanguage{brazil}
	
	\frenchspacing 
	
	\begin{center}
		\LARGE A EXTENSÃO COMO MECANISMO DE \\DIVULGAÇÃO DOS CURSOS \footnote{Trabalho realizado dentro da área de Conhecimento CNPq/CAPES: Multidisciplinar, com financiamento do Departamento de Extensão do Campus Porto Velho Zona Norte}
		
		\normalsize
		Rodrigo Lopes da Silva\footnote{Bolsista de extensão, email rodrigolopes.88@hotmail.com, Campus Porto Velho Zona Norte.} 
		Lidiane Cristina Jucá Gadêlha Jardim\footnote{Colaboradora, email  lidiane.jardim@ifro.edu.br, Campus Porto Velho Zona Norte.} 
		Deivid da Silva Barros\footnote{Colaborador, email  deivid.barros@ifro.edu.br, Campus Porto Velho Calama.} \\
		Thiago Pacife de Lima\footnote{Orientador, email thiago.lima@ifro.edu.br, Campus Porto Velho Calama.} 
		Ilma Paula Carvalho da Silva\footnote{Co-orientadora, email ilma.silva@ifro.edu.br, Campus Porto Velho Zona Norte.}
	\end{center}
	
	\noindent Este trabalho apresenta os resultados obtidos a partir da realização do Projeto de Extensão “Conhecendo o IFRO Campus Porto Velho Zona Norte”, cujo objetivo foi contribuir para a divulgação da instituição e dos cursos ofertados no referido Campus. Através do projeto, foi realizada a visitação ao Campus por 270 estudantes do ensino médio, sendo em média 45 alunos por visita. Participaram do projeto as seguintes escolas: E.E.E.F.M. Bela Vista, E.E.E.F.M. Marcelo Cândia, E.E.E.F.M.  Professor Eduardo Lima e Silva, E.E.E.F.M.  Castelo Branco e E.E.E.F.M.  Professor Flora Calheiros Cotrin. Trata de uma pesquisa descritiva e aplicada sob uma abordagem quantitativa. A pesquisa foi realizada na cidade de Porto velho, a partir de entrevistas realizadas com 304 inscritos no Processo seletivo 2016/1 para os cursos concomitantes, além dos 270 estudantes que participaram do projeto. Os estudantes colaboradores da CAED entraram em contato com 304 dos 498 inscritos no processo seletivo e, considerando as informações coletadas, foi possível constatar que dentre os entrevistados, 25\% dos visitantes se inscreveram no processo seletivo e que 29\% dos inscritos foram oriundos das escolas participantes, além disso, 18\% dos entrevistados revelaram que souberam do processo seletivo a partir da divulgação nas escolas. Considerando que o Campus Porto Velho Zona Norte iniciou sua atuação há menos de 5 anos, ações que promovam a divulgação da instituição e dos cursos oferecidos certamente contribuirão para ampliar sua visibilidade e o interesse da população em estudar no IFRO, além de proporcionar ao futuro candidato a possibilidade de conhecer o curso para o qual está se inscrevendo, contribuindo para o êxito no processo educativo. A partir do projeto verifica-se que a extensão é um forte mecanismo para conduzir essa integração, contribuindo também para redução da taxa de evasão.
	
	\vspace{\onelineskip}
	
	\noindent
	\textbf{Palavras-chave}: Processo seletivo. Cursos técnicos. Visitação.
	
\end{document}
