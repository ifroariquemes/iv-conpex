\documentclass[article,12pt,onesidea,4paper,english,brazil]{abntex2}

\usepackage{lmodern, indentfirst, color, graphicx, microtype, lipsum}			
\usepackage[T1]{fontenc}		
\usepackage[utf8]{inputenc}		

\setlrmarginsandblock{2cm}{2cm}{*}
\setulmarginsandblock{2cm}{2cm}{*}
\checkandfixthelayout

\setlength{\parindent}{1.3cm}
\setlength{\parskip}{0.2cm}

\SingleSpacing

\begin{document}
	
	\selectlanguage{brazil}
	
	\frenchspacing 
	
	\begin{center}
		\LARGE CONVERSA AFIADA: RODA DE LEITORES\footnote{Trabalho realizado dentro da (área de Conhecimento CNPq/CAPES: Educação) com financiamento do IFRO Campus Vilhena.}
		
		\normalsize
		Larissa Elisa dos Santos Oliveira \footnote{Bolsista (PIPEEX), larissaelisa2306@gmail.com, Campus Vilhena.} 
		Elza Moreira Alves \footnote{Orientadora, elza.moreira@ifro.edu.br, Campus Vilhena.} 
	
	\end{center}
	
	\noindent Na prática docente, há uma ânsia de se formar leitores e, por conseguinte, pensadores. Nesse sentido, o projeto “Conversa afiada: roda de Leitores” refere-se a uma prática pedagógica cujo objetivo é incentivar a leitura, bem como formar leitores, conscientes e críticos, tendo como base obras clássicas da literatura brasileira. Acredita-se na relevância da atividade devido à modalidade de ensino oferecida pelo Instituto Federal de Rondônia, pois ao oferecer cursos técnicos integrados ao ensino médio, supõe-se que haja a necessidade de estudar, conhecer, divulgar os clássicos da literatura brasileira. Nessa perspectiva, ao aplicar uma metodologia discussão que seja dinâmica e atrativa acredita-se que é uma maneira de auxiliar o aluno a compreender e valorizar a riqueza e a importância da Literatura brasileira. Além disso, deve-se considerar que as obras literárias são requisitadas em Exames Nacionais e Vestibulares para ingresso em cursos de nível superior. Para a realização das oficinas de discussão, são utilizados recursos e materiais como: livros físicos e virtuais, cadernos, lápis, caneta, borracha, dispositivos de armazenamento de dados. As oficinas são realizadas em espaços internos e externos do Campus. Até o presente momento pode-se afirmar que as rodas de conversa estão trazendo resultados satisfatórios aos participantes, uma vez que estes interagem com as temáticas discutidas, expressando seus pontos de vistas acerca do momento histórico da narrativa, bem como enfatizando o estereótipo das personagens, fazendo paralelos com a contemporaneidade; questionando aos colaborados aspectos que não lhes eram claros. Isso demonstra 	que os participantes têm consciência da importância da leitura dos clássicos brasileiros e, por conseguinte, da realização da prática pedagógica ofertada.
	
	\vspace{\onelineskip}
	
	\noindent
	\textbf{Palavras-chave}: Literatura. Ensino Médio. Leitura.
	
\end{document}
