\documentclass[article,12pt,onesidea,4paper,english,brazil]{abntex2}

\usepackage{lmodern, indentfirst, color, graphicx, microtype, lipsum}			
\usepackage[T1]{fontenc}		
\usepackage[utf8]{inputenc}		

\setlrmarginsandblock{2cm}{2cm}{*}
\setulmarginsandblock{2cm}{2cm}{*}
\checkandfixthelayout

\setlength{\parindent}{1.3cm}
\setlength{\parskip}{0.2cm}

\SingleSpacing

\begin{document}
	
	\selectlanguage{brazil}
	
	\frenchspacing 
	
	\begin{center}
		\LARGE EDUCAÇÃO PROFISSIONAL INTERCULTURAL: EXPERIÊNCIAS PEDAGÓGICAS DE ENSINO TÉCNICO COM  IMIGRANTES\footnote{Trabalho desenvolvido na área de Conhecimento CNPq/CAPES: Ciências Humanas – Educação, contado com financiamento parcial do Edital N.17, de 15 de setembro de 2014/IFRO/Campus Zona Norte.}
		
		\normalsize
		Rosa Martins Costa Pereira\footnote{Autora e Coordenadora Geral do Projeto, E-mail: rosa.martins@ifro.edu.br, PRODIN/Reitoria.} 
	    Miralba Uchoa de Carvalho\footnote{Professora do Projeto (aulas teóricas), E-mail: miralba@ifro.edu.br, DEINF/Reitoria.} \\
		Adelson Barbosa de Souza\footnote{Professor do Projeto (aulas práticas), E-mail: adelson.barboza@ifro.edu.br, DEINF/Reitoria.} 
		Reuria da Silva Moreira\footnote{Bolsista do Projeto, E-mail: reuria.moreira@ifro.edu.br, Estudante, Campus Porto Velho Zona Norte. } 
	\end{center}
	
	\noindent Esse trabalho apresenta resultados do Projeto “Pintando a esperança: curso de pintor profissional com haitianos” que teve como escopo capacitar profissionalmente um grupo de imigrantes haitianos em situação de vulnerabilidade social. A delimitação aqui proposta se constitui em um relato de experiências pedagógicas de ensino técnico no contexto intercultural, com atividades teóricas e práticas que foram realizadas a fim de que os imigrantes buscassem sua própria autonomia profissional, não apenas técnicas de trabalho, mas também nas relações de trabalho e na conquista da cidadania no Brasil. A metodologia do curso foi organizada em três momentos: a) realização de curso teórico, com utilização de data show, figuras, texturas, tintas, instrumentos e outros materiais necessários para a compreensão dos fundamentos e das vivências profissionais na área de pintura na construção civil; b) realização de curso prático para a aplicação dos fundamentos estudados e para a correção de possíveis dificuldades enfrentadas pelos alunos e c) inauguração da obra para apresentar os resultados do projeto. O local foi restaurado (a igreja da comunidade) e fotografado antes e depois das obras realizadas pelos próprios imigrantes (alunos do curso). Foram organizadas apostilas para que o aluno-imigrante pudesse ter um material de consulta sempre que precisar. Um dos maiores desafios pedagógicos foi a comunicação português-crioulo, enfrentado com a presença de tradutor. A equipe utilizou diferentes estratégias de ensino, como uso de imagens, com apresentação de slides, dinâmicas em grupos, além das aulas práticas na infraestrutura local.  Os resultados obtidos provenientes deste projeto foram: a) certificação de 49 alunos e alunas, quase o dobro da meta inicial definida no projeto; b) certificação de alunos que já trabalham como pintor, mas que nunca tiveram certificação nem no Haiti e nem no Brasil; c) inserção de alunos do curso no mercado de trabalho; d) aprendizagem de nova profissão para alunos que não eram pintores e e) socialização das técnicas e procedimentos tecnológicos para a pintura profissional na sociedade brasileira.
	
	\vspace{\onelineskip}
	
	\noindent
	\textbf{Palavras-chave}: Educação Profissional. Cultura. Migração Internacional.
	
\end{document}
