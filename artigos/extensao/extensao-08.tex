\documentclass[article,12pt,onesidea,4paper,english,brazil]{abntex2}

\usepackage{lmodern, indentfirst, color, graphicx, microtype, lipsum}			
\usepackage[T1]{fontenc}		
\usepackage[utf8]{inputenc}		

\setlrmarginsandblock{2cm}{2cm}{*}
\setulmarginsandblock{2cm}{2cm}{*}
\checkandfixthelayout

\setlength{\parindent}{1.3cm}
\setlength{\parskip}{0.2cm}

\SingleSpacing

\begin{document}
	
	\selectlanguage{brazil}
	
	\frenchspacing 
	
	\begin{center}
		\LARGE LEITURA E ARTE CONTRIBUINDO PARA A FORMAÇÃO DE MENORES EM \\SITUAÇÃO DE RISCO\footnote{Trabalho realizado dentro da área de Línguística, Letras e Artes com financiamento da (PROEX-IFRO/2015).}
		
		\normalsize
		Pedro Gabriel Silva e Silva \footnote{de Extensão, pedrobieljip@gmail.com, Campus Ji-Paraná.} 
		Renata Fachiano Mazali \footnote{Bolsista de Extensão, renata.fachiano@gmail.com, Campus Ji-Paraná.} 
		Cleuza Diogo Antunes\footnote{Orientador(a), cleuzadiogo@yahoo.com.br, Campus Ji-Paraná.} 
 
	\end{center}
	
	\noindent A leitura e a arte muito podem contribuir para a formação cidadã dos indivíduos, conferindo-lhes autonomia e contribuindo com sua autoestima por meio do desenvolvimento de novas habilidades. Visando contribuir para a cidadania, cultura, intelectualidade e artístico de crianças vulneráveis do Abrigo Municipal de Ji-Paraná/RO, o projeto Leitura Cidadã foi desenvolvido em 2015 por uma equipe multiprofissional do IFRO e UNIR com crianças que estavam previamente inseridas em ambientes de violência, portanto seu comportamento refletia tal realidade. Reconstruindo o comportamento cidadão individual de cada um e auxiliando no crescimento intelectual para a reinserção posterior à sociedade e à família, as atividades do projeto foram guiadas por esses princípios. Foram desenvolvidas em visitas semanais, múltiplas atividades relacionadas com a leitura, tais como: contação de histórias, dramatizações, brincadeiras, atividades manuais, exposição de vídeos, produção textual e pintura em tela. As crianças escreveram alguns textos durante as atividades que serviram para análise conforme o projeto foi se desenvolvendo. Por meio de um questionário avaliativo, foi possível observar que o incentivo dado pelas atividades contribuiu para o interesse maior das crianças pela leitura, as oficinas de artes serviram para despertar o lado artístico-criativo durante a recriação de obras de artistas brasileiros. A aprovação vista nas respostas das crianças ao questionário avaliativo demonstra que as atividades contribuíram para resgatar a cidadania e promover a autonomia dos menores envolvidos. Todas as metas propostas foram alcançadas e o impacto sentido foi à participação voluntária dos menores nas atividades propostas e a elaboração de uma coletânea de textos onde constou pelo menos um texto de cada menor participante. De acordo com o depoimento dos menores no momento da avaliação o projeto contribuiu para tornar a permanência mais agradável no abrigo e para estimulá-los a continuarem estudando e se preparando para o futuro. A realização de atividades como a pintura da tela e a coletânea de textos resgatou a autoestima de muitos que não se consideravam capazes e ao final sentiram-se orgulhosos com os resultados.
	
	\vspace{\onelineskip}
	
	\noindent
	\textbf{Palavras-chave}: Leitura. Cidadania. Arte. Cultura.
	
\end{document}
