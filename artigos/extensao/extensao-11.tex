\documentclass[article,12pt,onesidea,4paper,english,brazil]{abntex2}

\usepackage{lmodern, indentfirst, color, graphicx, microtype, lipsum}			
\usepackage[T1]{fontenc}		
\usepackage[utf8]{inputenc}		

\setlrmarginsandblock{2cm}{2cm}{*}
\setulmarginsandblock{2cm}{2cm}{*}
\checkandfixthelayout

\setlength{\parindent}{1.3cm}
\setlength{\parskip}{0.2cm}

\SingleSpacing

\begin{document}
	
	\selectlanguage{brazil}
	
	\frenchspacing 
	
	\begin{center}
		\LARGE A PRÁTICA PEDAGÓGICA DE ESPORTES COLETIVOS\footnote{Trabalho realizado dentro da área de Conhecimento CNPq/CAPES: Ciências da Saúde com financiamento do IFRO.}
		
		\normalsize
		Nathielly Christina de Fátima\footnote{Colaborador(a), nathiellychristina@gmail.com, Campus Ji-Paraná} 
		Fabrício Gurkewicz Ferreira\footnote{Orientador(a), fabricio.gurkewicz@ifro.edu.br, Campus Ji-Paraná} 
		Nilza Maria Pereira\footnote{Co-orientador(a), nilza.pereira@ifro.edu.br, Campus Ji-Paraná} 
		Edivan Carlos da Cunha\footnote{Co-orientador(a), edivan.carlos@ifro.edu.br, Campus Ji-Paraná.} 
	\end{center}
	
	\noindent O esporte é um dos maiores fenômenos presente na sociedade atualmente. Desde o inicio de sua formatação como instituição, no séc. XIX na Inglaterra, ele permanece em contínua evolução (PAES 1996). Por suas múltiplas possibilidades, os seus desdobramentos são observados em diferentes setores sociais tais como politica, economia, cultura e educação. A sua apropriação neste ultimo contexto, pode proporcionar aos seus participantes muito além do domínio de gestos motores. A diversidade de situações vivenciadas na prática esportiva pode contribuir de forma significativa na formação integral dos alunos. Nesse sentido, esse projeto tem por objetivo, uma vez que se encontra em desenvolvimento, proporcionar o acesso à prática sistematizada de esportes coletivos possibilitando a construção do conhecimento relacionado aos aspectos técnicos, táticos, físicos, cognitivos, afetivos e sociais das modalidades, contribuindo para a formação integral dos participantes, tanto direcionada para a parte esportiva quanto para o lazer e a cidadania. O projeto conta com a participação de setenta alunos distribuídos através de três modalidades (Flag Football, Futsal e Voleibol). Os encontros acontecem duas vezes por semana nas dependências do Campus. Para a verificação do alcance das metas pretendidas são realizadas observações e registros em fichas com posteriores encontros individuais e coletivos com os participantes para problematizar as dificuldades e buscar caminhos para a sua solução. Diante das experiências vivenciadas até o momento, podemos certificar que, não obstante as dificuldades referentes ao saber fazer das modalidades e a assiduidade, boa parte dos alunos ampliou a compreensão em relação às práticas esportivas, visualizando-as não apenas como um fim em si mesmas, mas como um espaço no qual é possível construir relações interpessoais salutares e edificar valores que beneficiem tanto o individual quanto o coletivo.
	
	\vspace{\onelineskip}
	
	\noindent
	\textbf{Palavras-chave}: Projeto. Esporte. Esportes Coletivos.
	
\end{document}
