\documentclass[article,12pt,onesidea,4paper,english,brazil]{abntex2}

\usepackage{lmodern, indentfirst, nomencl, color, graphicx, microtype, lipsum,textcomp}			
\usepackage[T1]{fontenc}		
\usepackage[utf8]{inputenc}		

\setlrmarginsandblock{2cm}{2cm}{*}
\setulmarginsandblock{2cm}{2cm}{*}
\checkandfixthelayout

\setlength{\parindent}{1.3cm}
\setlength{\parskip}{0.2cm}

\SingleSpacing

\begin{document}
	
	\selectlanguage{brazil}
	
	\frenchspacing 
	
	\begin{center}
		\LARGE EFEITO DE EXTRATOS DE PLANTAS MEDICINAIS DO ESTADO DE RONDÔNIA SOBRE A GERMINAÇÃO DE SEMENTES\footnote{Trabalho realizado dentro da (área de Conhecimento CNPq/CAPES: Ciências Exatas e da Terra com financiamento do IFRO – Campus Vilhena}
		
		\normalsize
		Autor 1Leonardo Colombo Paniagua\footnote{Bolsista (PIP), leocol@gmail.com.br, Campus Vilhena.} 
		Aline Santiago\footnote{Colaborador(a), aline.santiago@ifro.edu.br, Campus.Vilhena} 
		Camila Ferreira Abrão\footnote{Orientador(a), camila.abrao@ifro.edu.br, Campus.Vilhena} 
		 Tatiane de Abreu Curado Rezende\footnote{Co-orientador(a), tatiane.rezende@ifro.edu.br, Campus.Vilhena} 
	\end{center}
	
	% resumo em português
	\begin{resumoumacoluna}
		Algumas espécies do gênero Piper podem apresentar efeito fungicida, inseticida ou mesmo alelopático sobre a germinação das sementes de outras espécies. Diante disso, o trabalho teve como objetivo avaliar o efeito do extrato alcóolico das folhas da espécie Piper aduncum sobre a germinação de sementes de alface. O experimento foi conduzido com 4 repetições de 50 sementes semeadas em placas de petri forradas com duas folhas de papel de filtro umedecidas com extrato alcoólico da folha da Piper aduncum, nas concentrações 0, 10, 20, 30\%, as sementes foram mantidas a temperatura de 25 ºC com fotoperíodo de 24 horas por sete dias. Concluiu-se que o aumento das concentrações do extrato reduz a porcentagem de germinação das sementes de alface, mostrando assim seu efeito inibitório.
		
		\vspace{\onelineskip}
		
		\noindent
		\textbf{Palavras-chave}: Alelopatia. Concentração. Piper.
		
	\end{resumoumacoluna}
	
	\section*{Introdução}
	
	Este projeto trata da realização do estudo do efeito do extrato de plantas do estado de Rondônia sobre a germinação de sementes de alface. Esse efeito sobre a germinação de sementes pode ser avaliado pela atividade alelopática, capacidade que determinados compostos fitoquímicos possuem de interferir na germinação ou no desenvolvimento de plantas (SOARES, 2000).
	Estudos mostram que algumas espécies de plantas possuem efeito inseticida, larvicida, antibacteriano e antifúngico, mostrando assim sua grande utilidade na agricultura (SILVA \& BASTOS,2007; SOUZA et al., 2008; FIQUEIRA et. all, 2011).
	Dentre as plantas estudadas, as do gênero Piper demonstraram possuir características alelopáticas sobre outras plantas (ALMEIDA et al, 2011; LUSTOSA et al., 2007). A espécie Piper aduncum apresenta várias classes de substâncias (flavonóides, terpenos e esteróides) que já foram isoladas (SILVA, 2008). A busca por substâncias de efeito alelopático a partir de extratos de plantas por meio de bioensaios em laboratório tem se mostrado muito eficiente, pois segundo Coelho et al. (2011) em condições de laboratório se pode controlar parâmetros de difícil controle no campo ou em ambiente natural.
	Desta forma, o objetivo deste trabalho é verificar o efeito alelopático do extrato das folhas de Piper aduncum, em diferentes concentrações, na germinação da alface.
	
	\section*{Material e Método}
	
	O trabalho foi realizado no Laboratório de Química do Instituto Federal de Rondônia – Campus Vilhena. Para o experimento foi utilizada sementes de alface adquiridas no comércio local. As sementes foram colocadas para germinar em câmara BOD a temperatura de 25ºC, com fotoperíodo de 24horas, sendo semeadas em placas de petri, sobre duas folhas de papel de filtro, umedecidas com 2 ml de extratos alcóolicos da folha da \textit{Piper Aduncum}, em diferentes concentrações 0, 10, 20, 30\%. Para cada tratamento, foi utilizado 200 sementes divididas em quatro repetições de 50 sementes.
	As avaliações foram diárias, iniciando-se no primeiro dia após a semeadura e finalizando no sétimo dia (BRASIL, 2009). O critério para a germinação foi a emissão da radícula.Texto MM.
	
	\section*{Resultados e Discussão}
	
	De acordo com os dados obtidos, observou-se que a germinação decresce com aumento das concentrações do extrato e este fato é bem mais evidente nas concentrações mais elevadas em que não foi registrado valores de germinação, concentração em 30\%. Resultados semelhantes foram encontrados por Lustosa et al. (2007), em sementes de alface onde foram testadas diferentes concentrações de extrato aquoso de \textit{Piper aduncum} e \textit{Piper tectoniifolium}, observando-se que a Piper aduncum apresentou maior inibição da germinação quando comparada com a \textit{Piper tectoniifolium}, sendo que a \textit{P. aduncum} mantém maior taxa de inibição a partir dos 3\%. 
	O extrato de \textit{P. Aduncum} influenciou negativamente a taxa de germinação, tendo efeito inibitório em altas concentrações de extrato, sem formação de plântulas. Nota-se que ocorre formação de plântulas até a concentração de 20\%. Sendo que o maior número de plântulas anormais se encontra nessa concentração. 
	Foi observado no experimento que nas concentrações em que não há formação de plântulas, houve atrofiamento das raízes e os hipocótilos ficaram envolvidos no tegumento da semente. O efeito alelopático dessa família também pode ser observado em outras espécies. Silva et al. (2013) comprovaram o efeito alelopático de \textit{Piper hispidinervium} no desenvolvimento inicial de milho. Efeitos negativos no crescimento de plântulas também foram encontrados por Lustosa et al. (2007) ao utilizar o extrato de \textit{Piper aduncum} em sementes de alface, para ele esses efeitos são mais drásticos no crescimento inicial das plântulas, causando acentuada redução do crescimento. 
	Além do efeito alelopático, notou-se a ação antifúngico do extrato, e que com o aumento da concentração a ação ficou bem evidente. Sendo necessário testes específicos para a confirmação do resultado. A atividade inseticida das piperáceas foi relatada nos trabalhos de Silva et al. (2007) onde observou-se que tanto o extrato aquoso de raízes como o de folhas de \textit{P. aduncum}, apresentaram atividade inseticida sobre adultos de \textit{Aetalion sp.}, praga de importância econômica na Amazônia; e de Souto et al. (2011) onde foram realizados estudos preliminares da atividade inseticida de óleos essenciais de espécies de Piper linneus em operárias de Solenopis saevissima f Smith, em laboratório.
	
	
	\section*{Conclusões}
	
	O extrato da \textit{P. Aduncum} sobre as sementes de alface foi prejudicial a germinação e desenvolvimento inicial das mesmas, mostrando um efeito alelopático inibitório. Mesmo sendo necessário teste específicos para demonstrar a ação antifúngica, esse fato ficou evidente com o aumento da concentração dos extratos.
	
	\section*{Instituição de Fomento}
	
	Instituto Federal de Rondônia – Campus Vilhena.
	
	\section*{Referências}
	
	\noindent ALMEIDA, F. A. C. et al. Infestação e germinação em sementes de milho tratadas com extratos de \textit{piper nigrum e annona squamosa.} \textbf{Revista Brasileira de Produtos Agroindustriais,} Campina Grande, v.13, n. Especial, p. 411-426, 2011.
	
	\noindent BRASIL, Ministério da Agricultura, Pecuária e Abastecimento. \textbf{Regras para análise de sementes}. Brasília: Mapa/ACS, 2009. 399 p.
	
	\noindent COELHO, M. F. B. et al., Atividade alelopática de extrato de sementes de juazeiro. \textbf{Revista Horticultura Brasileira}, Vitoria da Conquista, v. 29, n. 1, p. 108-111, 2011. FIGUEIRA, G, M.; DUARTE, M. C. T. SILVA,; C. A. L.; DELARMELINA C. Atividade antimicrobiana do extrato e do óleo essencial de \textit{Piper} spp cultivadas na coleção de germoplasmas do CPQBA-Unicamp. \textbf{Revista Brasileira de Farmacognosia}. 2011, v.14, n. 1, p.51-53.
	
	\noindent LUSTOSA, F. L. F. et al. Efeito alelopático de extrato aquoso de \textit{Piper aduncum} L. e \textit{Piper tectoniifolium kunth} na Germinação e crescimento de Lactuca sativa L. \textbf{Revista Brasileira de Biociências}, Porto Alegre, v. 5, supl. 2, p. 849-851, 2007.
	
	\noindent SILVA, J. E. N. et al. Efeito alelopático de Piper hispidinervium sobre desenvolvimento inicial de milho (Zea mays). \textbf{Revista Agrarian}, Dourados, v.6, n.20, p.148-153, 2013.
	
	\noindent SILVA, D. M. M. H.; BASTOS, C. N.. Atividade antifúngica de óleos essenciais de espécies de \textit{Piper} sobre \textit{Crinipellis} perniciosa, \textit{Phytophthora palmivora} e \textit{Phytophthora capsici}. \textbf{Fitopatol. Bras}. 2007, vol.32, n.2, p. 143-145. ISSN 0100- 4158.
	
	\noindent SILVA, W. C. et al. Atividade inseticida de Piper aduncum L. (Piperaceae) sobre Aetalion sp. (Hemiptera: Aetalionidae), praga de importância econômica no Amazonas. \textbf{Acta Amazonica}, Manaus, v.37, n. 2, p. 293-298, 2007.
	
	\noindent SOUTO, R. N. P. et al. Estudos preliminares da atividade inseticida de óleos essenciais de espécies de Piper linneus (piperaceae) em operárias de Solenopis saevissima f Smith (Hymenoptera: formicidae), em laboratório. \textbf{Biota Amazônia}, Macapá, v. 1, n. 1, p. 42-48, 2011.
	
	\noindent SOARES, G. L. G., Inibição da germinação e do crescimento radicular de alface (cv. Grands Rapids) por extratos aquosos de cinco espécies de Gleicheniaceae. \textbf{Floresta e Ambiente.} 7: 190-197, 2000.
	
	\noindent SOUSA, P. J. C.; BARROS, C. A. L.; ROCHA, J. C. S.; LIRA, D. S.; MONTEIRO, G.
	M.; MAIA, J. G. S. Avaliação toxicológica do óleo essencial de Piper aduncum L.
	\textbf{Revista Brasileira de farmacognosia.} 2008, v. 18, n. 2, p. 217-221.
	
\end{document}