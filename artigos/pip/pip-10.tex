
\documentclass[article,12pt,onesidea,4paper,english,brazil]{abntex2}

\usepackage{lmodern, indentfirst, nomencl, color, graphicx, microtype, lipsum}			
\usepackage[T1]{fontenc}		
\usepackage[utf8]{inputenc}		

\setlrmarginsandblock{2cm}{2cm}{*}
\setulmarginsandblock{2cm}{2cm}{*}
\checkandfixthelayout

\setlength{\parindent}{1.3cm}
\setlength{\parskip}{0.2cm}

\SingleSpacing

\begin{document}
	
	\selectlanguage{brazil}
	
	\frenchspacing 
	
	\begin{center}
		\LARGE DESCENDENTES DA LÁSTIMA:
		GRUPOS NEONAZISTAS NA SOCIEDADE CONTEMPORÂNEA.\footnote{ 7.05.03 -História / História Moderna e Contemporânea}
		
		\normalsize
	Karine Medeiros Anselmo1\footnote{Karine Medeiros Anselmo, karinemedeiros1999@gmail.com, Campus Cacoal.} 
	Herisson Galescky Torres2\footnote{Herisson Galescky torres, Herisson\_eoe@hotmail.com, Campus Cacoal.} 
	Sérgio Nunes de jesus3\footnote{Sérgio Nunes de jesus,sérgio30canibal@gmail.com Campus Cacoal.} 
	\end{center}
	
	% resumo em português
	\begin{resumoumacoluna}
		Objetivou-se, por meio deste trabalho, pesquisar a influência da ideologia neonazista praticada pelos nacionalistas na sociedade contemporânea, além de analisar esse processo, que ainda é aceito e adotado em grupos ‘neonazis’. Os grupos nacionalistas atuam em todo o mundo, principalmente, na Europa, nos Estados Unidos e no Brasil, devido à miscigenação étnica-cultural. No Brasil, o alvo dos preconceituosos são os migrantes nordestinos, mas também, negros homossexuais e prostitutas – esses tidos como escória da sociedade. Assim, observa-se nos pensamentos neonazistas o processo de negação ideológica de diferentes culturas e credos, a acentuar o racismo e, portanto, a discriminação, uma vez que a ideologia neonazista se baseia em conceitos e princípios que têm como finalidade a persuasão humana de adesão.
		
		\vspace{\onelineskip}
		
		\noindent
		\textbf{Palavras-chave}: Sociedade. Ideologia. Grupos Neonazistas.
	\end{resumoumacoluna}
	
	\section*{Introdução}
	
O século XX foi um importante período para o desenvolvimento econômico, social, político e ideológico da atualidade, devido às guerras mundiais, sobretudo a Segunda (1939-1945), caracterizada pelo ódio, racismo e violência.

Com a derrota alemã na Primeira Guerra Mundial (1914-1918), o país via-se necessitado de um “Salvador da Pátria”, e Adolf Hitler foi o consagrado, com uma ideologia antissemita essencial para vitória nas urnas. Logo, “a gravidade dos problemas enfrentados fez com que as soluções radicais fossem bem-vindas (ARNAUT \& MOTTA, 1994, p.07).

A partir desse pressuposto, pode-se observar que, a ideologia de Hitler não desapareceu, mas foi adaptada à atualidade, pois o século XXI abrange novos ideais de convívio em sociedade. Os tipos de preconceito foram alterados e os meios de praticá-lo também, uma vez que as tecnologias, principalmente a Internet e as Redes Sociais, proporcionam essas condições aos neonazistas. Sendo que, no Brasil, o alvo dos atos preconceituosos são os nordestinos, negros, homossexuais e os drogados, todos caracterizados pela exclusão social.
	
	\section*{Material e Método}
	
	Técnico em Agroecologia Integrado ao Ensino Médio, nos encontros do grupo PDA (Práticas Discursivas na Amazônia), sob a orientação do professor Sérgio Nunes de Jesus, do Curso Técnico em Agroecologia do Instituto Federal de Educação, Ciência e Tecnologia de Rondônia – IFRO/Campus Cacoal.
	
	As atividades foram efetivadas por meio de leituras bibliográficas e discussões referentes ao tema em voga, durante o período de fevereiro a junho de 2016. Para alcançar os resultados, foram utilizados computadores com acesso à Internet e, por conseguinte, a livros e artigos científicos que tratam do neonazismo.
	
	\section*{Resultados e Discussão}
	
	O neonazismo estabelece-se novamente no final da década de 1970, atribuindo características oriundas dos sete anos de terror nazista. Segundo Szklarz (2014), essa expressa ideias velhas, entretanto com uma nova configuração, impactante e brutal, a afeta uma parcela da população vulnerável. Visto que: 
	
	\begin{citacao}
		No Brasil, grupos como carecas do ABC, White Power escolhem suas vítimas de forma aleatória. Alguns perseguem gays, outros nordestinos, negros, judeus, bolivianos ou usuários de drogas. Isso quando seus integrantes não se matam em disputas entre as facções (SZKLARZ, 2014, p. 228).
	\end{citacao}
	
	Desse modo, as vítimas citadas anteriormente são referidas como escórias da sociedade, considerados pelos neonazis os culpados pelos problemas contidos na atualidade, bem como o desemprego, tráficos, crimes e a desordem social, a expressar o ódio por meio da violência, física, verbal ou psicológica.
	
	Assim, em meados de 1960, surge na Inglaterra os skinheads, ou “Cabeças Raspadas”, uma subcultura caracterizada pelos princípios punk. Entretanto, posteriormente, um novo estilo skinheads surge, sendo esse ligado à ideologia nazista.
	
	E, nos Estados Unidos da América, por exemplo, os membros da supremacia branca são apreciadores dos Skinheads alemães e praticam a violência contra imigrantes e negros. Apesar do decréscimo do número de integrantes, as pessoas visam um mundo “purificado” por uma única raça, sendo essa branca, segundo eles, estimada superior.
	
	Sendo que, no Brasil o neonazismo é caracterizado por serem praticado na região Sul e Sudeste por meio de ações físicas e verbais, mas em outras regiões do país os integrantes realizam suas atividades através da internet. Logo, eles buscam baixar conteúdo com teores nacionalistas e os expandem via redes sociais, assim observa-se que os ideais são praticados em grandes centros onde há distinção entre as classes, diferentemente da região norte do país onde esse aspecto não é apresentado intensamente.
	
	Desse modo, pode-se observar que, os grupos possuem características distintas, mas todos estão vinculados à ideologia neonazista, com base no exacerbado nacionalismo e na xenofobia, que permite a prática discriminatória. Porém, é válido ressaltar que, determinados participantes aderem ao neonazismo unicamente por causa da adrenalina advinda da violência (Szklarz, 2014).
	
	Na Europa, destacam-se quantitativamente as ações neonazis, devido à miscigenação racial proveniente do ascendente fluxo imigratório do século XXI, caracterizado pela presença de refugiados de guerras dos continentes vizinhos, sobretudo do continente africano.
	
	Existem selas neonazis espalhadas pelo mundo, além dos integrantes também atuarem na política, como a negação de exílio dos refugiados da guerra na Síria, iniciada em 2011, a utilizar como justificativa a desordem sociocultural e econômica da Europa.
	
	E, a partir deste ponto pode-se observar que a mídia não expõe as ações neonazistas de acordo com os atos praticados, assim omitindo informações a sociedade. Pois, o neonazismo é uma ação que fragiliza as classes sociais inferiores, onde as políticas de acesso a serviços são escassas, portanto, as vítimas primárias são as crianças e jovens fragilizados psicologicamente com a desdenha política e dos extremas-direitas.
	
	E, a mídia mundial também não evidencia ações neonazistas, que de acordo com Szklarz (2014), mataram mais que o terrorismo, até o ano de 2014.
	
	É válido ressaltar que, a sociedade tem acesso à história da Segunda Guerra Mundial (1939-1945), e, portanto, conhece as atrocidades causadas por Hitler, em sua época. Porém, desconhecem a presença dos ideais nazistas na atualidade, caracterizadas pela discriminação por fatores étnico-raciais e culturais.
	
	\section*{Conclusões}
	
Constatou-se que, após as Guerras Mundiais, do início do século XX, líderes mundiais instituíram a ONU (Organização das Nações Unidas), para atribuição dos direitos humanos, e manutenção da paz mundial.

Contudo, na década de 1970, o nazismo retorna à Europa, vinculado à ideologia nacional-socialista. Assim, apesar de possuir a finalidade de conter e condenar os ativistas neonazis, a ONU não exerce essa função efetivamente.

Os neonazistas são pessoas que possuem um caráter dualístico pois, no âmbito social que lhes agradam, a violência é repudiada, mas nos grupos, é excitada e praticada com naturalidade. Consequentemente, a alterar os princípios éticos e morais da sociedade contemporânea.

Assim, pode-se observar que os indivíduos praticantes do neonazismo expõem seus ideais em locais públicos, com o objetivo de explanar seus ideais, a afetar os indesejados, e principalmente à alteração da formação discursiva e ideológica de jovens promissores à cultura neonazi.

Portanto, pode-se considerar que, o neonazismo está presente na atualidade e em contraposição, as classes sociais, econômicas e políticas dominantes, a incluir o poder da mídia, principalmente da telecomunicação, evitam expor os episódios acontecidos, uma vez que, esse contexto apresenta-se como realidade subjetiva, a lesar importantes indivíduos hierárquicos socialmente, quando exposto.
	
	\section*{Instituição de Fomento}
	
	Instituto Federal de Educação, Ciência e Tecnologia de Rondônia – IFRO, Campus Cacoal.
	
	\sloppy
	\section*{Referências}
	
\noindent ARNAUT, Luiz; MOTTA, Rodrigo P. Sá. A segunda grande guerra. 4. ed. São Paulo: Atual, 1994.

\noindent SZKLARZ, Eduardo. Nazismo: como ele pôde acontecer. São Paulo: Abril, 2014.
	
\end{document}