\documentclass[article,12pt,onesidea,4paper,english,brazil]{abntex2}

\usepackage{lmodern, indentfirst, nomencl, color, graphicx, microtype, lipsum,textcomp}			
\usepackage[T1]{fontenc}		
\usepackage[utf8]{inputenc}		

\setlrmarginsandblock{2cm}{2cm}{*}
\setulmarginsandblock{2cm}{2cm}{*}
\checkandfixthelayout

\setlength{\parindent}{1.3cm}
\setlength{\parskip}{0.2cm}

\SingleSpacing

\begin{document}
	
	\selectlanguage{brazil}
	
	\frenchspacing 
	
	\begin{center}
		\LARGE GARIMPO BOM FUTURO, BREVE ESTUDO DE CASO DA EXPLORAÇÃO DA CASSITERITA NO\\MUNICÍPIO DE ARIQUEMES
		
		\normalsize
		Francisco Clayton N. da Cunha\footnote{Bolsista (modalidade), FCLAYTON\_NC@yahoo.com.br, campus Porto Velho-Zona Norte.} 
	Esiomar Andrade S. Filho\footnote{Orientador(a), esiomar.silva@ifro.edu.br, campus Porto Velho-Zona Norte.} 
	Denise Ton Tiussi\footnote{Co-orientador(a), denise.tiussi@ifro.edu.br, campus Porto Velho-Zona Norte.}
	\end{center}
	
	% resumo em português
	\begin{resumoumacoluna}
	Em 1983 foi descoberto o garimpo Bom Futuro, no Município de Ariquemes, sendo denominado o maior a céu aberto do mundo. No local era garimpada a cassiterita, esta passava por beneficiamento, retirando assim o estanho, que e utilizado em diversas finalidades. O garimpo contribuiu e ainda contribui para a economia do Estado de Rondônia, sendo este o minério mais explorado e exportado do estado. Com isso destacaremos a finalidade deste trabalho que é propor um breve estudo do garimpo de cassiterita, parte integrante da economia do Estado de Rondônia.
	
		\vspace{\onelineskip}
		
		\noindent
		\textbf{Palavras-chave}:garimpo, cassiterita, Economia.
	\end{resumoumacoluna}
	
	\section*{Introdução}
	
O garimpo Bom Futuro é de extrema importância para o Estado de Rondônia. Foi com sua descoberta que elevou-se o crescimento econômico e populacional, pois proporcionou a vinda de imigrantes de outros estados para aventurarem-se na garimpagem de cassiteritas.

O Brasil possui aproximadamente 10\% das reservas mundiais de estanho contido, sendo a terceira maior do mundo. É o quinto maior produtor mundial, com 16.830 toneladas (metal contido no concentrado) produzidas em 2013 (7,1\% do total). As reservas brasileiras estão localizadas em sua maior parte na região amazônica: província mineral do Mapuera (mina do Pitinga), no Amazonas e na província estanífera de Rondônia, nas minas de Bom Futuro, Santa Bárbara, Massangana e Cachoeirinha (PONTES, 2013).

No começo a garimpagem de cassiterita no garimpo Bom Futuro ajudou a elevar o Brasil a ser o maior produtor mundial de estanho, mas hoje em dia não esta em primeiro lugar nesta produção, no entanto, continua sendo fonte de riqueza do estado.
	
	\section*{Material e Método}
	
Trata-se de pesquisa teórica de natureza exploratória e descritiva. A partir dos conceitos de atividade econômica e desenvolvimento regional, abordou-se o Garimpo Bom Futuro e seus impactos econômicos, sociais e humanos.

A produção do trabalho foi antecedida de ampla pesquisa bibliográfica e intensa observação participativa.
	
	\section*{Resultados e Discussão}
	
Localizado no Município de Ariquemes o Garimpo Bom Futuro tem uma grande importância para o Estado de Rondônia, sendo a maior reserva de cassiterita do Brasil. Em 1983, após sua descoberta, houve uma grande mudança na população deste local, onde muitos agricultores, madereiros e comerciantes trocaram suas atividades pela garimpagem, vindo até mesmo imigrantes de outros estados para aventurarem na garimpagem deste minério, criando assim uma população de mais de 15000 garimpeiros entre homens e mulheres (RONALLTTI, 2008).

No auge do ciclo de exploração, a cassiterita encontrada neste garimpo correspondia a 80\% da produção no país. Ajudando nesta época a elevar o pais a posição de maior produtor mundial do triênio entre 1988 – 1990 (RODRIGUES,2001), mas o estanho retirado desta área ainda é uma grande fonte de riqueza para Rondônia. Hoje, o Garimpo de Bom Futuro não é tão importante quanto foi no passado, apesar de abrigar, na contemporaneidade, aproximadamente três mil pessoas, o Garimpo não possui a mesma importância econômica, social e política que possuía há 20 anos.

\subsection*{A história da cassiterita}

\textbf{1º Ciclo} – iniciou-se em 1958 com o processo exploratório em altíssima escala, esse processo era manual e alterou sobremaneira as bases econômicas, migratórias e políticas do então Território Federal. Dessa forma, o grande auge da migração pela exploração da Cassiterita se deu na década de 1960.

\textbf{2º Ciclo} – iniciou-se no final da década de 1960, apesar de a primeira fase do Ciclo da Cassiterita continuar em ascensão, ainda que a exploração desse minério fosse manual, com técnicas e métodos rudimentares, os garimpeiros ou pequenas empresas de garimpo utilizavam-se das seguintes ferramentas: enxadão, picaretas, pás, peneiras etc. Nesse período, a produção da Cassiterita era maior do que quando realizado com equipamentos e máquinas sofisticados.

Com a abertura da BR 364 e a chegada dos primeiros caminhões a exploração da cassiterita cresceu, alterando o processo de povoamento do território. Ocorrendo uma mudança na economia deste, que até então era fundamentada no extrativismo rudimentar. Inicialmente a garimpagem da cassiterita era realizada de forma clandestina e manual, os garimpos eram densamente povoados, onde através deste povoamento se desenvolveu as primeiras pistas de pouso de aeronaves no Estado 

A extração desse minério era um grande sacrifício, geralmente a equipe de trabalho era constituída por duas ou três pessoas responsáveis pela escavação, que a mesma atingia a profundidade de 3 metros, mas somente após 2 a 3 semanas se fazia a retirada do cato (minério no estado bruto). Na sequência, se lavava o material bruto e separavam o material de descarte e minério. Desta forma, era encaminhado às cantinas compradoras e de lá transportado para Porto Velho que seguia para o sul do Brasil.

\textbf{3º Ciclo} – Na década de 1970, uma nova prática política e econômica foi implantada. O governo decidiu proibir a extração da Cassiterita realizada via processo manual. Desse modo, o Governo Militar, com base no Código de Mineração de 1967, deu exclusividade a empresas mineradoras de grande e médio porte, as quais vieram à Rondônia e instalaram-se nesse Estado, iniciando o processo de garimpagem mecanizado.

Dessa maneira, na década de 1970, Rondônia era um dos sete maiores produtores de Cassiterita do mundo, inclusive a qualidade do estanho rondoniense era melhor do que o produzido na Malásia. Esse fato é atestado por Oliveira (2001, p. 66):

\begin{citacao}	
$[...]$ O engenheiro Frederico Hoepken enviou amostras do minério para o Rio de Janeiro com o intuito de confirmar suas suspeitas de que se tratava de cassiterita, bem como testar seu grau de teor. O professor Elysiário Távora, encarregado de examinar o minério, atestou que realmente se tratava da cassiterita e que esta era de teor elevadíssimo, 5 kg de SnO2/m3 .Comparando com a da Malásia (área de maior produção até então), cujo teor era de 0, 350 kg Sn 02/m3, observou que tratava-se de um minério de alto valor no mercado nacional e internacional.

\end{citacao}

\subsection*{Impactos na economia do Estado de Rondônia}

	Na economia do Estado de Rondônia a cassiterita destaca-se no setor primário como minério mais explorado desta região, sendo encontrada em grandes escalas no garimpo Bom Futuro. O crescimento da produção de cassiterita rendeu ao estado 47\% do PIB a nível nacional, produzindo assim, 13.667 toneladas deste minério em 2012. Rondônia tem quase a metade da jazida de cassiterita do Brasil, sendo um grande exportador deste minério.
	
	\section*{Conclusões}
	
Diante do exposto, percebemos a importância do garimpo Bom Futuro para o crescimento do Estado de Rondônia, pois foi através da garimpagem da cassiterita que obtivemos uma elevação na economia.

A cassiterita sendo um minério de extrema importância tanto para o estado quanto para o Brasil, elevou os índices econômicos do estado de Rondônia e do Brasil, tornando-o um dos primeiros produtores de estanho do mundo.

Conclui-se que, apesar da importância do Garimpo Bom Futuro e dos bons resultados da atividade mineradora em Rondônia, nossa economia continua necessitando de mais investimentos e políticas públicas na área da mineração e em vários outros fatores de produção, como no modal viário, ferroviário e hidroviário, setor de energia, recursos humanos, entre outros. Dessa forma, os agentes econômicos que aqui investem e trabalham terão condições de produzir mais e melhor, e a população como um todo será a maior beneficiada.
	
	\section*{Referências}
	
	\sloppy
	
\noindent BRASIL. Extensão de Cassiterita gera danos socioambientais. Disponível em:
<http//:www.verbetes.cetem.gov.br>. Acessado em: 8 jun. 2016.

\noindent BRASIL. Rondônia historia de ciclos econômicos. Disponível em:<http//:www.sociologiaemtela/rondonia-historia-de-ciclo-economicos>. Acessado em 08 jun. 2016.

\noindent BRASIL. Rondônia detêm quase a metade da cassiterita do Brasil. Disponível em: <http//:www.portal.amazonia.com.br>. Acessado em 08 jun 2016.

\noindent DANIEL, João Paulo. Cassiterita: pedra da morte.
< Http://pastorjoaodaniel. blogspot.com.br/2009/12/cassiterita-pedra-da- morte.html. >Acesso em 08 jun 2016.

\noindent PONTES, Eduardo P. Sumário mineral 2014 – DNPM. Disponível em:
<http://www.dnpm.gov.br/dnpm/sumarios/estanho-sumario-mineral-2014> Acesso em 08 de jun 2016.

\end{document}
