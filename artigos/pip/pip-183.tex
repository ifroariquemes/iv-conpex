\documentclass[article,12pt,onesidea,4paper,english,brazil]{abntex2}

\usepackage{lmodern, indentfirst, nomencl, color, graphicx, microtype, lipsum,textcomp,amsmath}			
\usepackage[T1]{fontenc}		
\usepackage[utf8]{inputenc}		

\setlrmarginsandblock{2cm}{2cm}{*}
\setulmarginsandblock{2cm}{2cm}{*}
\checkandfixthelayout

\setlength{\parindent}{1.3cm}
\setlength{\parskip}{0.2cm}

\SingleSpacing

\begin{document}
	
	\selectlanguage{brazil}
	
	\frenchspacing 
	
	\begin{center}
		\LARGE TECNOLOGIA E MANEJO DO COCO, ESPÉCIE DE USO MÚLTIPLO DA AMAZÔNIA1\footnote{Trabalho realizado dentro da área de Conhecimento CNPq/CAPES: Engenharia Florestal com financiamento do Edital de Pesquisa 03 de 2015.}
		
		\normalsize
	Carla Karoline Cruz Pereira\footnote{Bolsista (PIBIC-EM), carlaregma@gmail.com, Campus Ji-Paraná.} 
		Luiz Carlos Rezende Nunes\footnote{Bolsista (PIBIC-EM),resends98@gmail.com, Campus Ji-Paraná.} 
		Andreza Mendonça\footnote{Orientadora, andreza.mendonca@ifro.edu.br, Campus Ji-Paraná.} 
		Alice Sperandio Porto\footnote{Co-orientadora, alice.porto@ifro.edu.br, Campus Ji-Paraná.}
	Maria Elessandra R. Araújo\footnote{Co-orientadora, maria.elessandra@ifro.edu.br, Campus Ji-Paraná.} 
	\end{center}
	
	% resumo em português
	\begin{resumoumacoluna}
	Nas comunidades rurais em Rondônia a extração do óleo de coco é uma alternativa de diversificação da produção familiar. O objetivo do projeto foi manejar o endosperma do coco para diversificar a produção a partir da extração do óleo e produção de farinha na região central de Rondônia. Os cocos foram coletados nas propriedades rurais na zona rural de Ji-Paraná, Rondônia. Em seguida, os endospermas foram beneficiados e secos em estufa de ventilação forçada nas temperaturas de 60, 70 e 80oC até alcançarem os teores de umidade 6,8 e 10\%. Após a secagem, foram triturados e prensados em prensa hidráulica. O óleo e a torta foram avaliados quanto a qualidade nos índices de acidez e proteína, respectivamente, de acordo com Adolfo Lutz (2008). Verificou-se que todos os óleos extraídos tiveram índice de acidez variando de 1,28 a 3,29 meq KOH inferiores ao valor recomendado pela Resolução 270 de 2005 da ANVISA. As tortas de coco tiveram índice de proteína variando de 6,55 a 10,72\%, indicando que o produto pode ser utilizado para produção de alimento ou ração.
		\vspace{\onelineskip}
		
		\noindent
		\textbf{Palavras-chave}: Uso múltiplo. Óleos vegetais. Cocos nucifera L.
	\end{resumoumacoluna}
	
	\textual
	
	\section*{Introdução}
	
	Na região central do estado de Rondônia, os procedimentos de extração do óleo de coco e seus resíduos são atividades, no geral, familiares em moldes rudimentares. A falta de políticas pública específica para a atividade bem como investimentos em tecnologia para o aprimoramento da produção colabora para a continuidade da atividade a margem da economia. A secagem do endosperma, por exemplo, é um dos procedimentos mais importantes para extração do óleo e formação da farinha. Nesse processo, a temperatura, é um dos fatores relevantes, podendo afetar as propriedades organolépticas e nutricionais.
	
	A qualidade do óleo produzido nas propriedades rurais é avaliada de maneira empírica pela cor e sabor a partir do comércio nas feiras livres da região central de Rondônia. É importante salientar que há poucos estudos que descrevem todo o processo de preparo e prensagem do endosperma do coco que assegurem maior quantidade e qualidade do óleo extraível, nem tão pouco indicam o manejo adequado dos resíduos da prensagem para formação de sub-produtos como farinha de coco.
	
	Portanto, faz-se necessário estudo sobre a secagem e manejo adequado do endosperma do coco para extração do óleo por meio de prensa que garanta maior quantidade e qualidade do óleo extraível e menor volume de resíduos.
	
	\section*{Material e Método}
	
	Os cocos (Cocos nucifera L.) foram coletados nas propriedades particulares na zona rural do município de Ji-Paraná, Rondônia. Após a coleta, o endosperma foi separado do restante do fruto com auxílio de um facão e uma colher de inox.
	
	O teor de água dos endospermas foi determinado pelo método em estufa a temperatura de 105oC por 24 horas de acordo com Brasil (2009). Os endospermas foram secos em estufa de ventilação forçada nas temperaturas de 60, 70 e 80oC até alcançarem os teores de água de 10, 8 e 6\% antes da extração do óleo. Os tratamentos foram determinados por meio do acompanhamento da perda de massa das sementes durante a secagem.
	
	A massa das amostras, corresponde a cada um dos graus de umidade desejado, foi previamente determinado por meio da Equação 1 (ALMEIDA et al., 2006).
	
	\begin{equation}
	Xi = \begin{matrix}
	wi & & ws \\ 
	& wi &
	\end{matrix}
	\end{equation}
	
\noindent 	em que:
	
\noindent 	$Xi$ -- umidade
	
\noindent 	$wi$ -- massa umida
	
\noindent 	$ws$ -- massa seca
	
\subsection*{Extração por prensa}
	
	O endosperma ao atingir o teor de água desejado foi triturado com auxílio de um liquidificador e prensado em uma prensa hidráulica sob pressão de 15 toneladas por um período de 4 horas. Foi usado 500g de endosperma por amostra para extração do óleo. Cada tratamento (combinação da temperatura e umidade) tinha quatro repetições. Ao final da extração, o óleo extraído e o resíduo foram pesados em uma balança analítica de precisão de 0,01g.
	
\subsection*{Qualidade dos óleos extraídos}
	A qualidade dos óleos de coco foi determinada a partir das análises dos índices de acidez e peróxido de acordo com a metodologia descrita por Adolfo Lutz (2008).
	
\subsection*{Aproveitamento da torta de coco}
	
	A torta resultante de cada tratamento da extração do óleo foi determinado o teor de água de acordo com metodologia de Brasil (2009) e comparado com a legislação vigente para farinha e quanto a qualidade serão avaliados os teores de proteína de acordo com a metodologia descrita por Adolfo Lutz (2008).
	
	
	
	\section*{Resultados e Discussão}
	
Verificou-se que todos os óleos extraídos tiveram índice de acidez variando de 1,28 a 3,29 meq KOH g-1 (Tabela 1) inferiores ao valor recomendado pela Resolução 270 de 2005 da ANVISA para óleos vegetais brutos prensados a frio (4,0 meq KOH g-1).

Os óleos tiveram o mesmo padrão de coloração: amarelo claro. O índice de acidez é um dos parâmetros referenciais para determinar a qualidade da conservação de óleos vegetais (BRASIL, 2005). Deve-se ressaltar que o óleo de coco é rico em ácido láurico, o que o torna resistente à oxidação não enzimática e ao contrário de outros óleos e gorduras apresentam temperatura de fusão baixa e bem definida (24,4 a 25,6°C). As gorduras láuricas são muito usadas na indústria cosmética e alimentícia onde em virtude de suas propriedades físicas e resistência à oxidação são muito empregadas no preparo de gorduras especiais para confeitaria, sorvetes, margarinas e substitutos de manteiga de cacau (MACHADO et al.; 2006).

O óleo de coco virgem não é um medicamento, e sim um alimento complementar coadjuvante na prevenção de diversas doenças. Por isso, deve ser consumido diariamente para que o organismo obtenha uma reserva de ácidos graxos, presentes no óleo de coco (SANTOS et al.; 2013).

\begin{table}[]
	\centering
	\caption{Índice de acidez dos óleos de coco extraído por meio de prensa sob diferentes teores de água (6, 8 e 10\%) e temperaturas (60,70 e 80°C).}
	\label{my-label}
	\begin{tabular}{lll}
		\hline
		Temp. & Teor de água & Índice de acidez \\
		(°C)  & (\%)         & mg KOH g-1       \\
		\hline
		& 6            & 3,25             \\
		60    & 8            & 3,29             \\
		& 10           & 1,63             \\
		\hline
		& 6            & 2,40             \\
		70    & 8            & 2,17             \\
		& 10           & 2,13             \\
		\hline
		& 6            & 1,35             \\
		80    & 8            & 1,31             \\
		& 10           & 1,28 \\
		\hline            
	\end{tabular}
\end{table}

As tortas de coco tiveram índice de proteína variando de 6,55 a 10,72\%, indicando que o produto pode ser utilizado para produção de alimento ou ração. A farinha de coco pode ser utilizada na produção de bolos e massas como fonte de proteína. A legislação vigente determina teor máximo de 15\% de umidade para as farinhas (INMETRO, 2005). A cultura do coco apresenta uma série de vantagens agroeconômicas, sociais e ambientais se comparada a outras culturas desenvolvidas, vantagens estas que viabilizam a atividade tornando-a rentável, capaz de retornar o capital investido. Socialmente, o cultivo do coqueiro permite a fixação do homem no campo, garantindo a ocupação de expressivo contingente rural durante o ano inteiro (FONTENELE, 2005).
	
	\section*{Conclusões}
	
Os óleos de coco com teor de água a 10\% tiveram os menores índices de acidez independente da temperatura de secagem.

Os teores de água menores de 10\% favoreceram o aumento do teor de proteína presente na torta de coco.
	
	\section*{Instituição de Fomento}
	
	Instituto Federal de Rondônia, Câmpus Ji-Paraná por meio do edital 03 de 2015, concessão de duas bolsas de Iniciação Científica, modalidade: Ensino Médio.
	
	\section*{Referências}
	
	\sloppy
	
\noindent ADOLFO LUTZ. Métodos físico-químicos para análises de alimentos. 4ª.edição.1ª.edição digital. p.595-629. 2008.

\noindent ALMEIDA, F. de A. C.; DUARTE, M. E. R. M.; MATA, M. E. R. M. C. Tecnologia de
armazenamento em sementes. Campina Grande: UFCG, 2006.

\noindent BRASIL. Agência de Vigilância Sanitária. Resolução nº270 de 2005.
FONTENELE, R. E. S. Cultura do coco no Brasil: caracterização do mercado atual e perspectivas futuras. XLIII Congresso da SOBER. 2005.

\noindent MACHADO, G. C.; CHAVES, J.B.P.; ANTONIASSI, R. Composição em ácidos
graxos e caracterização física e química de óleos hidrogenados de coco babaçu. Revista Ceres. 53(308), 463, 2006.

\noindent MARTINS, C. R. e JÚNIOR, L.A.de J. Evolução da produção de coco no Brasil e o comércio internacional. Panorama 2010. EMBRAPA, 2011.

\noindent Lima, D. M. de A. e Wilkinson, J. (Orgs.). Inovação nas tradições da agricultura familiar. Brasília: CNPq/Paralelo 15, 2002.

\noindent SANTOS, J. R. M.; MARTINS, J. S., FREIRE, M. S.; SANTOS, J. C. O.
Caracterização química e físico-química do óleo de coco extra virgem (Cocos nucifera L.). 5° Congresso Norte-Nordeste de Química. 2013.
	
\end{document}
