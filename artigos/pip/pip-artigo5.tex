\documentclass[article,12pt,onesidea,4paper,english,brazil]{abntex2}

\usepackage{lmodern, indentfirst, nomencl, color, graphicx, microtype, lipsum}			
\usepackage[T1]{fontenc}		
\usepackage[utf8]{inputenc}		

\setlrmarginsandblock{2cm}{2cm}{*}
\setulmarginsandblock{2cm}{2cm}{*}
\checkandfixthelayout

\setlength{\parindent}{1.3cm}
\setlength{\parskip}{0.2cm}

\SingleSpacing

\begin{document}
	
	\selectlanguage{brazil}
	
	\frenchspacing 
	
	\begin{center}
		\LARGE ESTUDOS DOS HABITOS ALIMENTARES DE ESTUDANTES DO ENSINO BÁSICO, TECNICO E TECNOLÓGICO
		
		\normalsize
		Fernanda Lima e Silva\footnote {Bolsista, limaf5522@gmail.com, Campus Porto Velho Calama, membro do GESSTEC/IFRO. } 
		Danielle Menezes Marrieli\footnote{Colaboradora, dani.pimentinha2014@hotmail.com, Campus Porto Velho Calama, membro do GESSTEC/IFRO.} 
		Iranira Geminiano de Melo\footnote{Orientadora, iranira.melo@gmail.com, Campus Porto Velho Calama, membro do GESSTEC/IFRO. } 
		Célio José Borges\footnote {Co-orientador, ceborges@gmail.com, DEF/UNIR, Campus Porto Velho, membro do GESSTEC/IFRO.} 
	\end{center}
	
	% resumo em português
	\begin{resumoumacoluna}
		Este estudo teve a finalidade de informatizar questionário para avaliação dos hábitos alimentares dos alunos de curso técnico integrado ao ensino médio, do Instituto Federal de Educação, Ciência e Tecnologia de Rondônia (IFRO), Campus Porto Velho Calama, quanto aos fatores intervenientes da má alimentação no âmbito institucional, ponderando as possibilidades de intervenção da Instituição para a contribuição na melhoria da saúde alimentar dos alunos. Realizou-se a informatização do questionário “Hábitos alimentares”, a partir da linguagem de programação PHP e disponibilização na web. Os dados da pesquisa poderão ainda ser utilizados pelos professores de educação física para sensibilizar os alunos na adoção de práticas saudáveis.
		
		
		\vspace{\onelineskip}
		
		\noindent
		\textbf{Palavras-chave}: Saúde. Informatização. Alimentação.
	\end{resumoumacoluna}
	
	\section*{Introdução}
	
Uma rotina alimentar desequilibrada está diretamente vinculada a uma má qualidade vida, quando no âmbito da vida acadêmica esta situação tem se revelado persistente, pois a escassez de tempo é frequente, fazendo com que os alunos optem por substituir refeições completas por lanches rápidos e pouco saudáveis, esta prática associada ao stress e ao desgaste emocional pode desencadear doenças e uma saúde frágil (NAHAS, BARROS \& FRANCALACCI,2006).

Neste contexto, o estudo tem como objetivo informatizar questionários para avaliação dos hábitos alimentares dos alunos do ensino básico, técnicos e tecnológicos do IFRO. O sistema permite aos alunos retornar ao site e refazer o teste quando desejado, com o objetivo de propiciar a ele um acompanhamento dos hábitos alimentares. De modo que o aluno, percebendo que sua alimentação precisa melhorar, possa refletir e fazer as modificações necessárias e em uma nova avaliação possa ver sua evolução a partir da comparação da pontuação alcançada em cada avaliação. Os dados da pesquisa poderão ainda ser utilizados pelos professores de educação física para sensibilizar os alunos na adoção de práticas saudáveis.
	
	\section*{Material e Método}
	
A princípio foi desenvolvido um sistema com a intenção de digitalizar o questionário e criar um banco de dados para registrar as informações dadas pelos alunos. Neste sistema os alunos irão responder ao questionário “Você se alimenta bem?”, de Nahas (2013), que foi codificado utilizando linguagem de programação web PHP e o Sublime Text v3, que possui uma versão gratuita com recursos básicos para facilitar a programação em PHP. Ao finalizar a informatização do questionário ele foi disponibilizado no site gesstec.org/estilodevida/1-introducao.php e as respostas que são armazenadas nas variáveis PHP são enviadas para o banco de dados de onde podem ser compiladas para análise. Esse questionário é composto por perguntas objetivas sobre vários aspectos dos hábitos alimentares.

O sistema permite aos alunos retornarem ao site e refazerem a todos os questionários quando desejado, com o objetivo de propiciar ao mesmo um acompanhamento do seu rendimento nos índices avaliados pelo questionário. Com  o auxílio deste sistema os alunos podem conhecer o resultado da avaliação alimentar e podem planejar modificações para melhorar a pontuação em uma próxima avaliação.
	
	\section*{Resultados e Discussão}
	
O resultado da primeira etapa do projeto é um site com questionário que possibilita avaliar os hábitos alimentares dos alunos, tendo como vantagem o conhecimento que os alunos terão da pontuação alcançada, imediatamente ao preenchimento do questionário, e ainda a mensagem de incentivo que receberão dependendo da pontuação obtida: de 160 a 200 pontos – Excelente; de 120 a 159 pontos – Bom, mas pode melhorar; Inferior a 120 pontos – Precisa melhorar bastante.

Na próxima fase do projeto o site será divulgado para os alunos e se procederá a coleta de dados utilizando a ferramenta desenvolvida nessa primeira etapa, possibilitando outras reflexões a respeito da alimentação dos estudantes como exemplo: Os alunos se alimentam diferentes de acordo com o ano de ensino? É dessemelhante entre os estudantes de acordo com o curso? As moças se alimentam melhor do que os rapazes? Em quais aspectos os hábitos alimentares se aproximam e em quais diferem? O que predomina na alimentação dos alunos, os alimentos naturais ou os industrializados? Qual o perfil alimentar dos alunos do ensino técnico integrado ao médio?
	
	\section*{Conclusões}
	
A reflexão sobre os hábitos alimentares proporciona uma auto avaliação sobre a própria saúde, e isto pode conscientizar e motivar para melhorar a dieta. Com acesso e utilização do questionário, disponibilizado no site, e o conhecimento do resultado pessoal o aluno tem uma avaliação clara e específica da alimentação dele e isto pode sensibiliza-lo a melhorar a qualidade das refeições e proteger a saúde de forma eficiente.	

Assim, a primeira etapa da pesquisa é indispensável a segunda, que irá analisar o perfil alimentar dos alunos a partir da utilização desse questionário que foi informatizado, permitindo maior agilidade na coleta e compilação dos dados e, consequentemente, em novas publicações.

	\section*{agradecimentos}
	
	À Pró-Reitor de Pesquisa, Inovação e Pós-Graduação (PROPESP/IFRO) e ao Departamento de Pesquisa, Inovação e Pós-Graduação (DEPESP/Campus Porto Velho Calama) pela oportunidade de divulgação da pesquisa e apoio ao Grupo de Estudos Saúde, Sociedade e Tecnologia (GESSTEC).
	
	\section*{Instituição de Fomento}
	
Fundação Rondônia de Amparo ao Desenvolvimento das Ações Científicas e Tecnológicas e à Pesquisa do Estado de Rondônia (FAPERO) e Conselho Nacional de Desenvolvimento Científico e Tecnológico (CNPq).
	
	\section*{Referências}
	
	\noindent NAHAS, Markus V.; BARROS, Mauro V. G.; FRANCALACCI, Vanessa. O pentáculo do bem-estar - base conceitual para avaliação do estilo de vida de indivíduos ou grupos. Revista Brasileira de Atividade Física e Saúde. Vol. 5, N. 2, 2006.
	
	\noindent NAHAS, Markus V. Atividade Física, Saúde e Qualidade de Vida. 6 ed. Londrina: Midiograf, 2013.
	
\end{document}