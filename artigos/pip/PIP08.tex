
\documentclass[article,12pt,onesidea,4paper,english,brazil]{abntex2}

\usepackage{lmodern, indentfirst, nomencl, color, graphicx, microtype, lipsum}			
\usepackage[T1]{fontenc}		
\usepackage[utf8]{inputenc}		

\setlrmarginsandblock{2cm}{2cm}{*}
\setulmarginsandblock{2cm}{2cm}{*}
\checkandfixthelayout

\setlength{\parindent}{1.3cm}
\setlength{\parskip}{0.2cm}

\SingleSpacing

\begin{document}
	
	\selectlanguage{brazil}
	
	\frenchspacing 
	
	\begin{center}
		\LARGE CONCENTRAÇÃO DE PIGMENTOS FOTOSSINTÉTICOS EM ESPÉCIES VEGETAIS DE UM TRECHO DE FLORESTA ESTACIONAL SEMIDECIDUAL, NO MUNICÍPIO DE \\COLORADO DO OESTE/RO\footnote{Trabalho realizado dentro da (área de Conhecimento CNPq/CAPES: Ecologia de Ecossistemas.) com financiamento do IFRO.}
		
		\normalsize
		Caroline Alves Lima\footnote{Bolsista, modalidade PIBITI. E-mail: carolyne.ifro@hotmail.com, Campus Colorado do Oeste.} 
		Matheus Henrique  Brandão\footnote{ Colaborador, modalidade PITI. E-mail: mhbrandao@hotmail.com, Campus Colorado do Oeste.} 
		Anderson Pereira da Silva\footnote{Bolsista, modalidade PIBIC EM. E-mail: andersonifro@gmail.com. Campus Colorado do Oeste.} \\
	Carolina Ferreira Galvão de Holanda\footnote{Orientadora, modalidade PITI. E-mail: roberta.holanda@ifro.edu.br, Campus Colorado do Oeste } 
	Jéssica Danila Krugel\footnote{Co-orientadora, modalidade PITI. E-mail: jessica.kruegel@ifro.edu.br, Campus Colorado do Oeste.}
	
	\end{center}
	
	% resumo em português
	\begin{resumoumacoluna}
	As clorofilas são pigmentos fotossintéticos que atuam na conversão da energia luminosa em química, fornecendo, energia necessária para a produção de carboidratos. Este estudo foi realizado em um trecho de Floresta Estacional Semidecidual, no município de Colorado do Oeste, com o objetivo de verificar reduções nas concentrações dos pigmentos fotossintéticos, como indicativo de foto- oxidação. Foram realizados o estabelecimento das parcelas, amostragem e herborização. Os resultados demonstraram maior concentração de clorofila a em relação a clorofila b, maior razão clorofila a/b e clorofilas totais/carotenoides. As espécies florestais apresentaram-se com concentrações de pigmentos fotossintéticos em conformidade com as de espécies que não estão sob excesso de luminosidade.
		\vspace{\onelineskip}
		
		\noindent
		\textbf{Palavras-chave}: Amazônia. Foto-proteção. Pigmentos fotossintéticos.
	\end{resumoumacoluna}
	
	\section*{Introdução}
	
	As clorofilas são pigmentos fotossintéticos com grande atuação na fotossíntese, processo em que a energia solar é transformada pelos vegetais em energia química, fornecendo a energia necessária para o seu desenvolvimento. As clorofilas envolvidas na fotossíntese são principalmente as clorofilas a e b; que atuam juntamente com os pigmentos acessórios.
	
	A fotossíntese pode ser considerada como um dos processos biológicos mais importantes na terra. Porém há fatores que a limitam, como luz, temperatura, salinidade, estruturas da folha, estrutura dos cloroplastos, etc (MATHIAS, 2005). Dentre os fatores limitantes está o excesso de luz que pode inibir a fotossíntese, por causar fotoinibição, a qual é reversível, e a fotoxidação, que é irreversível.
	
	Os cloroplastos sofrem muitos danos decorrentes da fotoxidação (MATHIAS, 2005); que, por sua vez, pode acarretar na degradação das proteínas, sendo necessária sua substituição, para que possam desempenhar eficientemente sua função. Deste modo, a maquinaria proteolítica dos cloroplastos é essencial para o controle de qualidade das proteínas e manutenção do funcionamento desta organela (BORGONOVE, C. M. 2009).
	
	A determinação dos teores de clorofila da folha se faz importante, porque a atividade fotossintética da planta depende em parte da capacidade da folha em absorver luz. Investigar tal processo se faz necessário, porque quando a produção de clorofilas está comprometida, pode ser um indicativo de danos à maquinaria fotossintética (TAIZ e ZEIGER, 2013).
	
	A presença e a abundância dos pigmentos fotossintéticos variam de acordo com a espécie, sendo necessária sua medição para verificar potenciais perdas decorrentes de foto-oxidação. Diante disto, o objetivo deste estudo foi avaliar a concentração dos pigmentos fotossintéticos em angiospermas, de um trecho de Floresta Estacional Semidecidual, e suas relações com parâmetros físicos do ambiente no município de Colorado do Oeste/RO.
	
	\section*{Material e Método}
	
	A área de estudo compreende um trecho de Floresta Estacional Semidecidual, do município de Colorado do Oeste, localizado no cone sul do estado de Rondônia, nesta área foi realizado o estabelecimento das parcelas, determinação de diâmetro à altura do peito (DAP), circunferência à altura do peito (CAP) e altura, coleta de amostras para análise de clorofila a, b e carotenoides totais (LICHTENTHALER e WELLBURN, 1983) em espectrofotômetro de absorção por ultravioleta/visível. Soma-se a essas análises a realização de herborização e determinação da luminosidade, temperatura e umidade, utilizando-se de termo- higrômetro (ITHT2250) e luxímetro (INSTRUTEMP). As análises estatísticas consistiram de análise de correlações; além de estatística descritiva.
	
	\section*{Resultados e Discussão}
	
	Houve maior concentração de clorofila a em relação a clorofila b, maior razão clorofila a/b e clorofilas totais/carotenoides (tabela 1).
	
	\begin{table}[h]
		\centering
		\tiny
		\caption{Estatística descritiva das concentrações de clorofila a, b e carotenoides.}
		\label{my-label}
		\begin{tabular}{cccccc}
			\hline
			\textbf{Variáveis}                    & \textbf{Mínimo} & \textbf{Meios de comunicação} & \textbf{Máximo}  & \textbf{Desvio Padrão} & \textbf{Erro Padrão} \\ \hline
			Clorofila a                  & 0,262  & 1,475                & 2,127   & 0,542         & 0,082       \\
			Clorofila b                  & 0,106  & 0,937                & 1,818   & 0,504         & 0,076       \\
			Clorofila total              & 0,468  & 2,521                & 3,852   & 1,001         & 0,151       \\
			Carotenóides                 & 0,006  & 0,120                & 0,528   & 0,112         & 0,019       \\
			Clorofila a/b                & 0,678  & 2,088                & 13,951  & 2,050         & 0,309       \\
			Clorofila total/carotenóides & 1,783  & 64,191               & 434,212 & 97,467        & 16,715      \\ \hline
			\multicolumn{6}{c}{Todas as colunas em mg.g-1.}                                                     
		\end{tabular}
	\end{table}
	
	As análises de correlação entre os pigmentos fotossintéticos e suas razões indicaram que houve apenas correlação positiva forte entre as concentrações de clorofila a e b (r = 0,79; p = 0,000); o que pode ser um indicativo de proporções equivalentes de irradiância de ondas longas e curtas dentro da floresta. Também foi encontrada correlação negativa regular entre clorofila b e carotenóides (r = -0,41; p = 0,0116); certamente devido à presença de radiação de ondas curtas, indicando áreas sombreadas pontuais, o que pode ter contribuído para a redução da produção de carotenóides devido ao incremento na produção de clorofila b.
	
	É importante destacar que a inexistente correlação entre clorofila a e carotenóides deve-se, provavelmente, a ausência de radiação excessiva no interior da floresta, que apresentava clareiras bem expressivas, além de apresentar uma grande diversidade de espécies; dentre elas, pioneiras, secundárias, cipós, em variados estádios fenológicos (floresta bastante heterogênea).
	
	As concentrações médias de pigmentos fotossintéticos deste estudo apresentaram-se abaixo daquelas de um estudo realizado por Gonzales et al. (2013), onde as concentrações de clorofila a, b e carotenóides foram respectivamente 843, 403 e 339 mg.g-1 em clones de seringueiras sadias. Segundo Hatfield et al. (2008), os teores foliares de clorofila estão estreitamente relacionados com o estresse vegetal e a senescência dos tecidos vegetais, estes processos associados normalmente ocorrem com perda de clorofilas e redução na capacidade fotossintética; o que provavelmente não está ocorrendo na floresta em estudo.
	
	Nenhum pigmento apresentou correlação com os parâmetros biométricos mensurados (DAP, CAP e altura). Além disso, não foi possível estabelecer relação entre parâmetros ambientais e pigmentos fotossintéticos, tendo em vista que os primeiros se mantiveram constantes.
	
	\section*{Conclusões}
	
Os pigmentos fotossintéticos apresentaram-se em concentração adequada com as de plantas que não estão sob excesso de luminosidade; portanto, não há indicativo de perdas foto-oxidativas relacionadas ao conteúdo de carotenoides presentes nas plantas e a intensidade luminosa no local. Os parâmetros biométricos e abióticos não influenciaram as concentrações dos pigmentos.
	
	\sloppy
	\section*{Referências}
	
\noindent BORGONOVE, C. M. Degradação protéica em Cloroplastos. \textbf{Seminários em Genética e Melhoramento de Plantas.} Piracicaba – SP. 2009. 

\noindent GONZALES, G. C.; CATANAEO, C.; FIORI, M. S.; SILVA, S. G.; MISHAN, M. M.; 

\noindent FURTADO, E. L. Pigmentos fotossintéticos em clones de seringueira sob ataque de oídio. \textbf {Ciência Florestal}, v. 23, n. 3, p. 499-506, 2013. 

\noindent HATFIELD, J. L.; GITELSON, A. A.; SCHEPERS, J. S.; WALTHALL, C. L.; 
Application of spectral remote sensing for agronomic decisions. \textbf {Agronomy Journal}, v. 100 (Supplement), p. 117-131, 2008. 

\noindent KLUGE, R. A. \textbf {Fotossintese}. SP. 2007. 

\noindent LICHTENTHALER, H.; WELLBURN, A. Determinations of total carotenoids and chlorophylls a and b of leaf extracts in different solvents. \textbf {Biochemical Society Transactions}, n. 603, p. 591-592, 1983. 

\noindent MATHIAS, M. D. \textbf {fisiosiologia vegetal}, unesp. SP. 2005. 
\noindent TAIZ, L.; ZEIGER, E.; Fisiologia vegetal. 5ª Ed., \textbf {Artmed}. Porto Alegre. 2013.

	
\end{document}