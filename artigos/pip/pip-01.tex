\documentclass[article,12pt,onesidea,4paper,english,brazil]{abntex2}

\usepackage{lmodern, indentfirst, nomencl, color, graphicx, microtype, lipsum}			
\usepackage[T1]{fontenc}		
\usepackage[utf8]{inputenc}		

\setlrmarginsandblock{2cm}{2cm}{*}
\setulmarginsandblock{2cm}{2cm}{*}
\checkandfixthelayout

\setlength{\parindent}{1.3cm}
\setlength{\parskip}{0.2cm}

\SingleSpacing

\begin{document}
	
	\selectlanguage{brazil}
	
	\frenchspacing 
	
	\begin{center}
		\LARGE ANÁLISE SENSORIAL DE BISCOITO DE\\CUPUAÇU COM MEL\footnote{Trabalho realizado dentro das Ciências Agrárias com financiamento do IFRO.}
		
		\normalsize
		Jhennifer Stefany Teles Gonçalves\footnote{Bolsista (DEPESP-IFRO), jhennifer.stg@gmail.com, Campus Ariquemes} 
		Tais Pamela Marcadelli da Silva	\footnote{Bolsista (DEPESP-IFRO), tmarcadelli12@gmail.com, Campus Ariquemes} \\
		Daise dos Santos \footnote{Orientadora, daise.santos@ifro.edu.br, Campus Ariquemes} 
		Antonio Bisconsin-Junior\footnote{Co-orientador, antonio.bisconsin@ifro.edu.br, Campus Ariquemes}
	\end{center}

\noindent O consumidor está se tornando cada vez mais exigente e preocupado com a segurança alimentar, seja nutricional ou microbiológica. No entanto, os biscoitos geralmente são consumidos para satisfazer as necessidades sensoriais, e não nutricionais, portanto, a qualidade sensorial é o fator determinante para a aceitação e preferência do consumidor. O objetivo do presente trabalho foi elaborar um  biscoito de cupuaçu com mel, permitindo agregar mais valor comercial e nutricional ao produto. Três formulações de biscoito de cupuaçu com mel foram preparadas, variando a proporção de mel e açúcar: (biscoito 1) 100\% açúcar; (biscoito 2) 50\% mel e 50\% açúcar; (biscoito 3) 100\% mel. As três formulações de biscoitos foram avaliadas usando o teste de aceitação, a nível laboratorial, sendo os atributos cor, impressão global, aroma, sabor e textura avaliados empregando escala hedônica estruturada de nove pontos (9=gostei extremamente; 5=nem gostei/nem desgostei; 1=desgostei extremamente). Além da aceitação, foi avaliada a intenção de compra de cada biscoito. Foram recrutados 84 consumidores de biscoito no IFRO Campus Ariquemes, sendo 23 funcionários e 61 alunos. Nos atributos aparência e aroma, as três formulações obtiveram a média das notas próximas, 7 e 6,5, respectivamente. Porém nos atributos impressão geral, sabor e textura, o biscoito 1, apresentou médias superiores às demais formulações (p>0,05). Na intenção de compra, 80\% dos consumidores declararam que comprariam o biscoito 1, enquanto que apenas 43 e 39\% comprariam o biscoito 2 e 3, respectivamente. Os resultados demonstram que a formulação do biscoito 1, que não apresenta mel, apenas açúcar, foi a que apresentou maior aceitação entre os consumidores e maior chance de comercialização.

\vspace{\onelineskip}

\noindent
\textbf{Palavras-chave}: Aceitação. Intenção de Compra. \textit{Theobroma grandiflorum.}

\end{document}