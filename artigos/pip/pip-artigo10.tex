\documentclass[article,12pt,onesidea,4paper,english,brazil]{abntex2}

\usepackage{lmodern, indentfirst, nomencl, color, graphicx, microtype, lipsum}			
\usepackage[T1]{fontenc}		
\usepackage[utf8]{inputenc}		
\usepackage{textcomp}
\setlrmarginsandblock{2cm}{2cm}{*}
\setulmarginsandblock{2cm}{2cm}{*}
\checkandfixthelayout

\setlength{\parindent}{1.3cm}
\setlength{\parskip}{0.2cm}

\SingleSpacing

\begin{document}
	
	\selectlanguage{brazil}
	
	\frenchspacing 
	
	\begin{center}
		\LARGE INVENTARIO CENSO QUALI-QUANTITATIVO DA ARBORIZAÇÃO DA AVENIDA JI-PARANÁ NA CIDADE DE JI-PARANÁ-RO
		
		\normalsize
	Jefferson Uere P. da Costa\footnote{IColaborador(a), email, Campus..} 
	Carlos Momo \footnote{Informações do Autor 2.} 
	Guilherme B. deAzevedo\footnote{2Orientador(a), email, Campus.} 
	\end{center}
	
	% resumo em português
	\begin{resumoumacoluna}
		A arborização é um elemento relevante para que cidade proporciona uma melhor qualidade de vida para seus habitantes. No entanto a falta de planejamento e manejo inadequado são bem evidentes nas cidades brasileiras. O objetivo deste trabalho foi verificar a situação atual da arborização da avenida Ji-paraná, no município de Ji-Paraná RO, levantando a quantidade de indivíduos por espécie, analisando a sanidade dos indivíduos e verificando se está ocorrendo interferência com o meio. Para isso foi realizado um inventário censo quali-quantitativo, utilizando uma ficha de campo para anotar as características e os parâmetros, foram anotados a espécie, a sanidade do indivíduo, se ocorre interferências com o meio, intensidade da poda realizada, se à necessidade da poda, altura da primeira bifurcação, e afastamento do indivíduo ao meio fio, e a construção. Foram encontrados 357 indivíduos distribuídas em 15 espécies, no geral com bom estado fitossanitário, duas espécies representando 54,62\% do total encontrado, sendo fícus 33,33\% e o ipê rosa 21,9\%. Não foram encontradas muitas interferências, apesar de mais de 90\% estarem com altura bifurcação abaixo do indicado. Os fícus foi a espécie com maior intensidade de podas, geralmente pesada ou drásticas. E necessário definir programa de podas de manejo com técnicas adequadas para não prejudicar a sanidade do indivíduo e ter uma melhor relação com o meio.
		
		\vspace{\onelineskip}
		
		\noindent
		\textbf{Palavras-chave}: : Arborização urbana, inventario censo quali-quatitativo, diagnostico
	\end{resumoumacoluna}
	
	\section*{Introdução}
	
A arborização urbana e a implantação de jardins ou florestas urbanas, que são formados por conjuntos arbóreos de diferentes origens e que desempenham diferentes papeis melhorando a qualidade de vida. Segundo, Basso e Corrêa (2014), a arborização urbana traz melhorias no microclima, diminuição de poluição do ar, sonora e visual, abrigo para a fauna que vive nas cidades, qualificação de lugares urbanos e sua identidade com as comunidades. Assim torna-se cada vez mais importante a medida que as cidades crescem verticalmente ou horizontalmente.

	O crescimento acelerado do meio urbano interligado ao mal planejamento atribui características insalubres ao meio urbano, apresentando problemas como, excesso de ruído, emissão de poluentes no ar e na água, escassez de recursos energéticos e de água, falta de tratamento adequado de resíduos, alterações no regime de chuvas e de ventos, formação de ilhas de calor, inversão térmica, e aumento do consumo de energia para condicionamento artificial e transporte.
	
	Muitos trabalhos apontam o uso da vegetação arbórea como fator amenizador dos problemas sócias, ecológicos e paisagísticos da cidade, no entanto se arborização for mal planejada, não irá cumprir seus objetivos básicos, podendo causar inúmeros problemas, como a queda de árvores, entupimento de calhas e bueiros, dificultar a circulação de pessoas, quebrar calçadas e conflitar com a rede elétrica, causando sérios acidentes.
	
	O inventário da arborização urbana tem como objetivo geral conhecer o patrimônio arbustivo e arbóreo de uma localidade, obter informações sobre quantidades e qualidades dos recursos florestais e características das áreas sobre as quais as árvores estão crescendo. Tal estudo é fundamental para o planejamento e manejo da arborização, fornecendo informações sobre a necessidade de poda, tratamentos fitossanitários ou remoção e plantios, bem como para definir prioridades de intervenções ao ponto de se verificar os erros e acertos na arborização de uma cidade. Um dos aspectos mais importantes do inventario é quando este é realizado de forma a fornecer uma continua atualização das informações.
	
	A cidade de Ji-Paraná teve seu desenvolvimento acelerado a partir de 1960 com a construção da BR-364, com o êxodo rural do sul do pais, promovido pelo Instituto brasileiro de Reforma Agraria-IBRA, atual INCRA- Instituto Nacional de Reforma Agraria, a acelerou o fluxo de imigrantes vindos do centro-sul (IBGE, 2015). Esse crescimento desordenado não possibilitou o planejamento adequado da arborização de suas ruas e avenidas, onde muitas vezes esta não ocorre ou ocorre de forma inadequada.
	
	Nesse contexto o objetivo geral deste estudo é inventariar a arborização na avenida Ji-Paraná, na cidade de Ji-paraná em Rondônia. E os objetivos específicos são levantar a quantidade de arvores por espécie, analisar a sanidade das árvores e avaliar interferências com o meio. Dessa forma gerar informações que subsidiem futuras ações de manejo.
	\section*{Material e Método}
	Ji-paraná é um município de estado de Rondônia, o segundo mais populoso do estado, como população estimada em de 130.419 habitantes, área da unidade territorial, 6.896,604 km², densidade demográfica, 16,92 hab/km². (IBGE, 2015).
	Localizada a uma latitude 10°53’07” sul e a uma longitude 61°57’06” oeste, estando a uma altitude de 170 metros. Clima predominante equatorial úmido, segundo classificação de Koppen, com temperatura média anual entre 25C° e 36C°. E a precipitação anual varia de 1800mm a 2400mm.
	
	Foi realizado um inventario censo quali-quatitativo, dos indivíduos arbóreos da Av. Ji-paraná, 1,7 km, principal, do bairro Urupá, residencial de classe alta e média. Para anotar os dados será utilizado uma ficha de campo, e as características e parâmetros anotados foram considerado importantes para ter conhecimento da atual situação de qualidade das arvores. Levantar a quantidade de arvores por espécie, analisar a arborização quanto ao espaçamento e condição geral da arvore, avaliar interferências com o meio, rede elétrica, construções e calçadas.
	
	Texto MM.
	
	\section*{Resultados e Discussão}
	
O levantamento da arborização da Avenida Ji-paraná em Ji-Paraná/RO, resultou em 357 indivíduos, divididos em 15 espécies, distribuídas no canteiro central e no calçamento de pedestres. Dentre as espécies encontradas fícus representa 33,33\% e o ipê rosa 21,9\%, do total de indivíduos, restante das espécies ficando abaixo de 10\%, constatando que a distribuição dos indivíduos por espécie se encontra fora da fixa recomendada que é 12 15\%, com duas espécies  representando54,62\%.

Os 277 indivíduos encontradas no canteiro central estão distribuídas em 7 espécies, contendo uma não identificada (Tabela 1). O número total de indivíduos apresenta uma distribuição irregular, sendo que as quatros espécies mais frequentes representam 93,14\% do total. Sendo, FICUS 41,88\%, IPÊ AMARELO 12,27\%, IPÊ ROSA 26,71\%, E PUPUNHA 12,27\%.

A literatura indica que a arborização deve ser heterogênea, indicando que cada espécie não deve ultrapassar 12 a 15\% do total da população. Somente o FICUS e o IPÊ ROSA, ultrapassam essa faixa recomendada, juntos representam mais da metade 68,59\%. Arborização homogênea oferece riscos bem evidente, esteticamente a arborização com uma espécie predominante gera monotonia na paisagem urbana, administrativamente corre o risco de um surto de doença, ou praga especifica que pode dizimar a arborização em pouco tempo. As medidas recomendadas seriam a plantio de outras espécies, e a possível remoção de alguns indivíduos.

Contudo na mesma avenida ou logradouro a uniformização na escolha das espécies utilizadas facilita no manejo, como época de poda, limpeza, entre outros, sendo recomendado utilizar de três a cinco espécies por logradouro. Esta dominância, porém, não pode ser estendida ao bairro e município.

É uma situação recorrente na maioria das cidades brasileiras ter dominância de umas espécies na arborização da cidade, como em Jaboticabal, SP foram realizadas análises em 2002 levantadas 59 espécies sendo que, 6 representaram 85\% do total. (SILVA FILHO, 2002) SANTOS \& TEIXEIRA (1990) constataram esse fato ao fazerem o levantamento no Bairro Centro de Santa Maria onde apenas 5 espécies abrangem mais de 70\% de sua arborização.
	
	\section*{Conclusões}
	
	A arborização da avenida Ji-paraná apresentou boa riqueza de espécie, porém com distribuição não uniforme, poucas espécies representam mais da metade do total de indivíduos presentes. De modo geral os indivíduos encontram-se em boas condições fitossanitárias, com exceção do FICUS que apresentou quase metade dos indiviso com estado sanitário prejudicado, não sendo aconselhável sua implantação no canteiro central.
	
	Apesar de maior parte dos indivíduos não apresentaram poda de formação, com altura da primeira bifurcação menor que o indicado, houve pouca interferência
	
	com o meio, no entanto necessitando de poda de contenção lateral, principalmente nos FICUS.
	
	O afastamento das arvores em relação ao meio fio se demostrou adequado, evitando interferência no tráfego, já em relação a construção se demostrou próximo, com espécies de grande porte como PATA DE VACA e SIBIPIRUNA interferindo nas construções.
	
Faz se necessário um programa de manejo de podas, definindo técnicas e critérios adequados, para tal manutenção, garantindo assim uma convivência harmoniosa entre as arvores, pessoas e edificações.	
	
	\section*{Referências}
	\noindent BIONDI, D. Arborização urbana: aplicada à educação ambiental nas escolas. Curitiba, 2008. 120 p.
	
	\noindent BORJA, P. C. Avaliação da qualidade ambiental urbana - uma contribuição metodológica. 281 f. Dissertação (Mestrado em Arquitetura e Urbanismo) - Faculdade de Arquitetura, Universidade Federal da Bahia, Salvador, 1997.
	
	\noindent 
	CORMIER, Nathaniel S.; PELLEGRINO, Paulo Renato Mesquita. Infraestrutura verde: uma estratégia paisagística para a água urbana. Paisagem e Ambiente: ensaios. FAUUSP, São Paulo, n. 25, 2008, p. 127-142.
	
	\noindent GONÇALVES, W.; PAIVA, H. N. Silvicultura Urbana: implantação e manejo. Viçosa, MG: Aprenda Fácil, 2006. 201p. (Coleção Jardinagem e Paisagismo, Série Arborização Urbana, v.4).
	
	
	\noindent INSTITUTO BRASILEIRO DE GEOGRAFIA E ESTATÍSTICA (IBGE). Informações
	completas Ji-Paraná RO. Disponível em: http://www.cidades.ibge.gov.br/xtras/perfil.php?lang=\&codmun=110012\&search=||infogr\%E 1ficos:-informa\%E7\%F5es-completas>. Acesso em 20 de novembro de 2015.
	
	\noindent LIMA, D. C. R. Monitoramento e desempenho da vegetação no conforto térmico em espaços livres urbanos: o caso das praças de Maringá/ PR. 170 f. Dissertação (Mestrado em Engenharia Urbana) – Departamento de Engenharia Civil, Universidade Estadual de Maringá, Maringá, 2009
	
	\noindent MARTINI, A. Estudo fenológico em árvores de rua. In: BIONDI, D.; LIMA NETO, E. M. de (Org.). Pesquisas em arborização de ruas. Curitiba, 2011. p. 29 - 48.
	
\noindent MARTINI, A. Microclima e conforto térmico proporcionado pelas árvores de rua na cidade de Curitiba. Universidade Federal do Paraná, Curitiba-PR, 2013.

\noindent MARTINI, A.; BIONDI, D.; BATISTA, A. C.; LIMA NETO, E. M. de Microclimae
conforto térmico de um fragmento florestal na cidade de Curitiba - PR, Brasil.In:

\noindent CONGRESO FORESTAL LATINOAMERICANO, 5., 2011, Lima. Anais.... Lima: [s.n.],
2011. Não paginado.

\noindent  MASCARÓ, L.; MASCARÓ, J. J. Ambiência urbana. 3. ed. Porto Alegre: +4 Editora, 2009. 200 p.

\noindent KRAN, F. S. Qualidade de vida na cidade de Palmas – TO: uma análise através de indicadores habitacionais e ambientais urbanos. 142 f. Dissertação (Mestrado em Ciências do Ambiente) - Universidade Federal do Tocantins, Palmas, 2005.

\noindent  PASSERINO, L. C. M. Zoneamento da qualidade do ambiente urbano: um estudo de caso em Balneário Camboriú - SC. 92 f. Dissertação (Mestrado em Engenharia de Produção) – Universidade Federal de Santa Catarina, Florianópolis, 2004.

\noindent  SANTOS, N.R.Z. dos; TEIXEIRA, I.F. Levantamento quantitativo e qualitativo da arborização do Bairro Centro da cidade de Santa Maria-RS. In: ENCONTRO NACIONAL SOBRE ARBORIZAÇÃO URBANA, 1990, Curitiba. Anais...Curitiba: FUPEF, 1990. 368 p. p. 263-276

\noindent SANTOS, N. R. Z.; TEIXEIRA, J. F. Arborização de vias públicas: ambiente x vegetação.
Santa Cruz do Sul: Instituto Souza Cruz, 2001. 135 p.

	
\end{document}