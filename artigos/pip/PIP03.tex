\documentclass[article,12pt,onesidea,4paper,english,brazil]{abntex2}

\usepackage{lmodern, indentfirst, nomencl, color, graphicx, microtype, lipsum}			
\usepackage[T1]{fontenc}		
\usepackage[utf8]{inputenc}		

\setlrmarginsandblock{2cm}{2cm}{*}
\setulmarginsandblock{2cm}{2cm}{*}
\checkandfixthelayout

\setlength{\parindent}{1.3cm}
\setlength{\parskip}{0.2cm}

\SingleSpacing

\begin{document}
	
	\selectlanguage{brazil}
	
	\frenchspacing 
	
	\begin{center}
		\LARGE APLICANDO A TEORIA NA PRÁTICA – UM ESTUDO DE CASO NA SECRETARIA MUNICIPAL DE SAÚDE DE PORTO VELHO - RO\footnote{Trabalho realizado dentro da área de Conhecimento CNPq/CAPES: Ciências Sociais Aplicadas, com apoio do IFRO- Campus Porto Velho Zona Norte.}
	
	\normalsize
	Maria Genilda Batista da Silva\footnote{Aluna pesquisadora do Curso de Tecnologia em Gestão Pública, nilda\_vandecy@hotmail.com, Campus Porto Velho Zona Norte.} 
	Reuria da Silva Moreira\footnote{Aluna pesquisadora do Curso de Tecnologia em Gestão Pública reuria.moreira@ifro.edu.br, IFRO – Campus Porto Velho Zona Norte.} \\
	Adonias Soares da S.Júnior\footnote{Orientador (a), adonias.silva@ifro.edu.br, Campus Porto Velho Zona Norte.} 
	Sandra SenaReis\footnote{Aluna pesquisadora do Curso de Tecnologia em Gestão Pública, sandrasenareis14@gmail.com, Campus.} 
	\end{center}
	
	% resumo em português
\begin{resumoumacoluna}
	O projeto foi realizado em campo onde busca enfatizar a teoria na prática observando a execução administrativa aplicada por eles. As metodologias e  métodos utilizados foram de levantamento bibliográfico, entrevista, gravações e anotações. Com a crescente demanda na distribuição dos remédios dentro do Município esta pesquisa objetivou conhecer o sistema de logística e administrativa utilizado no almoxarifado de medicamentos da Secretaria Municipal de Saúde- SEMUSA. A falta de farmacêuticos e gestores por indicação politica são uma das limitações para o bom funcionamento. Uma das soluções seria um sistema de informação interligando as unidades com o almoxarifado, e a capacitação dos profissionais na área de farmácia.
	
	\vspace{\onelineskip}
	
	\noindent
	\textbf{Palavras-chave}: palavra 1. palavra 2. palavra 3.
\end{resumoumacoluna}

\section*{Introdução}

O projeto teve como objetivo investigar as técnicas de logística utilizadas na organização material do almoxarifado daquele órgão, bem como propor soluções para o aprimoramento do serviço público prestado. Segundo Christopher (1993), logística é o processo de gerenciamento estratégico da compra, do transporte e da armazenagem de matérias primas, partes e produtos acabados, ou seja, a missão da logística é dispor o produto/serviço certo, no tempo certo, no lugar certo e nas condições desejadas, ao mesmo tempo em que fornece uma maior contribuição à organização. Assim Matias-Pereira (2007) define bem a organização pública como centros de competências instituídos para o desempenho de funções estatais,  através de seus agentes, cuja atuação é imputável à pessoa jurídica a que pertencem. Não tem personalidade jurídica e nem vontade própria. Logo, como “centro de competência” a organização pública deve promover os interesses públicos com primazia e qualificação da equipe que, por sua vez, terá às mãos equipamentos adequados para utilizar e desenvolver as tarefas públicas com agilidade e competência. De acordo com suas atribuições, no inciso V - garantir infraestrutura necessária ao funcionamento das Unidades Básicas de Saúde, dotando-as de recursos materiais, equipamentos e insumos suficientes para o conjunto de ações propostas. Foi realizado na Secretaria Municipal de Saúde- SEMUSA, órgão público do município de Porto Velho – RO. Fruto das ações propostas na disciplina Gestão Patrimonial e Logística do curso superior de Tecnologia em Gestão Publica do Campus Porto Velho Zona Norte. O problema encontrado não é na distribuição de medicamentos, e sim nas unidades de saúde, que não repassam informações corretas para a efetividade dos serviços. A falta de farmacêuticos e gestores por indicação politica são uma das limitações para o bom funcionamento. Uma das soluções seria um sistema de informação interligando as unidades com o almoxarifado, e a capacitação dos profissionais na área de  farmácia.

\section*{Material e Método}

A pesquisa realizada foi em campo onde busca enfatizar a teoria na prática observando a execução administrativa aplicada por eles. As metodologias e  métodos utilizados foram de levantamento bibliográfico, entrevista, gravações e anotações. Dar-se-á como resultado da pesquisa que o almoxarifado de medicamentos da Secretaria possui uma estrutura física e o corpo operacional a serem melhorados.Foi abordado sobre a informação geral,quadro funcional,área de estocagem, boas práticas de estocagem e distribuição de medicamentos através dos mapas elaborados pelos responsáveis das unidades de Saúde.



\section*{Resultados e Discussão}

Foi delimitado apenas um almoxarifado, devido aos transtornos de mudanças os Departamentos ainda estão se organizando no novo endereço, pois estão lá apenas há sete meses, a equipe de pesquisa optou por realizar a visita técnica no almoxarifado de medicamentos, onde foi realizada uma entrevista com os farmacêuticos responsáveis pelo setor. O ambiente é bem estruturado, na entrada há uma recepção onde os Diretores das Unidades de Saúde fazem pedidos de medicamentos e entrega dos mapas. A equipe é composta por seis farmacêuticos que atendem diariamente as unidades de saúde do Município, e atualmente existem:

23 Unidades de Saúde Básica; 06 Unidades de Urgência e Emergência; 02 Unidades Prisionais; 11 Unidades de Saúde Fluvial e 21 Unidades de Saúde Distrital.

Os mapas são uma espécie de relatório de controle. Há uma sala de administração onde são montados os processos de pesquisa de preço e discriminação dos produtos tudo detalhado para enviar a uma equipe de montagem de processo de compras na SEMAD Secretaria Municipal de Administração, o Governo paga por habitante em medicamentos R\$10,00. Estes são comprados de acordo com o numero de habitantes, e é feito por meio de pregão eletrônico, com prazo de entrega e vistoria dos lotes, após chegada existe uma conferência manual de item por item. Na área do estoque de medicamentos, só entram pessoas autorizadas, pois o acesso é restrito. Com o resultado verificou-se que o problema esta nas unidades de saúde, que não repassam informações corretas para efetividade dos serviços.


\section*{Conclusões}

Portanto foi possível verificar que a falta de farmacêuticos e gestores por indicação politica são uma das limitações para o bom funcionamento do órgão. Uma das soluções seria um sistema de informação interligando as unidades como almoxarifado, a capacitação dos profissionais na área de farmácia e a profissionalização do gestor das unidades de saúde do município, para um gerenciamento efetivo do estoque farmacêutico Secretaria Municipal de Saúde.


\section*{Instituição de Fomento}

Instituto Federal de Educação, Ciência e Tecnologia de Rondônia – IFRO.

\sloppy

\section*{Referências}

\noindent MATIAS-PEREIRA, José. Curso de Administração Pública: foco nas instituições e ações governamentais. São Paulo: Atlas, 2009.

\noindent CHRISTOPHER, Martin. Logísticos e gerenciamento da cadeia de suprimentos: estratégias para redução de custos e melhoria de serviços. São Paulo: Pioneira, 1997.

\noindent CALIXTO FABIANO, Logística: enfoque prático – 2. Ed. São Paulo: Saraiva 2014. Disponível site: http://www.portovelho.ro.gov.br/artigo/semusa-secretaria-municipal-de-saude


\end{document}
