\documentclass[article,12pt,onesidea,4paper,english,brazil]{abntex2}

\usepackage{lmodern, indentfirst, nomencl, color, graphicx, microtype, lipsum}			
\usepackage[T1]{fontenc}		
\usepackage[utf8]{inputenc}		

\setlrmarginsandblock{2cm}{2cm}{*}
\setulmarginsandblock{2cm}{2cm}{*}
\checkandfixthelayout

\setlength{\parindent}{1.3cm}
\setlength{\parskip}{0.2cm}

\SingleSpacing

\begin{document}
	
	\selectlanguage{brazil}
	
	\frenchspacing 
	
	\begin{center}
		\LARGE INFORMÁTICA E SOCIEDADE: \\UMA ANÁLISE DO PPC DO CURSO TÉCNICO DE INFORMÁTICA DO IFRO – CAMPUS PORTO VELHO
		
		\normalsize
	Lucas Matheus Ferreira Martins\footnote{Bolsista PIBIC-EM, Lucas.pvh.martins@gmail.com, Campus Porto Velho Calama.} 
	Xênia de Castro Barbosa\footnote{Orientadora, xenia.castro@ifro.edu.br, Campus Porto Velho Calama.} \\
		Gisele Caroline Nascimento dos Santos\footnote{Colaboradora,gisele.santos@ifro.edu.br, Reitoria - IFRO.}  
	\end{center}
	
	% resumo em português
	\begin{resumoumacoluna}
		O texto visa apresentar os resultados da pesquisa desenvolvida na modalidade Iniciação Científica Júnior (ICJ) a partir do projeto de pesquisa intitulado: “Informática e sociedade: um estudo da efetividade do curso técnico de informática do IFRO”, especificamente o plano de trabalho que tratou da análise do Projeto Pedagógico do Curso (PPC) Técnico de Informática Integrado ao Ensino Médio do Campus Porto Velho Calama. O estudo foi desenvolvido com base no método Histórico (RÜSEN, 2001) e nos procedimentos da crítica hermenêutica, considerando o documento em sua forma, conteúdo e contexto de produção. O estudo tem como propósito contribuir para uma remodelagem do curso, superando lacunas presentes no PPC e tornando-o mais adequado ao tempo histórico e aos processos sociais manifestos em Porto Velho na atualidade.
		
		\vspace{\onelineskip}
		
		\noindent
		\textbf{Palavras-chave}: Projeto pedagógico. Educação. História.
		
		
	\end{resumoumacoluna}
	
	\section*{Introdução}
	
	A presente pesquisa foi desenvolvida a partir do plano de trabalho que está vinculado ao projeto de pesquisa “Informática e sociedade: um estudo da efetividade do curso técnico de informática do IFRO”, que teve dentre seus objetivos promover a análise do Projeto Pedagógico do Curso Técnico de Informática Integrado ao Ensino Médio à luz das representações e expectativas sociais registradas por meio das entrevistas produzidas durante a pesquisa.
	
	O PPC de um curso é documento norteador das práticas pedagógicas e administrativas a ele relacionadas. Um instrumento que mobiliza docentes, técnicos e gestores na construção de itinerários formativos significativos e relevantes para a vida em social, profissional e o exercício da cidadania. Entendemos que esse documento não é neutro, reflete disputas de interesses variados, tanto na escala institucional quanto na nacional, e semelhante ao que ocorre ao processo educativo,de modo geral, ele é mediado pelo contexto sociocultural, pelas condições em que se efetiva os processos de ensino e de aprendizagem e pelos aspectos organizacionais conforme Barroso (2006) e Dourado(2007).
	
	\section*{Material e Método}
	
	O método que deu suporte para este estudo foi método Histórico (RÜSEN, 2001), pautado nos procedimentos da crítica hermenêutica e heurística. Primeiramente efetuou-se descrição e crítica do documento do ponto de vista formal, identificando os elementos que o constituem e a forma como se relacionam, para em seguida discutir seus conteúdos e contexto de produção, evidenciando fatores históricos, econômicos e sociais.
	
	As expectativas dos estudantes em relação ao curso, a avaliação que fazem do curso, as dificuldades apresentadas e o nível de conhecimento sobre o PPC foram registradas a partir de 56 (cinquenta e seis) entrevistas realizadas com estudantes do curso Técnico de Informática do Campus Calama, nas suas diversas turmas e turnos (matutino evespertino).  
	
	A análise dos dados foi feita a partir da literatura educacional e das políticas públicas e procurou-se cotejar as expectativas e avaliações dos estudantes com as informações expressas do documento, destacando pontos que precisariam se revistos para atendimento das expectativas e das novas dinâmicas que perpassam o espaço de Porto Velho.
	
	\section*{Resultados e Discussão}
	
	Na impossibilidade de apresentar o conjunto das análises realizadas, em função das limitações de espaço, apresentaremos tão somente os resultados e discussões sobre os elementos formais que constituem o PPC em análise.
	
	O PPC apresenta a fundamentação social, os objetivos, diretrizes e ementas que compõem o curso “Técnico de Informática”. Está organizado em 8 partes que se estendem ao longo de 70 páginas. Foi aprovado ad referendum no Conselho Superior do Instituto Federal de Educação, Ciência e Tecnologia de Rondônia pela Resolução n. 40/2010, e na memória de servidores que já estavam no Campus à época, não houve debate sobre ele e nem mesmo construção coletiva, ficando os docentes alijados de sua elaboração.
	
	O curso Técnico de Informática Integrado ao Ensino Médio foi planejado, em PPC para ocorrer na modalidade presencial, integrado ao Ensino Médio, ou seja, com componentes curriculares do núcleo comum e do núcleo específico, que visam habilitar tanto para o exercício da profissão de técnico em Informática quanto para o exercício da cidadania e o prosseguimento dos estudos, em nível superior.
	
	A carga horária do curso é de 3.732 horas, a forma de ingresso é mediante processo seletivo anual, podendo haver uma ou mais chamadas, até o completo preenchimento do número de vagas. A oferta anual é de 80 vagas, sendo distribuídas 40 no turno matutino e 40 no turno vespertino. O documento assegura aos ingressantes o prazo mínimo de 04 anos e máximo de 08 anos para a conclusão do curso.
	
	O Projeto Pedagógico do Curso postula que:
	
	\begin{citacao}
	Ao implantar programas de educação básica e de qualificação específica, contribui-se consideravelmente para o aumento da empregabilidade dos trabalhadores. Com isso, a própria possibilidade de inserção e reinserção  da força de trabalho é ampliada. (IFRO, 2010, p.10)
	\end{citacao}

Sugere ainda que vivemos uma“nova realidade”,com complexidades próprias que afetam o acesso ao trabalho e emprego, comprometendo a capacidade de subsistência do trabalhador. No tópico “Justificativa” expressa tom de adequação do estudante-trabalhador à realidade.Não propõem ensinara pensar ou a questionara ordem estabelecida,lutar para alterá-la,nem tampouco tornar-se empreendedor de ações capazes de gerar novas oportunidades de trabalho e renda.O objetivo geral expresso no PPC é oferecer habilitação técnica de nível médio,que capacite para a atuação profissional no âmbito da programação em informática.	Ao	comparar	esse	objetivo	com	as	respostas	dos	egressos entrevistados, percebe-se que esse objetivo tem sido alcançado,embora egressos também tenham atuado na área de Redes,e em outras áreas,inclusive o comércio
e o voluntariado.

	\section*{Conclusões}
	
Os Objetivos específicos expressos no documento são:
	\begin{enumerate}[label=(\alph*)]
		\item Formar profissionais com competência para: utilizar ambientes de desenvolvimento de sistemas, os próprios sistemas operacionais e os bancos de dados; realizar testes de programas de computador, registrando as análises e refinamento dos resultados; executar manutenção de programas de computadorimplantados;
		
		\item Desenvolver pesquisas, testes, produção e adaptação de tecnologias apropriadas, para aplicação em diversos setores da gestãoempresarial;
		
		\item Trabalhar técnicas para operacionalização de computadores, instalação e desinstalação de hardwares e softwares, e ao mesmo tempo trabalhar estratégias de gerenciamento e supervisão de sistemas de informações, redes de computador e plantas industriais automatizadas pelas tecnologias deinformática;
		
		\item Desenvolver um processo de ensino e aprendizagem em que seja possível interpretar as necessidades do usuário, especificar  adequadamente equipamentos e/ou serviços, instalar e manter sistemas conforme padrões de qualidade aceitáveis e utilizar programas e equipamentoscomputacionais;
		
		\item Oferecer um processo de ensino e aprendizagem que auxilie na construção da autonomia do aluno para a sua vida pessoal e cidadã. (IFRO, 2010).
	\end{enumerate}

Assim, notamos objetivos referentes ao que se espera que o aluno seja capaz de fazer (as habilidades, competências e atitudes), bem como os objetivos institucionais com o curso.

Embora não se possa afirmar que o curso prevê uma metodologia centrada no aluno, e que seja construtivista, o projeto de curso destaca o papel do aluno no processo de aprendizagem e prevê atividades nos quais ele pode exercer liderança e desenvolver sua criatividade, bem como solucionar problemas, como as atividades de pesquisa e extensão. Não se pode afirmar, todavia, que os docentes estejam atuando com tirania sobre os alunos, e que seja negativo repassar informações e conhecimentos construídos por professores de sólida formação técnica e intelectual; o que se questiona são as limitações do diálogo, o pouco valor atribuído às experiências devida dos estudante se os casos em que a informação é repassada sem favorecer a aprendizagem, ou seja, sem dar condições para que o aluno se aproprie dela, transforme, correlacione, amplie e torne-a significativa em sua vida.

Um ponto positivo no PPC é a tentativa de integrar os conteúdos acadêmicos com o mundo do trabalho, o que é feito de modo específico durante o estágio e a oferta das disciplinas técnicas, e de modo geral nas disciplinas de História, Sociologia e Filosofia.
	\section*{Considerações}
	
	O estudo apresenta como resultado a necessidade de revisão do PPC do curso técnico de Informática do Campus Calama/IFRO, de modo a considerar a situação do trabalho e do mercado de trabalho, especialmente na escala local e regional, e incorporar de modo efetivo o paradigma da politecnia (SAVIANI, 1989), com fito de oferta de uma educação mais abrangente e menos fragmentadora do humano.
	\section*{Instituição de Fomento}
	
	FAPERO
	
	\section*{Referências}
	
	\noindent BARROSO, J. Regulação das políticas públicas de educação: espaços, dinâmicas e atores. Lisboa: Educa, 2006.
	
	\noindent DOURADO, L. F. Políticas e gestão da educação básica no Brasil: limites e perspectivas. Educ. Soc., Campinas, vol. 28, n. 100 - Especial, p. 921-946, out. 2007.
	FREIRE, P. Pedagogia do oprimido. 1.ed. Rio de Janeiro: Paz e Terra 1974.
	
	\noindent IFRO. Instituto Federal de Educação, Ciência e Tecnologia de Rondônia. Projeto Pedagógico do Curso Técnico de Informática. Porto Velho, 2010.
	
	\noindent SAVIANI, D. Sobre a concepção de politecnia. Rio de Janeiro: Ministério da Saúde,1989.
	
	
	
	
\end{document}