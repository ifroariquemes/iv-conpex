\documentclass[article,12pt,onesidea,4paper,english,brazil]{abntex2}

\usepackage{lmodern, indentfirst, nomencl, color, graphicx, microtype, lipsum}			
\usepackage[T1]{fontenc}		
\usepackage[utf8]{inputenc}		
\usepackage{textcomp}
\setlrmarginsandblock{2cm}{2cm}{*}
\setulmarginsandblock{2cm}{2cm}{*}
\checkandfixthelayout

\setlength{\parindent}{1.3cm}
\setlength{\parskip}{0.2cm}

\SingleSpacing

\begin{document}
	
	\selectlanguage{brazil}
	
	\frenchspacing 
	
	\begin{center}
		\LARGE IDENTIDADE CULTURAL
		
		\normalsize
		Alana Beatriz Casarotto
	Kamila Ambrozio Rodrigues
		Thalita Kathiusk Xavier Cardoso 
	
	\end{center}
	


	
	\section*{Introdução}
	

A identidade cultural,seja do ambiente,de um grupo de pessoas ou até mesmo de um país,acaba sendo literalmente a identidade deste de termina do povo,ou seja,a partir dela pode se observar/ter o conhecimento do modo em que vivem, onde moram, como se alimentam,do que se alimentam,o que costumam fazer,enfim,a partir da identidade cultural é possível saber a culturade um determinado lugar um povo.

Essa identidade pode variar muito,por exemplo:dentro de um país como o Brasil,pode se encontrar/observar diversos costumes, diversas maneiras de viver, logo, diversas identidades culturais.Essa variação de identidades pode ocorrer por diversos motivos,talvez pelo fato de que(no caso do Brasil)somos divididos em regiões e consequentemente cada região tem sua característica; sejam climáticas, geográficas, em relação a distribuição populacional,etc.E essas características do local influenciam muito na formação de sua cultura,como na alimentação;existem lugares que possuem mais recursos(climático sou geográficos)que favor e cemo cultivo de um determinada planta,fazendo assim com que se tenha grande ocorrência dessa cultivar,e a partir daí vários produtos podem vir;cosméticos, alimentos, produto base de um objeto, formando uma cultura específica do local, se diferenciando do outros,ou seja,se forma a identidade cultural de um local.
	
	\section*{Material e Método}
	
O trabalho foi desenvolvido a partir das pesquisas bibliográfica se documentais,estudos temáticos e teóricos,como objetivo de com poros conteúdos na disciplina de Sociologia, ministradas pelo professor Jaridson Costa,no campus Cacoal,como alunos do 2ºano,do curso Técnico em Agropecuária Integrado ao Ensino Médio,do Instituto Federal de Educação, Ciência e Tecnologia de Rondônia IFRO .A base teórica do trabalho foi realizada a partir de pesquisas bibliográficas, documentais, e livros didáticos como requisito avaliativo para o 3° bimestre.
	
	\section*{Resultados e Discussão}
	
A identidade cultural é como se fosse um grupo vivo de relações,tanto sócias,como patrimônios simbólicos.Muitas vezes é a partir de laque se estabelece valores nos membros de uma sociedade.A identidade cultural é um conceito que vai mudando frequentemente,ou seja,de trânsito intenso e que possui uma certa complexidade.Se pode compreendera formação de uma identidade em manifestações que podem conter uma vasta quantidade de situações que vão da fala,a té a participação em certos eventos.

No caso do lugar onde vivemos,no geral,podemos observar que não temos uma cultura definida, ou seja, nossa cultura está sempre em constante transformação, adquirindo e deixando diversos “rastros”. Até mesmo dentro do nosso país, pode-se observar essas
variações, algumas regiões com hábitos (sejam alimentares, musicais entre outros) que diferenciam-se uns dos outros.

Falando no geral,podemos dar com o exemplo da identidade cultural o Brasil,podemos utilizar o carnaval,que é uma festa típica brasileira e que acontece anualmente,isso é uma característica peculiar do nosso país.O futebol como esporte nacional e o samba como referência na música(DaMatta,1997)

Logo,podemos entender que identidade cultural seria as características de um lugar de um povo e seus hábitos.É que,a nossa identidade é muito vasta e que sempre está se inovando.

	
	\section*{Conclusões}
	
Em relação a o texto a borda do podemos concluir que a identidade cultural é influente para o que cultivamos hoje,em relação a cultura.O fato determos costumes distintos de outros países nos identifica com uma maneira particular,com uma cultura própria formada a partir da formação da nossa identidade como povo brasileiro.

Os hábito se costumes brasileiros é a função de costumes e crenças de imigrantes que colonizaram nossa terra,mas não são só as pessoas que constitui a identidade cultural,a cultura também sofre influências do ambiente para que possamos sobreviver e por isso a identidade cultural está em constante transformação.
	
	\section*{Referências}
	
	\noindent	A Noção de Culturanas Ciências Sociais-RegysCuche(1997)
	
\end{document}