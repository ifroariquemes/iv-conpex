\documentclass[article,12pt,onesidea,4paper,english,brazil]{abntex2}

\usepackage{lmodern, indentfirst, nomencl, color, graphicx, microtype, lipsum,textcomp}			
\usepackage[T1]{fontenc}		
\usepackage[utf8]{inputenc}		

\setlrmarginsandblock{2cm}{2cm}{*}
\setulmarginsandblock{2cm}{2cm}{*}
\checkandfixthelayout

\setlength{\parindent}{1.3cm}
\setlength{\parskip}{0.2cm}

\SingleSpacing

\begin{document}
	
	\selectlanguage{brazil}
	
	\frenchspacing 
	
	\begin{center}
		\LARGE POLÍTICAS PÚBLICAS DE MOBILIDADE URBANA: A REESTRUTURAÇÃO POLÍTICA E ECONÔMICA E OS IMPACTOS SOCIAIS E TERRITORIAIS\footnote{Trabalho realizado dentro da área de Políticas Públicas com financiamento do IFRO Campus Porto Velho Zona Norte.}
		
		\normalsize
	Daiana Cavalcante Gomes\footnote{Bolsista PIBIC, daianasabina@gmail.com, Campus Porto Velho Zona Norte.} 
	Jenerson Queiroz Lima Duarte\footnote {Colaborador,Jenerson Queiroz Lima Duarte, jenersonduartee@gmail.com, Campus Porto Velho Zona Norte.} 
	Maria Beatriz Souza Pereira\footnote{Colaboradora, Maria Beatriz Souza Pereira, mbe.pereira@gmail.com.br, Campus Porto Velho Zona
		Norte.} 
	Marcilei Serafim Germano\footnote{Orientador, Marcilei Serafim Germano,marcilei.germano@ifro.edu.br, Campus Cacoal.} 
	\end{center}
	
	% resumo em português
	\begin{resumoumacoluna}
	A presente pesquisa tem como tema as políticas públicas de mobilidade urbana: a reestruturação política e econômica e os impactos sociais e territoriais sobre a população, fazendo um recorte no estudo na perspectiva de analisar as relações entre as políticas públicas de mobilidade urbana e os impactos sociais e territoriais destas quanto ao acesso aos serviços urbanos na cidade de Porto Velho
	– Rondônia. Foi aplicado aos acadêmicos do Campus Porto Velho Zona Norte através de questionário eletrônico enviado aos alunos através dos e-mails. Como resultado obtido através das respostas dos e-mails, as pessoas se propõem as mudanças para a melhoria da mobilidade urbana.
		
		\vspace{\onelineskip}
		
		\noindent
		\textbf{Palavras-chave}: Crise. Políticas Públicas. Mobilidade Urbana.
	\end{resumoumacoluna}
	
	\textual
	
	\section*{Introdução}
	
	Os problemas acumulados nas cidades hoje ganham crescente relevância social e econômica, porém permanece órfã de interesses políticos. Observamos que as políticas públicas urbanas quando planejadas na perspectiva intra urbanas, setoriais e locais continuam a replicarem os problemas amplamente conhecidos. A mobilidade urbana compreende todos os meios transportes, não apenas o transporte público. Este presente estudo buscou conhecer a qualidade da mobilidade dos acadêmicos quanto aos deslocamentos pela cidade.
	
	\section*{Material e Método}
	
	Pesquisa bibliográfica e análise documental sobre as formulações de políticas públicas de mobilidade urbana e os impactos destas nos serviços prestados à população, com a finalidade de compreender os elementos históricos e sociais, fundamentais para o objetivo da pesquisa, sistematizando elementos teóricos relevantes ao tema. A metodologia empregada nesta pesquisa consiste nos seguintes procedimentos: Revisão bibliográfica; análise documental e a pesquisa de campo: aplicação de questionário. A amostra da pesquisa consistiu em questionário enviado aos alunos do Campus Porto Velho Zona Norte através de e-mails.
	
	\section*{Resultados e Discussão}
	
	Como resposta aos questionários a pesquisa evidenciou que os participantes se propõem a mudança na rotina para a melhora da mobilidade. Como exemplo podemos citar que, 47,7\% dos participantes se propuseram a usar o transporte público caso esse oferecesse mais qualidade. Entretanto as pessoas buscam por qualidade de vida, e buscam outros meios para alcança-la, alegam que o transporte público é inseguro (55,6\%) é ineficiente (52,6\%), e sentem-se inseguros enquanto pedestres (73,7). Diante do exposto fica claro que o gargalo do trânsito no horário de pico, na cidade de Porto Velho, dá-se por falta de incentivo à qualidade, a segurança, e, eficiência do transporte público. O poder público ao não cumprir com a efetivação dos direitos sociais tem feito uso da “teoria da Reserva do Possível” alegando que os recursos financeiros são insuficientes.
	
	\section*{Conclusões}
	
	Como podemos perceber com as informações supracitadas os cidadãos tem conhecimento da problemática apresentada no questionário. Dentre as respostas (47,7\%) se propõe a usar o transporte público se fosse mais adequado e oferecesse mais qualidade aos usuários, e, destes quando questionados acerca da possibilidade de comprar um carro/moto e abrir mão do transporte público 82,4\% disseram que sim. As pessoas buscam por qualidade de vida, e buscam outros meios para alcança-la, alegam que o transporte público é inseguro (55,6\%) eineficiente (52,6\%), e sentem-se inseguros enquanto pedestres (73,7). Quando questionados sobre a qualidade estrutural dos pontos de ônibus apontam que são péssimas (73,7\%), e, que passam mais de meia hora esperando pelo ônibus (42,1\%).
	
	O governo há muito não tem proposto medidas que supram a necessidade da população, sabido que os direitos sociais não são correspondidos da forma como deveriam ser viabilizados pela administração pública, o Estado, para possibilitar uma melhor condição de vida e promover igualdade entre os cidadãos. Em vista a essa vertente, o poder público ao não cumprir com a efetivação dos direitos sociais tem feito uso da “teoria da Reserva do Possível” alegando que os recursos financeiros são insuficientes. Conforme diz Andreas Krell apud Daniel Sarmento (2008, p.570), esta teoria seria “fruto de um Direito Constitucional equivocado”, visto que nos países de terceiro mundo, nosso caso, as necessidades básicas da sociedade não são atendidas satisfatoriamente, comprometendo a real aplicabilidade da teoria já que nos países de primeiro mundo ela não corresponde ao mínimo essencial.
	
	\section*{Instituição de Fomento}
	
	O Instituto Federal de Educação, Ciência e Tecnologia de Rondônia, Campus Porto Velho Zona Norte, possibilitou a pesquisa ao aprová-la em edital, a saber, N°19 do mês de novembro do ano de 2014.
	
	\section*{Referências}
	
	\sloppy
	
	\noindent ARRAIS, Tadeu, PINTO, José. Integrar para segregar: uma análise comparativa do tecido urbano-regional de Goiânia e Brasília. Diez años de cambios en el Mundo, en la Geografía y en las Ciencias Sociales, 1999-2008. Actas del X Coloquio Internacional de Geocrítica, Universidad de Barcelona, 26-30 de mayo de 2008. Disponível em: <http://www.ub.es/geocrit/-xcol/307.htm>. Acesso em: fevereiro de 2015.
	
	\noindent ÁVILA, Kellen Cristina de Andrade. Teoria da Reserva do Possível. 2013. Disponível em: <https://jus.com.br/artigos/24062/teoria-da-reserva-do-possivel>. Acesso em: agosto de 2016.
	
	\noindent BRASIL. Casa Civil. Subchefia para Assuntos Jurídicos. Lei n. 12.587, de 3 de janeiro de 2012. Institui as diretrizes da Política Nacional de Mobilidade Urbana.
	Disponível em: <http://www.planalto.gov.br/ccivil\_03/\_ato2011- 2014/2012/lei/l12587.htm>. Acesso em: outubro de 2014.
	
	\noindent BRASIL. Instituto de Pesquisa Econômica Aplicada - IPEA. Mobilidade e Acessibilidade Urbanas Sustentáveis. Disponível em:<www.ipea.gov.br\_portal\_images\_stories\_PDFs\_comunicado\_120106\_comunicadoi pea128 mobilidade\_e\_acessibilidade\_urbanas\_sustentaveis>. Acesso em: outubro de 2014.
	
	\noindent CHEPTULIN, Alexandre. A Dialética Materialista. Categorias e leis da Dialética. São Paulo: Alfa Ômega, 1982.
	GIL, Antônio Carlos. Métodos e técnicas de pesquisa social. 4. ed. São Paulo: Atlas, 1994.
	
	\noindent PELIANO, José Carlos Pereira. Acumulação de Trabalho e Mobilidade do Capital. Brasília: UnB, 1990.
	
	\noindent RIBEIRO. Luiz Cesar de Queiroz. Metrópoles: entre a coesão e a fragmentação, a cooperação e o conflito / Luiz Cesar de Queiroz Ribeiro (organizador); Luciana Corrêado Lago, Sergio de Azevedo, Orlando Alves dos Santos Junior, (colaboradores). – São Paulo: Editora Fundação Perseu Abramo; 2004.
	
\end{document}
