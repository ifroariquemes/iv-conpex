\documentclass[article,12pt,onesidea,4paper,english,brazil]{abntex2}

\usepackage{lmodern, indentfirst, color, graphicx, microtype, lipsum,textcomp}			
\usepackage[T1]{fontenc}		
\usepackage[utf8]{inputenc}		

\setlrmarginsandblock{2cm}{2cm}{*}
\setulmarginsandblock{2cm}{2cm}{*}
\checkandfixthelayout

\setlength{\parindent}{1.3cm}
\setlength{\parskip}{0.2cm}

\SingleSpacing

\begin{document}
	
	\selectlanguage{brazil}
	
	\frenchspacing 
	
	\begin{center}
		\LARGE QUALIDADE MICROBIOLÓGICA DA GOMA DE MANDIOCA PRODUZIDA EM RONDÔNIA1\footnote{Trabalho realizado dentro das Ciências Agrárias com financiamento da PROPESP (Edital 18/2015).}
		
		\normalsize
		Bruna Naielly Kloos Oliveira\footnote{Bolsista (CNPq-IFRO), bruna.kloos.oliveira@outlook.com, Campus Ariquemes.} 
	Fabiana Pereira Sampaio\footnote{Bolsista (CNPq -IFRO), fabysampaio@outlook.com, Campus Ariquemes.} 
	Antonio Bisconsin-Junior4\footnote{Orientador, antonio.bisconsin@ifro.edu.br, Campus Ariquemes.}
	\end{center}
	
	\noindent No estado de Rondônia, o processamento da mandioca é realizado com o objetivo de produzir, principalmente, farinha do grupo seca e goma de mandioca. A goma, por possuir teor de água maior que o da farinha, apresenta maior predisposição às alterações microbiológicas. As características microbiológicas da goma de mandioca foram pouco estudadas e por serem produzidas de forma semi-industrial ou artesanal, apresenta uma grande variabilidade nestas características. O objetivo deste trabalho foi avaliar a qualidade microbiológica da goma de mandioca do estado de Rondônia. As gomas utilizadas neste estudo foram produzidas nos municípios de Pimenta Bueno, Ariquemes e Porto Velho (Rondônia). Foram avaliadas as contagens de microrganismos mesófilos aeróbios, bolores e leveduras, e coliformes totais e termotolerantes. A amostra de goma de mandioca de Pimenta Bueno apresentou para mesófilos aeróbios 2,28 x 103 UFC/g, bolores e leveduras 1,55 x 103 UFC/g e nenhum coliforme total ou termotolerante, enquanto, a goma de Ariquemes obteve para mesófilos aeróbios 8,95 x 104 UFC/g, bolores e leveduras 1,14 x 104 UFC/g e nenhum coliforme total ou termotolerante, e a goma de Porto Velho apresentou contagens superiores a 106 UFC/g de mesófilos aeróbios, 5,17 x 104 UFC/g de bolores e leveduras, 2,5 x 10² UFC/g de coliformes totais e nenhum coliforme termotolerante. Os resultados demonstram que a goma de tapioca produzida no município de Pimenta Bueno apresenta a melhor qualidade microbiológica entre as outras gomas avaliadas. Ainda, apesar da alta contagem de microrganismos nas gomas de Porto Velho e Ariquemes, todas as gomas atendem a legislação vigente, pois nenhuma apresentou coliformes termotolerantes.
	
	\vspace{\onelineskip}
	
	\noindent
	\textbf{Palavras-chave}: Mesófilos Aeróbios. Bolores e Leveduras. Coliformes.
	
\end{document}
