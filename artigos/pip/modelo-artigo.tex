\documentclass[article,12pt,onesidea,4paper,english,brazil]{abntex2}

\usepackage{lmodern, indentfirst, nomencl, color, graphicx, microtype, lipsum}			
\usepackage[T1]{fontenc}		
\usepackage[utf8]{inputenc}		

\setlrmarginsandblock{3cm}{3cm}{*}
\setulmarginsandblock{3cm}{3cm}{*}
\checkandfixthelayout

\setlength{\parindent}{1.3cm}
\setlength{\parskip}{0.2cm}

\SingleSpacing

\begin{document}
	
	\selectlanguage{brazil}
	
	\frenchspacing 
	
	\begin{center}
		\LARGE ELABORAÇÃO DE UMA MÁQUINA DE SOLDA ELÉTRICA COMO INSTRUMENTOS DE ENSINO PARA DISCENTES 
		
		\normalsize
	Diego Leônidas\footnote{Trabalho realizado dentro da área de Engenharias com financiamento do IFRO - Campus Vilhena (2016).} 
	
	Esplendo Vieira\footnote{ Orientador, diego.vieira@ifro.edu.br, IFRO – Campus Vilhena.} 
	
	Gabriel Felipe de Andrade Oliveira\footnote{Colaborador, gabriellegalvha@hotmail.com, IFRO – Campus Vilhena} 
	
		João Roberto Bond da Silva\footnote{ Colaborador, jobond46@gmail.com, IFRO – Campus Vilhena.}
		
		Paulo César Macedo\footnote{Co-orientador, paulo.macedo@ifro.edu.br, IFRO – CampusVilhena.}
		
		Ronaldo Pozzobom\footnote{Bolsista, ronaldopozzobom@hotmail.com, IFRO – CampusVilhena}
		 
	\end{center}
	
	% resumo em português
	\begin{resumoumacoluna}
		Este trabalho tem o intuito de gerar um instrumento de ensino para os professores das disciplinas de Química, Física e Programação. Foi pesquisado sobre elementos que compõem uma máquina de solda e posteriormente analisado de acordo com o custo e benefício de cada um para a construção de uma máquina caseira, com o objetivo de ensinar os alunos o que foi explicado em sala deaula.
		
		\vspace{\onelineskip}
		
		\noindent
		\textbf{Palavras-chave}:Arduino. Eletrólise. Prática Didática 
	\end{resumoumacoluna}
	
	\textual
	
	\section*{Introdução}
	
	Este projeto começou com o objetivo de fabricar um instrumento de estudo para o ensino das disciplinas de química, física e programação. A máquina de solda serve como complementação do conhecimento mostrado em sala, assim o aluno poderá ver na prática o que foi explicado pelo professor, já que a Instituição possuía poucos materiais de ensino fabricados pelos alunos. No processo de confecção da máquina foram utilizados alguns componentes químicos e realizados alguns estudos específicos para conseguirmos um resultado plausível, que será discutido posteriormente neste artigo. Um segundo objetivo é trazer a correlação dos cursos de técnico em Informática e Eletromecânica do IFRO Campus Vilhena, pois o projeto possui alunos dos doiscursos.
	
	\section*{Material e Método}
	
	Para que houvesse uma economia na hora da fabricação da máquina, os alunos ficaram responsáveis por pesquisar quais os melhores materiais para utilizar na máquina, levando em conta o custo e o benefício que cada um deles traria. Os principais materiais utilizados foram cobre, sal de cozinha (Cloreto de Sódio/NaCl), água natural e placas de acrílico de 5 milímetros de espessura. Para que a máquina ficasse automatizada, foi utilizado Arduino, componentes de leitura do Arduino, eixo liso para que os eletrodos possam correr dentro da solução e rolamentos axiais.
	A metodologia foi muito simples, porém totalmente produtiva. Os alunos  ficaram responsáveis por fazer pesquisas sobre os materiais como aqui já dito e também sobre eletrônica básica, pois todo conhecimento sobre a área era necessário. Após esta etapa, os participantes do projeto começaram a montagem da caixa de acrílico, onde a substância que conduz a energia fica. Logo em seguida, foi a montagem dos mecanismos de automação damáquina.
	
	
	
	\section*{Resultados e Discussão}
	
	Ao iniciar a pesquisa, foram feitos alguns testes para que pudéssemos definir qual seria a melhor solução para utilizar na eletrólise, já que a mesma não iria gerar energia, apenas a conduziria, “injetando” uma certa quantia de elétrons pelo seu polo negativo e “aspirando” a mesma quantia pelo polo positivo.(FELTRE,2005)
	Após as pesquisas e testes, encontramos alguns valores (Tabela 1) que constataram que o cloreto de sódio (NaCl) seria a melhor substância para se utilizar na máquina mesmo não sendo o melhor condutor, por conta de a relação custo- benefício ser maior que o hidróxido de sódio que é complicado de ser encontrado e possui um preço bem maior que o cloreto de sódio.
	\begin{figure}[t]
		
	\end{figure}
	
	
	
	\section*{Conclusões}
	
	Texto con.
	
	\section*{Instituição de Fomento}
	
	Texto IF se houver.
	
	\section*{Referências}
	
	Refs.
	
\end{document}
