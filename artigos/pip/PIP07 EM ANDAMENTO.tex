\documentclass[article,12pt,onesidea,4paper,english,brazil]{abntex2}

\usepackage{lmodern, indentfirst, nomencl, color, graphicx, microtype, lipsum,multirow,textcomp}			
\usepackage[T1]{fontenc}		
\usepackage[utf8]{inputenc}		

\setlrmarginsandblock{2cm}{2cm}{*}
\setulmarginsandblock{2cm}{2cm}{*}
\checkandfixthelayout

\setlength{\parindent}{1.3cm}
\setlength{\parskip}{0.2cm}

\SingleSpacing

\begin{document}
	
	\selectlanguage{brazil}
	
	\frenchspacing 
	
	\begin{center}
		\LARGE BORBOLETAS FRUGÍVORAS (LEPIDOPTERA: NYMPHALIDAE) OCORRENTES NO REMANESCENTE FLORESTAL DO INSTITUTO FEDERAL DE RONDÔNIA – CAMPUS ARIQUEMES\footnote{Trabalho realizado dentro da área de Conhecimento CNPq/CAPES: Ciências Biológicas com financiamento do IFRO.}
		
		\normalsize
		Pedro Paulo Machado Nascimento\footnote{Bolsista (PIBITI), pedropmnascimento@gmail.com, Campus Ariquemes.} 
		Estéfano Gambarini Monteiro \footnote{Bolsista (PIBITI), estefanomonteiro87@gmail.com, Campus Ariquemes.} 
		Renata Alves de Sousa \footnote{Colaboradora, renatacbi18@gmail.com, Campus Ariquemes.} 
	Gisele Renata de Castro\footnote{Co-orientadora, gisele.renata@ifro.edu.br, Campus Ariquemes.} 
	Elaine Oliveira Costa de Carvalho\footnote{Orientadora, elaine.carvalho@ifro.edu.br, Campus Ariquemes}
	\end{center}
	
	% resumo em português
	\begin{resumoumacoluna}
		As borboletas frugívoras pertencem a família \textit{Nymphalidae} e se alimentam de frutas fermentadas, excrementos e matéria orgânica em decomposição disponíveis no ambiente. São muito estudadas como bandeira da conservação ambiental e por este motivo são objetos de pesquisa neste trabalho. O objetivo da pesquisa foi conhecer a diversidade de borboletas do remanescente florestal do Instituto Federal de Rondônia (IFRO) – \textit{Campus Ariquemes}. O trabalho foi realizado de agosto de 2015 a julho de 2016. Os dados foram coletados por meio de registro fotográficos a cada 15 dias, tendo em média 770 horas de esforço amostral. Foram registradas 39 espécies, subdivididas em 4 subfamílias e 11 tribos. A subfamília \textit{Satyrinae} foi a mais abundante, com 21 espécies. O remanescente florestal evidenciou uma grande diversidade de espécies da família \textit{Nymphalidae}.
		
		\vspace{\onelineskip}
		
		\noindent
		\textbf{Palavras-chave}:Conservação ambiental. Diversidade. Amazônia.
		
	\end{resumoumacoluna}
	
	\textual
	
	\section*{Introdução}
	
A ordem Lepidoptera é representada por insetos conhecidos como borboletas e mariposas, sendo o segundo grupo mais diverso de insetos, com cerca de 170.000 espécies descritas em 124 famílias conhecidas em todo o planeta (CARRANO- MOREIRA, 2014). No Brasil são conhecidas quase 26 mil espécies e das 124 famílias, 71 ocorrem no país (DUARTE et al., 2012). As borboletas da família Nymphalidae em sua maioria alimentam-se de material orgânico em decomposição, como frutas fermentadas, fezes e folhas podres conhecidas como borboletas frugívoras (SANTOS, 2011).Segundo Freitas \& Brown (2003), além da relevância anteriormente citada, as borboletas também são usadas como indicadores biológicos de qualidade da paisagem para monitoramento ambiental. Muitos cientistas do Brasil e de outros países estudam de várias formas as borboletas, mas na Amazônia este estudo ainda é escasso (VIEIRA et al., 2014). Devido ao relatado, o objetivo desta pesquisa é conhecer a diversidade de borboletas frugívoras do remanescente florestal do IFRO
– Campus Ariquemes com o intuito de promover a conservação do habitat natural e das espécies locais.

	
	\section*{Material e Método}
	
	A pesquisa foi realizada no remanescente florestal do IFRO – Campus Ariquemes, localizada no perímetro rural, ocupando uma área de 299 hectares, sendo 259 hectares de mata secundária, que situa-se no município de Ariquemes – Rondônia (Florestal S/A, 1984). O clima é característico de quase todo o estado sendo equatorial, predominantemente quente e úmido subdividido com temporadas de chuva e seca que pode durar até dois meses de temperatura anual média acima de 26ºC. (IBGE, 2014).
	Foi realizado uma transecção principal de 2000m e quatro transecções secundárias de 100m cada uma, onde realizou-se as revisões de ocorrência de espécies diversificadas de borboletas frugívoras que foram fotografadas para posterior identificação. Para a atração das borboletas utilizou-se isca com banana amassada e caldo de cana fermentado por 48 horas a uma proporção de 3 kg de banana para 1 litro de caldo de cana, ocorrendo visitas a cada 15 espécies dias.
	A identificação das espécies fotografadas foram realizadas de acordo com as referências bibliográficas especializadas, Uehara-Prado et al. (2004) e D’Abrera (1987).
	
	\section*{Resultados e Discussão}
	
	Foram fotografadas 39 espécies, subdivididas em 4 subfamílias (Biblidinae, Charaxinae, Nymphalinae e Satyrinae) e 11 tribos - Ageronini (2 espécies), Callicorini (2 espécies), Epicallini (5 espécies), Epiphilini (1 espécie), Anaeini (2 espécies), Preponini (1 espécie), Coeini (5 espécies), Morphini (5 spécies),Haeterini
	
	
	(5 espécies), Brassolini (5 espécies) e Satytini (6 espécies) como mostra na Tabela 1.
	A subfamília mais abundante foi a Satyrinae. A região neotropical contém a maior riqueza de Satyrinae do mundo (D’ABRERA, 1988). De acordo com Uehara- Prado et al. (2003) esta subfamília indica ser um grupo generalista quanto a habitats. Durante um ano de fotografias foi possível acompanhar as diversas espécies de estações de seca e de chuva, tendo os meses de maio a julho como os mais abundantes, provavelmente devido à alta disponibilidade de frutos (CHECA et al., 2014). As fotografias ocorreram em sua maioria de áreas abertas e trilhas da mata, onde havia maiores ocorrências de borboletas.
	% Please add the following required packages to your document preamble:
	% \usepackage{multirow}
	\begin{table}[h]
		\centering
		\caption{Lista de espécies de borboletas frugívoras fotografadas no remanescente florestal do Instituto Federal de Rondônia – Campus Ariquemes, de agosto/2015 a julho/2016.}
		\label{my-label}
	\begin{tabular}{ l l l }
		\cline{1-1}\cline{2-2}\cline{3-3}  
		\multicolumn{1}{|p{1.583cm}|}{\textbf{Subfamília}} &
		\multicolumn{1}{p{1.283cm}|}{\textbf{Tribo}} &
		\multicolumn{1}{p{3.600cm}|}{\textbf{Espécies}}
		\\  
		\cline{1-1}\cline{2-2}\cline{3-3}  
		\multicolumn{1}{|p{1.583cm}|}{\multirow{3}{*}{Biblidinae}} &
		\multicolumn{1}{p{1.283cm}|}{Ageronini} &
		\multicolumn{1}{p{3.600cm}|}{\textit{Hamadryas amphinome }Linnaeus, 1767  			
			
			\textit{Panacea procilla }Hewitson, 1852  			
			
			\textit{Callicore }\textit{cynosure }Doubleday, 1847}
		\\  
		\cline{2-2}\cline{3-3}  
		\multicolumn{1}{||}{} &
		\multicolumn{1}{p{1.283cm}|}{Callicorini} &
		\multicolumn{1}{p{3.600cm}|}{\textit{Diaethria }sp.  			
			
			\textit{Cat}\textit{onephele acontius }Linnaeus, 1771}
		\\  
		\cline{2-2}\cline{3-3}  
		\multicolumn{1}{||}{} &
		\multicolumn{1}{p{1.283cm}|}{Epicallini} &
		\multicolumn{1}{p{3.600cm}|}{\textit{Catonephele numilia }Cramer, 1775  			
			
			\textit{Eunica pusilla }H. Bates, 1864  			
			
			\textit{Eunica viola }H. Bates, 1864}
		\\  
		\cline{1-1}\cline{2-2}\cline{3-3}  
		\multicolumn{1}{|p{1.583cm}|}{\multirow{3}{*}{Charaxinae}} &
		\multicolumn{1}{p{1.283cm}|}{   			
			
			Epiphilini} &
		\multicolumn{1}{p{3.600cm}|}{\textit{Nessea obrinus }Linnaeus,1758  			
			
			\textit{Temenis laothoe }Cramer,1777}
		\\  
		\cline{2-2}\cline{3-3}  
		\multicolumn{1}{||}{} &
		\multicolumn{1}{p{1.283cm}|}{Anaeini} &
		\multicolumn{1}{p{3.600cm}|}{\textit{Hypna }\textit{Clytemnestra }Cramer, 1777}
		\\  
		\cline{2-2}\cline{3-3}  
		\multicolumn{1}{||}{} &
		\multicolumn{1}{p{1.283cm}|}{   			
			
			Preponini} &
		\multicolumn{1}{p{3.600cm}|}{\textit{Mem}\textit{phis moruus }Fabricius, 1775  			
			
			\textit{Archeoprepona demophon }Linnaeus, 1758}
		\\  
		\cline{1-1}\cline{2-2}\cline{3-3}  
		\multicolumn{1}{|p{1.583cm}|}{Nymphalinae} &
		\multicolumn{1}{p{1.283cm}|}{Coeini} &
		\multicolumn{1}{p{3.600cm}|}{\textit{Baetus beotus }Doubleday, [1849]  			
			
			\textit{Colobura dirce }Linnaeus, 1758  			
			
			\textit{Historis acheronta }Fabricius, 1775  			
			
			\textit{Historis odius }Fabricius, 1775}
		\\  
		\cline{1-1}\cline{2-2}\cline{3-3}  
		\multicolumn{1}{|p{1.583cm}|}{\multirow{2}{*}{   			
				
				Satyrinae}} &
		\multicolumn{1}{p{1.283cm}|}{   			
			
			Morphini} &
		\multicolumn{1}{p{3.600cm}|}{\textit{Tigridi}\textit{a acesta }Linnaeus, 1758  			
			
			\textit{Antirrhea philaretes }Linnaeus, 1758  			
			
			\textit{Caerois chorinaeus }Fabricius, 1775  			
			
			\textit{Morpho helenor }Cramer, 1776  			
			
			\textit{ Morpho marcus }Schaller, 1785}
		\\  
		\cline{2-2}\cline{3-3}  
		\multicolumn{1}{||}{} &
		\multicolumn{1}{p{1.283cm}|}{   			
			
			Haeterini} &
		\multicolumn{1}{p{3.600cm}|}{\textit{Morpho }\textit{Menelaus }Linnaeus,1758  			
			
			\textit{Cithaerias pireta }Stoll, [1780]  			
			
			\textit{Haetera piera }Linnaeus, 1758  			
			
			\textit{Haetera piera }Linnaeus, 1758  			
			
			\textit{Haetera piera }Linnaeus, 1758}
		\\  
		\cline{1-1}\cline{2-2}\cline{3-3}  
		\multicolumn{1}{|p{1.583cm}|}{\multirow{2}{*}{ }} &
		\multicolumn{1}{p{1.283cm}|}{Brassolini} &
		\multicolumn{1}{p{3.600cm}|}{\textit{Pierella hyalinus} Gmelin, [1790]  			
			
			\textit{Pierella lena} Linnaeus, 1767  			
			
			\textit{Bia actorion} Linnaeus,1763  			
			
			\textit{Caligo idomeneus} Linnaeus, 1758  			
			
			\textit{Catoblepia berecynthia} Cramer, 1777  			
			
			\textit{Catoblepiasoranus} Westwood,1851  			
			
			\textit{Opsiphanes quiteria} Stoll,[1780]}
		\\  
		\cline{2-2}\cline{3-3}  
		\multicolumn{1}{||}{} &
		\multicolumn{1}{p{1.283cm}|}{Satytin} &
		\multicolumn{1}{p{3.600cm}|}{\textit{Chloreuptychiasp.}  			
			
			\textit{Cissia myneca} Cramer, 1780  			
			
			\textit{Magneuptychia sp.}  			
			
			\textit{Taygetis echo }Cramer, 1775  			
			
			\textit{Taygetis thamyra} Cramer, 1779  			
			
			\textit{Taygetisvirgilia} Cramer,1776}
		\\  
		\hline
		
\end{tabular} 
	\end{table}
	
	\section*{Conclusões}
	
	O remanescente florestal aqui estudado apontou uma grande diversidade de borboletas frugívoras, necessitando ainda de maiores estudos para conhecer a abundância das mesmas, além de poder utiliza-las como indicadoras da conservação do ambiente. O trabalho de levantamento de espécies por meio da fotografia se mostrou eficiente. Por meio das iscas atrativas foi possível mantê-las paradas por mais tempo, possibilitando variadas fotografias, conseguindo apresentar as espécies existentes no remanescente estudado, sem que fosse necessário a morte das mesmas.
	
	\section*{Instituição de Fomento}
	
Instituto Federal de Educação, Ciência e Tecnologia de Rondônia – IFRO.
	
	\sloppy
	\section*{Referências}
	
	\noindent CARRANO-MOREIRA, A. F. Insetos: Manual de coleta e identificação. 2ª ed. Rio de Janeiro. Editora: Technical Books, 2014. 369p.
	
	\noindent CHECA, M.F. et al. Microclimate variability significantly affects the composition, abundance and phenology of butterfly communities im a highly theeatened neotropical dry forest. Florida Entomol. 97:1-13, 2014.
	
	\noindent	DUARTE, M. et al., Lepidoptera. in RAFAEL, J.A; MELO, G.A.R; CARVALHO, C.J.B;
	CASARI, S.A. & CONSTANTINO, R. Insetos do Brasil: Diversidade e Taxonomia. Ribeirão Preto. Editora Holos, 2012. p.625-682.
	5
	
	\noindent D’ABRERA, B. Butterflies of the Neotropical region. Part IV. Nymphalidae (partim). Victoria, Hill House, 1987. p.528-678.
	
	\noindent D’ABRERA, B. Butterflies of the Neotropical Region. Part V. Nymphalidae & Satyridae. 1.ed. Victoria: Hill House, 1988. p.680-877.
	
	\noindent FLORESTA S/A, F. R. Escritura Pública de venda e compra. Livro nº 002. Folhas 88 e 89. Cartório de Registro Civil. 1984.
	
	\noindent FREITAS, A.V. L.; FRANCINI, R. B. & BROWN J. R., K. S. Insetos como indicadores ambientais. In Métodos de estudos em biologia da conservação e manejo da vida silvestre (L Cullen Jr., C. Valladares-Pádua & R. Rudran, orgs.). Editora da UFPR, 2003. p.125-151.
	
	\noindent IBGE. Instituto Brasileiro de Geografia e Estatística. 2016. Ariquemes - Rondônia. Histórico e Formação Administrativa. Disponível em:
	<http://biblioteca.ibge.gov.br/visualizacao/dtbs/rondonia/ariquemes.pdf>. Acesso em: 14 set. 2016.
	
	\noindent SANTOS, J. P. et al. Fruit-feeding butterflies guide of subtropical Atlantic Forest and Araucaria Moist Forest in State of Rio Grande do Sul, Brazil. Biota Neotrop., 2011.
	
	\noindent VIEIRA, R. S. et al. Guia Ilustrado de Borboletas frugívoras da Reserva Florestal Adolpho Ducke. Instituto Nacional de Pesquisas da Amazônia. Manaus: Editora INPA, 2014.
	
	\noindent UEHARA-PRADO, M. et al. Guia de borboletas frugívoras da Reserva Estadual do Morro Grande e Região de Caucaia do Alto, Cotia (São Paulo). Biota Neotrop. 4., 2004.
	
	\noindent UEHARA-PRADO, M. Efeito de fragmentação florestal na guilda de borboletas frugívoras do Planalto Atlântico Paulista. Dissertação (Mestrado em Ecologia) Instituto de Biologia, Universidade Estadual de Campinas. Campinas, 2003. 144f.
	

\end{document}