\documentclass[article,12pt,onesidea,4paper,english,brazil]{abntex2}

\usepackage{lmodern, indentfirst, nomencl, color, graphicx, microtype, lipsum}			
\usepackage[T1]{fontenc}		
\usepackage[utf8]{inputenc}		
\usepackage{textcomp}
\setlrmarginsandblock{2cm}{2cm}{*}
\setulmarginsandblock{2cm}{2cm}{*}
\checkandfixthelayout

\setlength{\parindent}{1.3cm}
\setlength{\parskip}{0.2cm}

\SingleSpacing

\begin{document}
	
	\selectlanguage{brazil}
	
	\frenchspacing 
	
	\begin{center}
		\LARGE GEOTECNOLOGIA APLICADA NA LOCAÇÃO E IDENTIFICAÇÃO DE ÁRVORES NA PRAÇA DOS MIGRANTES, JI-PARANÁ/RO
		
		\normalsize
		Juliana Campos Curitiba\footnote{Juliana Campos Curitiba, email jucamposjipa@gmail.com} 
		
	\end{center}
	
	% resumo em português
	\begin{resumoumacoluna}
		 O uso de geotecnologia vem crescendo a cada dia no mundo e no Brasil. O GPS é uma ferramenta que está presente em várias áreas do conhecimento científico, sendo um grande aliado no desenvolvimento de pesquisas e contribuindo para uma compreensão do espaço geográfico através dos dados que permitem construir e formular hipóteses reais. O objetivo deste trabalho foi a utilização de Geotecnologia aplicada na locação e identificação de árvores na Praça dos Migrantes, Ji-Paraná/RO. Após a identificação das famílias, nome científico e nome comum das árvores e palmeiras presentes na praça são propicias para a ornamentação da praça, para a qualidade do ar, mas poderia haver mais exemplares, incluindo árvores frutíferas..
		
		\vspace{\onelineskip}
		
		\noindent
		\textbf{Palavras-chave}: . GPS. marcação.
		
		
	\end{resumoumacoluna}
	
	\section*{Introdução}
	
	O termo geotecnologia, pode ser descrita como as novas tecnologias ligadas às geociências, as quais trazem avanços significativos no desenvolvimento de pesquisas, em ações de planejamento, manejo e em tantos outros aspectos relacionados à estrutura do espaço geográfico. Fitz (2008, p. 11)
	
	Uma das aplicabilidades do GPS de navegação é conhecer e registrar a localização de determinado local na superfície terrestre. A proposta deste trabalho compreende a importância de localizar e identificar as árvores na Praça Jardim dos Migrantes, na cidade de Ji-Paraná-RO, com uso do aparelho GPS para obter as coordenadas geográficas.
	
	Essa temática tem como objetivo identificar as árvores de grande porte que configuram a paisagem da praça e proporcionar a população local que faz uso desse espaço, informações sobre as árvores, como a que família pertence, qual o nome científico e qual é o nome comum de cada uma delas, proporcionando a esse espaço um novo jeito de ter o habito de ler as informações sobre esse bem natural que é importante para a paisagem da cidade e proporciona bem estar para a saúde e promover a valorização do espaço público. 
	
	\section*{Material e Método}
	
O levantamento da locação das árvores foi feita na Praça dos Migrantes (Prefeitura de Ji-Paraná/RO, 2015) sob a coordenada geográfica latitude 10°52’34,8”S e longitude 61°57’43,8”W, com altitude 163.

Inicialmente foi elaborado um croqui do perímetro da Praça e com os pontos de locação de cada árvore para facilitar o caminhamento da execução da coleta dos pontos. Cada ponto de locação recebeu uma numeração para identificação da coordenada geográfica.

O GPS de navegação utilizado para a demarcação é modelo eXplorist 510 da marca Magellan. Outra tecnologia utilização como ferramenta para a identificação das árvores, foi uma câmera fotográfica da marca Sony e modelo Cyber-shot DSC- H7 8.1, com objetivo de coletar imagens detalhados de folhas, caule, raiz, flor e  fruto, para fazer a constatação da existência de árvores frutíferas e, ou ornamental e classificar cada árvore com nome da família, nome cientifico e nome comum.

Com os dados coletados em campo, foram geradas três informações refentes a distribuição de árvores. Duas com os valores das coordenadas geográficas das Árvores e das Palmeiras. E uma tabela com as informações do nome da família, nome científico e nome comum das espécies de Árvores e Palmeiras que foram identificadas na área de estudo, Praça dos Migrantes, Ji-Paraná/RO.
	
	\section*{Resultados e Discussão}
	
Localização das Arvores da Praça dos Migrantes – Ji-paraná, RO.
Foi identificado após o levantamento o total de 24 árvores. Desse total, a identificação 01 – 14 são árvores adultas. E do ponto de identificação nº 15 - 24 foram identificadas árvores jovens de porte ainda pequeno.

Localização de Palmeiras na Praça dos Migrantes – Ji-paraná, RO. Foi identificado após o levantamento o total de 08 palmeiras. 

\begin{table}[h]
	\centering
	\caption{Identificação das Árvores da Praça dos Migrantes – Ji-paraná/RO}
	\label{my-label}
	\begin{tabular}{lll}
		\hline
		Família         & Nomecientífico                              & Nome comum        \\
		\hline
		Arecaceae       & Roytonea Oleracea (Jacg.) O.F.Cook.         & Palmeira,Imperial \\
		Bigneniaceae    & Tabebuia Impetiginosa (Mart.ex DC.) Istandl & Ipê Roxo          \\
		Chysobalanaceae & Licania Kunthiana Hook.F.                   & Aoiti             \\
		Chysobalanaceae & Licania Apetala ( E. Mey.) Fritsch          & Caripé            \\
		Moraceae        & Ficus Benjamina L.                          & Figueira          \\
		Caesalpiniaceae & Schilobium Amazonicum Huber ex. Ducke       & Bandarra          \\
		Cupressaceae    & Thuja Campacta Standsh ex.Gordon            & Pinheiro 
		\\ \hline        
	\end{tabular}
\end{table}

Após o levantamento de campo do total de 24 árvores foram identificadas 06 (seis) famílias e destas apenas a família Chysobalanaceae apresentou duas espécies diferentes Licania Kunthiana Hook.F. de nome comum Aoiti e Licania Apetala ( E. Mey.) Fritsch, de nome comum Caripé.

Foi identificada ainda, uma família denominada Arecaceae, com apenas uma espécie denominada Roytonea Oleracea (Jacg.) O.F.Cook., de nome comum palmeiraimperial.	Constatamos também a existência de duas espécies que não é endêmica da região Norte do Brasil, a Thuja Campacta Standsh ex.Gordon da família Cupressaceae, de nome comum Pinheiro; a outra espécie é a Roytonea Oleracea (Jacg.) O.F.Cook., de nome comum PalmeiraImperial.

De todas as espécies identificadas na praça, não houve a constatação de nenhuma árvore frutífera. Todas as espécies identificadas classificam-se como árvores ornamentais.

A identificação escrita em placa fixada próximo ao caule de cada espécie com a informação do nome da família, nome científico e nome comum de cada árvore e palmeira serão futuramente realizadas buscando uma possível parceria com a Prefeitura e Secretária de Meio Ambiente do município de Ji-Paraná/RO.

	
	\section*{Conclusões}
	
O uso de geotecnologia no planejamento e na execução deste trabalho foi importante na aquisição de dados, porque foi possível verificar a área da Praça dos Migrantes, local da área de estudo, no programa Google Earth Pro, que utiliza outros tipos de geotecnologia, como sensoriamento remoto, que são as imagens desatélite.

Nos pontos das coordenadas geográficas, verificou-se que tanto os valores de latitude, como os valores de longitudes ficaram com valores próximos, como era esperado, porque as árvores e as palmeiras encontram em posições próximas uma das outras.

A identificação das famílias, nome científico e nome comum demonstrou que as árvores e palmeiras presentes na praça são propicias para a ornamentação da praça, para a qualidade do ar, mas entretanto poderia haver mais exemplares, incluindo árvores frutíferas.
	\section*{Instituição de Fomento}
	
Instituto Federal de Ciência e Tecnologia- IFRO / Campus Ji-Paraná Fornecimento de GPS
	
	\section*{Referências}
	

FITZ, P. R. Geoprocessamento sem complicação. São Paulo, Oficina de Textos, 2008, 161p.



\noindent MACARO, L. A. M. Geotecnologias aplicadas à caracterização da qualidade ambiental da bacia hidrográfica do igarapé pintado. 2013. 37 f. Monografia – Departamento de Engenharia Ambiental, Universidade Federal de Rondônia, Campus de Ji-Paraná - RO.

\noindent ROCHA, C. H. B. Geoprocessamento – Tecnologia Transdisciplinar. Minas Gerais, Ed. do Autor, 2000, 220p.

\noindent ZANOTTA, D. C; CAPELLETTO, E; MATSUOKA. O GPS: unindo ciênciae
tecnologia em aulas de física. Revista Brasileira de Ensino de Física, v. 33, n. 2, 2313 (2011). Disponível em: <http://www.scielo.br/scielo.php?pid=S1806-11172011000200014\&script=sci\_arttext> . Acesso em: 06 de agosto2016.



	
\end{document}