\documentclass[article,12pt,onesidea,4paper,english,brazil]{abntex2}

\usepackage{lmodern, indentfirst, nomencl, color, graphicx, microtype, lipsum}			
\usepackage[T1]{fontenc}		
\usepackage[utf8]{inputenc}		

\setlrmarginsandblock{2cm}{2cm}{*}
\setulmarginsandblock{2cm}{2cm}{*}
\checkandfixthelayout

\setlength{\parindent}{1.3cm}
\setlength{\parskip}{0.2cm}

\SingleSpacing

\begin{document}
	
	\selectlanguage{brazil}
	
	\frenchspacing 
	
	\begin{center}
		\LARGE MÍDIA E POLÍTICA: \\RONDÔNIA NAS PÁGINAS DA REVISTA VEJA.
		
		\normalsize
	Wdmial Gabriela Borges Romanini\footnote{Wdmila Gabriela Borges Romanini, wdmilagbr15@gmail.com, IFRO - Campus Cacoal.} 
	Allonya Peixoto da Silva\footnote{Allonya Peixoto da Silva, allonyapeixoto78@gmail.com, IFRO – Campus Cacoal.} 
		
	\end{center}
	
	% resumo em português
	\begin{resumoumacoluna}
		Com os estudos feitos nos conceitos das influências da mídia de massa e no comportamento social do indivíduo. A teoria do agenda setting se fundamenta  em um tipo de efeito social dos meios de informação, os quais podem citar a televisão, os jornais e as rádios, que selecionam, em grau de importância, os temas que os indivíduos falarão ou discutirão por um determinado período, porém, iremos aplicar esse conceito da teoria da mídia no ano de 1982, quando uma revista de mídia nacional tão renomada citou pela primeira vez o estado de Rondônia e quais foram as experiências para pessoas da região Sul e Sudeste quando se depararam com um estado inóspito.
		
		\vspace{\onelineskip}
		
		\noindent
		\textbf{Palavras-chave}: REVISTA VEJA. RONDÔNIA. MÍDIA.
	\end{resumoumacoluna}
	
	\section*{Introdução}
	
A teoria do agenda setting tem como finalidade social da mídia que assimila a seleção, disposição e incidência de notícias sobre os temas que o público falará e discutirá. O trabalho pretende apresentar uma revisão desta teoria com base histórica no estado de Rondônia, tendo como ponto de partida caminhos que levam a um breve histórico dos estudos e o desempenho do processo de agendamento e migratório. Sendo assim, espera-se obter um desempenho deste conceito assim como de suas características e limitações. O fundamento de investigação que estuda sobre  o  quê  e  como  os  assuntos  devem  ser  pensados  é  a  hipótese  do agenda setting, abordaremos o impacto dessa teoria na sociedade, ou seja, como foi no meio da sociedade e qual a visão da população. O interesse pelos efeitos dos meios de comunicação na opinião pública produziu uma abrangente literatura sobre o agenda setting, causando na população da época uma inversão da realidade apresentada pela revista “Veja”.
A agenda foi recolhida por meio da observação da revista semanal de repercussão nacional: \textbf{Veja}. O exame dos temas da agenda foi realizado por intermédio das capas, páginas, colunas, seções e editoriais. Os dados coletados foram objeto de análise de conteúdo (sinteticamente consiste em isolar, de um conjunto de mensagens, determinados elementos – palavras, frases, imagens, símbolos, etc. – em função de certas categorias previamente determinadas), classificados em categorias e quantificados.
	
	\section*{Material e Método}
	
	A agenda foi recolhida por meio da observação da revista semanal de repercussão nacional:\textbf{Veja} . O exame dos temas da agenda foi realizado por intermédio das capas, páginas, colunas, seções e editoriais. Os dados coletados foram objeto de análise de conteúdo (sinteticamente consiste em isolar, de um conjunto de mensagens, determinados elementos – palavras, frases, imagens, símbolos, etc. – em função de certas categorias previamente determinadas), classificados em categorias e quantificados.
	
	\section*{Resultados e Discussão}
	
A unidade de registro (o que se conta) escolhida foi o tema (análise temática), considerado pela literatura pertinente às técnicas de análise de conteúdo como o mais adequado para registrar opiniões, atitudes, valores, crenças e tendências.
Quanto à unidade de contexto (onde se conta) foram adotados dois critérios: (1) em relação às manchetes e aos títulos das matérias da capa tomou-se como referência à frase (no caso, as próprias manchetes, chamadas e títulos) e (2) em relação às colunas, o parágrafo. Nesse caso, se convencionou que um tema estaria configurado se presente em pelo menos 25\% do total de parágrafos do texto. Assim, para a compreensão em sua amplitude sobre o processo migratório para Rondônia torna-se imprescindível, se atentar para mais uma variável totalmente esquecida, o estudo da influência da mídia (o poder das palavras) em estabelecer uma agenda pública que tocasse no psicológico/cognitivo das pessoas, visando o convencimento e a construção do imaginário social (“eldorado”) com o objetivo de motivar e a mobilizá-los para a ação migratória, haja vista, que o vultoso fluxo migratório para um local inóspito, desconhecido, repleto de mitos e a ser desbravado, não poderia somente ser explicado por questões de mobilidade econômica e/ou social e  políticas, mas por algo mais profundo que tocasse a alma.

	
	\section*{Conclusões}
	
Observou-se que a mídia interferiu coercitivamente no modo em como a população desenvolve um senso crítico sobre a Região Norte em especial Rondônia, visto que a manipulação da informação interfere diretamente no meio social, estabelecendo opiniões por meio do poder do Estado. É válido ressaltar que o alcance do controle dos veículos midiáticos é de fundamental relevância para a manutenção e influência dos governos sobre a população. A análise dos dados demonstrou a existência de uma retroalimentação entre os assuntos do tema prevalecente. Pois, quando notamos em especial o tema e a frequência com que eram exibidos os diferentes assuntos pertencentes a tal matéria, mesmo com outros temas figurando, no geral, estes faziam direcionar os seus resultados para o objeto em questão, conforme a posição ideológica da revista e da edição selecionada.

	\section*{Agradecimento}
	Agradecemos ao professor orientador Davys Sleman de Negreiros pela atenção, aos nossos pais e ao Campus Cacoal-IFRO pelo incentivo à pesquisa.
	
	
	\section*{Instituição de Fomento}
	
Instituto Federal de Educação, Ciência e Tecnologia de Rondônia/IFRO – campus
Cacoal.

	
	\section*{Referências}
	
	\noindent DEBORD, G. “A sociedade do espetáculo”. RJ, Contraponto, 1997.
	
	\noindent HABERMAS, J. “A Mudança Estrutural da Esfera Pública”. RJ, Tempo Brasileiro, 1984. 
	
	\noindent GUIMARÃES, I. C. “A televisão brasileira na transição (um caso de conversão rá-pida à nova ordem)”. Comunicação \& política III, no 6,1986.
	
	
\end{document}