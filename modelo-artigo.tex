\documentclass[article,12pt,onesidea,4paper,english,brazil]{abntex2}

\usepackage{lmodern, indentfirst, nomencl, color, graphicx, microtype, lipsum}			
\usepackage[T1]{fontenc}		
\usepackage[utf8]{inputenc}		

\setlrmarginsandblock{3cm}{3cm}{*}
\setulmarginsandblock{3cm}{3cm}{*}
\checkandfixthelayout

\setlength{\parindent}{1.3cm}
\setlength{\parskip}{0.2cm}

\SingleSpacing

\begin{document}
	
	\selectlanguage{brazil}
	
	\frenchspacing 
	
	\begin{center}
		\LARGE APLICANDO A TEORIA NA PRÁTICA – UM ESTUDO DE CASO NA SECRETARIA MUNICIPAL DE SAÚDE DE PORTO VELHO - RO\footnote{Trabalho realizado dentro da área de Conhecimento CNPq/CAPES: Ciências Sociais Aplicadas, com apoio do IFRO- Campus Porto Velho Zona Norte.}
	
	\normalsize
	Maria Genilda Batista da Silva\footnote{Aluna pesquisadora do Curso de Tecnologia em Gestão Pública, genilda\_vandecy@hotmail.com,Campus Porto Velho Zona Norte} 
	Reuria da Silva Moreira\footnote{Aluna pesquisadora do Curso de Tecnologia em Gestão Pública reuria.moreira@ifro.edu.br, IFRO – Campus Porto Velho Zona Norte.} 
	Adonias Soares da S.Júnior\footnote{Orientador (a), adonias.silva@ifro.edu.br, Campus Porto Velho Zona Norte.} 
	Sandra SenaReis\footnote{Aluna pesquisadora do Curso de Tecnologia em Gestão Pública, sandrasenareis14@gmail.com, Campus.} 
	\end{center}
	
	% resumo em português
	\begin{resumoumacoluna}
		Resumo do artigo.
		
		\vspace{\onelineskip}
		
		\noindent
		\textbf{Palavras-chave}: palavra 1. palavra 2. palavra 3.
	\end{resumoumacoluna}
	
	\textual
	
	\section*{Introdução}
	
	Texto da introdução.
	
	\section*{Material e Método}
	
	Texto MM.
	
	\section*{Resultados e Discussão}
	
	Texto RD.
	
	\section*{Conclusões}
	
	Texto con.
	
	\section*{Instituição de Fomento}
	
	Texto IF se houver.
	
	\section*{Referências}
	
	Refs.
	
\end{document}
